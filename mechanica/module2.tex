\documentclass[12pt]{article}

\usepackage[dutch]{babel}
\usepackage{amsmath}

\title{Mechanica 2}
\author{Dries Van den Brande \and Andreas Declerck}

\begin{document}
    \maketitle

    \section{STATICA}
    \subsection{Stelsels glijdende vectoren}
    \begin{enumerate}
        \item \textbf{Wat is een centrale as? Stel zowel de vectoriële als de analytische vergelijking ervan op en bespreek het speciale geval van de vlakke en van de evenwijdige vectoren.} (HOOFDVRAAG)
        \item \textbf{Wat is de reductie van Poinsot?} (HOOFDVRAAG)
        \item \textbf{Geef de grafische constructie van de centrale as en de bijhorende redeneringen.} (HOOFDVRAAG)
        \item Teken en leg de verschillende types vectoren uit.
        \item Defineer: Algemene resultante. Welk soort vector is de algemene resultante? Leg uit.
        \item Defineer: Totaal moment in een punt. Welk soort vector is het totaal moment? Leg uit.
        \item Defineer: Gelijkwaardige stelsels, equipollent. Wat is een gevolg? Wat zijn de voorwaarden van de statica?
        \item Defineer: Nulstelsel.
        \item Wanneer reduceren we een vectorstelsel? Wat is het reduceren van een vectorstelsel?
        \item Wat zijn de elementaire bewerkingen dat we kunnen doen op de gelijkwaardige stelsels? Leg ze uit.
        \item Defineer: Koppel.
        \item Bereken het totaal moment van een koppel in een willekeurig punt P. Wat leer je hieruit?
        \item Wat zijn de eigenschappen van een koppel?
        \item \textbf{Bewijs de formule die de variantie van het moment van een punt naar een ander punt geeft en bespreek/bewijs de eigenschappen} (HOOFDVRAAG)
        \item Kan je de definitie van gelijkwaardige stelsels herzien met de nieuwgeleerde informatie in je hoofd? (TIP: ja) Geef een voorbeeld.
        \item Defineer: Invariante van een vectorstelsel. Hoeveel invarianten hebben stelsels glijdende vectoren? Welke zijn deze dan?
        \item Classificeer gelijkwaardige stelsels op basis van invarianten. Waarom is dit?
        \item 
    \end{enumerate}
    \subsection{Het massamiddelpunt}
    \begin{enumerate}
        \item \textbf{} (HOOFDVRAAG)
        \item 
    \end{enumerate}
    \subsection{Statica}
    \begin{enumerate}
        \item \textbf{} (HOOFDVRAAG)
        \item                                                                                                                                                                                   
    \end{enumerate}

\end{document}

\documentclass[12pt]{article}

\usepackage[dutch]{babel}
\usepackage{amsmath}

\title{Mechanica 2}
\author{Dries Van den Brande \and Andreas Declerck}

\begin{document}
    \maketitle

	\section{Hoofdstuk 1: Hoofdwet van de toestands vergelijking}%
	\label{sec:Hoofdstuk_1}
	\begin{enumerate}
		\item Geef de definitie van \emph{thermodynamica}.
		\item Geef de definitie van het \emph{systeem}.
		\item Definieer de \emph{omgeving van een systeem} en in de context van de ingenieur.
		\item Geef 2 onderverdelingen van systemen (1 met 3 soorten systemen en 1 met 2 soorten systemen)
		\item Geef 3 thermodynamische systemen.
		\item Geef de definitie van \emph{intensieke variabelen}?
		\item Geef de definitie van \emph{extensieve variabelen}.
		\item Geef de definitie van een \emph{systeem bij evenwicht}.
		\item Geef de \emph{nulde hoofdwet}.
		\item Geef de \emph{wet van Boyle}.
		\item Geef de \emph{wet van Charles \& Gay-Lussac}.
		\item Geef de \emph{ideale gaswet}.
		\item Geef de \emph{symbolische definitie van de temperatuur}.
		\item Geef de definitie van de \emph{Kelvin}.
		\item Geef de \emph{wet van Dalton}.
		\item Geef de \emph{compressibilitietsfactor Z}.
		\item Geef 2 oorzaken voor de afwijking van re\"ele gassen ten opzichte van ideale gassen.
		\item Wat gebeurt er als $T$ laag genoeg is?
		\item Geef de \emph{toestandsvergelijking van KamerLingen-Onnes} of de \emph{viriaal vergelijking}.
		\item Wat is de vergelijking van de \emph{Boyle temperatuur}?
		\item Definieer \emph{eigen volume}.
		\item Beschrijf in eigen woorden het \emph{PVT fasen diagram} en de projecties op het PT en PV vlak, verduidelijk.
		\item Wat zijn \emph{tielines} in het PV-vlak.
		\item Wat is \emph{co-extentielijn curve}.
		\item Wat kun je zeggen over de isothermen in de co-extensie zone?
		\item Definieer \emph{isotherme compressibiliteit}.
		\item Geef de \emph{Van der Waals vergelijking}.
		\item Duid in het PV-vlak de \emph{spinodaal} aan.
		\item Duid in het PV-vlak de \emph{binodaal} aan.
		\item Bespreek het princiepe \emph{corresponding states}.
		\item Bespreek de \emph{eerste hoofdwet}.
		\item Definieer de \emph{totale arbeit bij eindige verandering}.
		\item Wat is een \emph{isobaar}?
		\item Wat is een \emph{isochoor}?
		\item Bespreek de relatie tussen \emph{warmte en mechanische arbeit} door middel van gebruik te maken van de \emph{experimenten van Joule}.
		\item Definieer een \emph{adiabatisch proces}.
		\item Wanneer wordt het verschil in interne energie 0 ($\Delta U = 0 $)
		\item Bespreek de arbeit bij compressie bij  $T=cte$, wanneer is deze het meest effici\"ent?
		\item Bespreek de arbeit bij expantie bij  $T=cte$, wanneer is deze het meest effici\"ent?
		\item Definieer een \emph{reversiebel proces}.
		\item Geef 2 karakteristieke verschillen tussen een compressie/expantie in 1 stap en compressie/expantie in  $\infty$ stappen.
		\item Definieer de warmte capaciteit.
	\end{enumerate}
\end{document}

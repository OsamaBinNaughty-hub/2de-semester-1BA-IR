\documentclass[12pt]{article}

\usepackage[dutch]{babel}
\usepackage{amsmath}

\begin{document}
    \title{Analyse 2}
    \author{Dries Van den Brande \and Andreas Declerck}

    \maketitle

    \section{Milieuproblemen en Historische achtergrond}
    \subsection{Socio-economische data}
    \subsubsection*{Doel}
    \begin{itemize}
        \item Wat is het doel van dit hoofdstuk?
    \end{itemize}

    \subsubsection*{Inleiding}
    \begin{itemize}
        \item Wat waren de aanleidingen voor het ontstaan van de milieubewegingen?
        \item Wat is de \textbf{SOE}? Wanneer werd dit opgericht? Wat is de oorzaak van dit ?
        \item Wat is de \textbf{Conference on Human Enviroment}? Nadelen? oplossing van deze nadelen? Acties genomen?
        \item Wat zijn belangrijke gebeurtenissen in \textbf{1972}? Leg uit.
        \item Wat zijn belangrijke gebeurtenissen in \textbf{1973}? Leg uit.
        \item Wat zijn de 2 drivers van het leefmilieuproblematiek? geef voorbeelden van beide.
    \end{itemize}

    \subsubsection*{Menselijke ontwikkeling}
    \begin{itemize}
        \item Na welke gebeurtenis was "business booming" ? Maar.... ?
        \item Wat bedoelt men met kwaliteit en groei? Hoe moet men een bevolkingsgroei afremmen ?
        \item Hoeveel percent van de zwangerschappen is ongewenst?
        \item Leg economische ontwikkeling uit. 
        \item Wat is The Slum Challenge? Leg uit.
    \end{itemize}

    \subsubsection*{Energie \textbf{(EXAMEN)}}
    \begin{itemize}
        \item Stijgt het totale energieverbuik nog? Maar...? 
        \item Stijgt het energieverbuik per capita?
        \item Hoeveel percent van het energieverbuik gaat naar personenwagens + vb?
        \item Hoeveel percent van het energieverbuik gaat naar alle transport?
        \item Waarom begonnen we naar alternatieve energiebronnen zoeken?
        \item Geef voorbeelden waarom energieverbuik een enorme impact heeft op het leefmilieu.
        \item Wat bedoelen we met "ton olie equivalent" ?
    \end{itemize}

    \subsubsection*{Ecolosche voetafdruk \textbf{(EXAMEN)}}
    \begin{itemize}
        \item Wat is de Ecolosche voetafdruk? Hoe berekenen we dit?
        \item Is de economische voetafdruk van de totale wereldbevolking groter of kleiner dan de biocapaciteit?
        \item Hoe beschrijf je visgrond en energiegrond in het kader van de ecologische voetafdruk?
        \item Hoe komt het dat als je je ecologische voetafdruk bepaalt via 2 verschillende websites, je niet steeds hetzelfde getal uitkomt
        \item Hoe komt het dat de oppervlakte die we nodig hebben om te leven groter is dan de beschikbare oppervlakte? Hoe groot zou de fout zijn?
        \item Hoeveel bedraagt de ecologische voetafdruk van de gemiddelde Belg?
    \end{itemize}
    

    \subsection{Land}
    \subsubsection*{Inleiding}
    \begin{itemize}
        \item Wat bedoelt men met "land" ?
        \item Hoeveel van het aardoppervlakte is besteed aan land? 
        \item Wat zijn de belangrijkste milieuproblemen van het land? Hoe worden ze veroorzaakt?
    \end{itemize}

    \subsubsection*{Landbouw}
    \begin{itemize}
        \item Waarom is landbouw een probleem geworden voor het milieu?
    \end{itemize}

    \subsubsection*{Degradatie}
    \begin{itemize}
        \item Wat is degradatie?
        \item Hoe komt het dat we degradatie hebben?
        \item Hoeveel percent van het land is gedegradeerd?
        \item Wat zijn de gevolgen van degradatie?
        \item Wat zijn de belangrijke degradatiemechanismen?
        \item Hoe leidt roofbouw tot landdegradatie?
    \end{itemize}

    \subsubsection*{Voedselpiramide}
    \begin{itemize}
        \item Van waar komt de meeste energie?
        \item Leg voedselpiramide uit op basis van energie? Is dit een efficiënt proces?
        \item Welk dier is het slechtst voor het milieu en waarom?
        \item Hoe verklaar je de verschillende opbrengsten voor woestijnen, estuaria, ...
    \end{itemize}

    \subsubsection*{Verwoestijning}
    \begin{itemize}
        \item Bestaad er toevallig een conferentie tegen verwoestijning? JA! de welke?
        \item Hoeveel percent van de drylands zijn gedegradeerd? Wat is de grootste redene?
    \end{itemize}

    \subsubsection*{Invloed van klimaatwijziging}
    \begin{itemize}
        \item Wat zijn de duidelijke bewijzen van de klimaatwijziging?
        \item Waarom is ontbossen/ontgronden slecht voor het milieu?
        \item Klimaatwijziging is zeer plaatsafhankelijk, wat zijn de duidelijke wijzigingen in Afrika, Z-Azië, N-Europa, poolgebieden, kleine eilanden?
    \end{itemize}

    \subsubsection*{Urbanisatie}
    \begin{itemize}
        \item Wat is urbanisatie?
        \item Is hier een wet voor in belgië? JA! welke?
    \end{itemize}

    \subsubsection*{Chemisch afval}
    \begin{itemize}
        \item Wat regelt de Basel conventie?
        \item Wat zijn POP's?
        \item Waarvoor gebruikt men schaliegas? Is dit giftig?
    \end{itemize}

    \subsubsection*{Maatregelen?}
    \begin{itemize}
        \item Benoem enkele maatregelen genomen door de VN.
    \end{itemize}

    \subsubsection*{Tegenstrijdigheden in landgebruik}
    \begin{itemize}
        \item Waarom is er een tegensrijd in de voedselveiligheid?
        \item Waarom is de vleesproductie slecht voor het milieu?
        \item Wat is de schaal voor voeding voor dieren VS mens?
        \item Hoeveel percent van al het voedsel gaat verloren?
        \item Wat bedoelt men met duurzame voeding?
        \item Noem en verklaar positieve en negatieve punten van biobrandstoffen.
        \item Waarom zou jij biobrandstof aanraden of afraden?
        \item Verklaar marktverstoring.
    \end{itemize}

    \subsubsection*{Europa}
    \begin{itemize}
        \item Wat zijn de belangrijste problemen in Europa?
        \item Hoeveel percent leeft op hoeveel percent van het land in Europa?
        \item Wordt pesticiden gesubsidieerd?
        \item Wat is de Ramsar conventie?
        \item Hoeveel percent van de wetlands is verdwenen?
        \item Is er degradatie door regenval? Hoeveel?
        \item Waarom is oververbouwing een probleem?
    \end{itemize}

    \subsubsection*{Poolgebieden}
    \begin{itemize}
        \item Wat gebeurt er in de poolgebieden in verband met het land?
    \end{itemize}


    \subsection{Bossen}
    \subsubsection*{Inleiding}
    \begin{itemize}
        \item 
    \end{itemize}

    \subsubsection*{Functies van bossen}
    \begin{itemize}
        \item 
    \end{itemize}

    \subsubsection*{Biodiversiteit}
    \begin{itemize}
        \item 
    \end{itemize}

    \subsubsection*{Bosbeschadiging}
    \begin{itemize}
        \item 
    \end{itemize}

    \subsubsection*{Europa}
    \begin{itemize}
        \item 
    \end{itemize}


    \subsection{Biodiversiteit}
    \subsubsection*{Inleiding}
    \begin{itemize}
        \item 
    \end{itemize}

    \subsubsection*{Porblemen}
    \begin{itemize}
        \item 
    \end{itemize}


    \subsection{Zoet water}
    \begin{itemize}
        \item 
    \end{itemize}


    \subsection{Marine systemen}
    \subsubsection*{Inleiding}
    \begin{itemize}
        \item 
    \end{itemize}

    \subsubsection*{Visvangst}
    \begin{itemize}
        \item 
    \end{itemize}

    \subsubsection*{Habitat}
    \begin{itemize}
        \item 
    \end{itemize}

    \subsubsection*{Europa}
    \begin{itemize}
        \item 
    \end{itemize}


    \subsection{Lucht en atmosfeer}
    \subsubsection*{Inleiding}
    \begin{itemize}
        \item 
    \end{itemize}

    \subsubsection*{Pollutie}
    \begin{itemize}
        \item 
    \end{itemize}

    \subsubsection*{Gevolgen}
    \begin{itemize}
        \item 
    \end{itemize}

    \subsubsection*{Politiek: principe van de vervuiler betaalt}
    \begin{itemize}
        \item 
    \end{itemize}

    \subsubsection*{Stratosferische ozonlaag}
    \begin{itemize}
        \item 
    \end{itemize}

    \subsubsection*{Broeikasgassen - Klimaatsverandering}
    \begin{itemize}
        \item 
    \end{itemize}

    \subsubsection*{Europa: toestand luchtkwaliteit}
    \begin{itemize}
        \item 
    \end{itemize}

    \subsubsection*{Noord-Amerika en Poolstreken}
    \begin{itemize}
        \item 
    \end{itemize}

    \subsubsection*{Lange afstandstransport polluenten}
    \begin{itemize}
        \item 
    \end{itemize}

    
    \subsection{Stedelijke gebieden}
    \begin{itemize}
        \item 
    \end{itemize}


    \subsection{Rampen}
    \subsubsection*{Natuurlijke rampen}
    \begin{itemize}
        \item 
    \end{itemize}

    \subsubsection*{Menselijk geïnduceerde rampen}
    \begin{itemize}
        \item 
    \end{itemize}


    \subsection{Kwetsbaarheid}
    \begin{itemize}
        \item 
    \end{itemize}


    \subsection{Lessen voor de toekomst}
    \begin{itemize}
        \item 
    \end{itemize}


\end{document}
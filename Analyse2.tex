\documentclass[12pt]{article}



\begin{document}
    \title{Analyse 2}
    \author{Dries Van den Brande \and Andreas Declerck}

    \maketitle
    \section*{Hoofdstuk 1: De lijnintegraal}
    \begin{enumerate}
        \item Wat zijn en bewijs de voorwaarden voor een bepaalde integraal afhankelijk van \textbf{vaste eindige grenzen}.
        \item Defineer \textbf{Veranderlijke grenzen} van de bepaalde integraal.
        \item Bewijs de formule van Leibnitz voor veranderlijke grenzen.
        \item Wat zijn \textbf{oneindige grenzen}?
        \item Waarom is \textbf{vraag 2} niet meer een sterke voorwaarde voor \textbf{oneindige grenzen}? Geef een tegenvoorbeeld en een oplossing.
        \item Defineer de uniforme versie van de oneigenlijke integraal. (uniforme convergentie)
        \item Wat is en bewijs de voorwaarde voor de bepaalde integraal afhankelijk met \textbf{oneindige grenzen}.
        \item Veralgemeen nu de voorwaarde voor de bepaalde integraal afhankelijk van een parameter.
        \item Veralgemeen de Rienmannintegraal.
        \item Bewijs dat de veralgemeende Rienmannintegraal convergeert naar de bepaalde integraal van $fg$ als de norm van P naar nul gaat.
        \item Defineer de lijnintegraal van $\vec{F}$ over continue boog AB
        \item Bewijs het verband tussen de lijnintegraal en de bepaalde integraal. Geldt deze formule als F en r geen vectoren zijn?
        \item Waarom blijft de lijnintegraal hetzelfde na keuze van een andere parametrizatie? Wat is het gevolg hiervan?
        \item Als de boog AB een interval op de x-as is. Wat is dat de Riemannsom?
    \end{enumerate}    
\end{document}
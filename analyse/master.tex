\documentclass[a4paper,12pt]{article}

\usepackage[dutch]{babel}
\usepackage{amsmath}

\title{Mechanica 2}
\author{Dries Van den Brande \and Andreas Declerck}

\setcounter{section}{6}

% Everything in this region will be overriden
\begin{document}

    \maketitle


    \section{De lijnintegraal}
    \begin{enumerate}
        \item Wat zijn en bewijs de voorwaarden voor een bepaalde integraal afhankelijk van \textbf{vaste eindige grenzen}.
        \item Defineer \textbf{Veranderlijke grenzen} van de bepaalde integraal.
        \item Bewijs de formule van Leibnitz voor veranderlijke grenzen.
        \item Wat zijn \textbf{oneindige grenzen}?
        \item Waarom is \textbf{vraag 2} niet meer een sterke voorwaarde voor \textbf{oneindige grenzen}? Geef een tegenvoorbeeld en een oplossing.
        \item Wat is en bewijs de voorwaarde voor de bepaalde integraal afhankelijk met \textbf{oneindige grenzen}.
        \item Veralgemeen nu de voorwaarde voor de bepaalde integraal afhankelijk van een parameter.
        \item Veralgemeen de Rienmannintegraal.
        \item Bewijs dat de veralgemeende Rienmannintegraal convergeert naar de bepaalde integraal van $fg$ als de norm van P naar nul gaat.
        \item Defineer de lijnintegraal van $\vec{F}$ over continue boog AB
        \item Bewijs het verband tussen de lijnintegraal en de bepaalde integraal. Geldt deze formule als F en r geen vectoren zijn?
        \item Waarom blijft de lijnintegraal hetzelfde na keuze van een andere parametrizatie? Wat is het gevolg hiervan?
        \item Als de boog AB een interval op de x-as is. Wat is dat de Riemannsom?
        \item Waarom kan men niet zomaar zeggen dat volgende stelling altijd waar is (zonder bewijs)? Met andere woorden waarom hieraan een gans hoofdstuk wijden?
            $$ \frac{d}{dy}\int_a^b f(x,y)dx = \int_a^b \frac{\partial}{\partial y}f(x, y)dx $$
        \item Geef de eigenschappen van de lijnintegraal.
        \item Wat is nu, na de kennis uit de eigenschappen, de voorwaarde om de lijnintegraal te bereken?
        \item Aan wat is de lijnintegraal van een gesloten kromme $\Gamma$ afhankelijk? Geef ook de notatie.
        \item Geef en bewijs de voorwaarde wanneer een lijnintegraal onafhankelijk is van de gekozen kromme (Grondstelling van de lijnintegraal). Wanneer is deze gelijk aan nul?
    \end{enumerate}

    
    \section{De dubbele integraal}
    \begin{enumerate}
        \item Definieer de dubbele integraal en geef enkele eigenschappen
        \item Wat stelt volgende integraal voor? Indien $f(x,y) \geq 0$ voor $(x,y) \in G$
            $$ \iint_G f(x,y)dO $$
        \item Wat bedoelt men met regelmatig tov de x- of y-as?
        \item Bewijs de formule van de dubbele integraal over een rechthoek.
        \item Geef en leidt de formule van Guldin af.
        \item Geef en bewijs de formule van Green-Riemann
        \item Wat zijn de voorwaarden opdat volgende vgl klopt?
            $$ \iint_G f(x,y)dxdy = \iint_g f(\phi(u,v), \psi(u,v)) \left|\frac{\partial(\phi, \psi)}{\partial(u,v)}\right| dudv $$
    \end{enumerate}

    \section{De oppervlakte\"integraal}
    \begin{enumerate}
        \item Hoe bereken je de oppervlakte van een parallellogram?
        \item Hoe bereken je de oppervlakte van een oppervlak? Leid formule af.
        \item Geef en toon de formule van oppervlakteïntegraal van f over S aan.
        \item Geef en toon de tweede formule van Guldin aan.
        \item Defineer de flux van een vectorfunctie $\vec{v}$ door een geörienteerd oppervlak S.
        \item Geef de formules van de 3 differentiaaloperatoren.
        \item Bewijs de formule van Stokes.
	\item Wanneer kiest men het beste voor de dubbele integraal met het vectorproduct over de formule met de jocobiaan?
    \end{enumerate}
    

    \section{De drievoudige integraal}
    \begin{enumerate}
        \item Defineer de drievoudige integraal over een balk.
        \item Wat zijn de voorwaarden opdat volgende vgl klopt?
        $$\iiint_g f(x,y,z)dxdydz = \iiint_g f(\Phi (u,v,w), \Psi (u,v,w), \rho (u,v,w)) \left|\frac{\partial(\Phi, \Psi, \rho)}{\partial(u,v,w)}\right| dudvdw$$
        \item 
    \end{enumerate}

    \section{Numerieke reeksen}
    

    \section{Reeksen van functies}
    

    \section{Differentiaalvergelijkingen}
    \begin{enumerate}
        \item Defineer: differentiaalvergelijking / orde van een diff.vgl. / oplossing van een diff.vgl. / Integraalkromme van een diff.vgl. / Normale diff.vgl.
        \item Wat leer je uit de oplossingen van de diff.vgl. van: $y'=g(x)$ / $y'=y$ / $y"=0$ / $y"+y=0$. Welke terminologie zullen we invoeren? Leg deze ook uit.
    \end{enumerate}
    
\end{document}
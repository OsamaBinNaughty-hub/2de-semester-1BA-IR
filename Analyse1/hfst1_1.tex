\documentclass[12pt]{article}

\usepackage[dutch]{babel}
\usepackage{amsmath}
\usepackage{amssymb}
\usepackage{physics}

\title{Formularium}
\author{Dries Van den Brande \and Andreas Declerck}

\newcommand{\R}{\mathbb{R}}
\newcommand{\Z}{\mathbb{Z}}


\newcommand{\R}{\mathbb{R}}
\newcommand{\Q}{\mathbb{Q}}

\begin{document}
    \maketitle

    \section{Reële getallen}
    \begin{enumerate}
        \item Defineer: Deelverzamelingen / Unie / Doorsnede / Verschil / Product van een verzameling.
        \item Bewijs: $A \cup \emptyset = A$ / $A \cup B = B \cup A$ / $A \cup (B \cup C) = (A \cup B) \cup C$ / $A \cup A = A$
        \item Defineer: unie van een oneindige rij verzamelingen.
        \item Bewijs: $A \cap \emptyset = \emptyset$ / $A \cap B = B \cap A$ / $A \cap (B \cap C) = (A \cap B)\cap C$ / $A \cap A = A$ / $A \cup (B \cap C) = (A \cup B) \cap (A \cup C)$ / $A \cap (B \cup C) = (A \cap B) \cup (A \cap C)$ / $A \subset B \Leftrightarrow A \cup B = B$
        \item Defineer: doorsnede van een oneindige rij verzamelingen.
        \item Bewijs: Er bestaat geen rationaal getal waarvan het kwadraat gelijk is aan 2.
        \item Aan welke axioma moet een verzameling $\R$ voldoen?
        \item Wat is het axioma van Archimedes? Waarom is deze interessant in dit hoofdstuk?
        \item Defineer: Axioma van de volledigheid.
        \item Defineer: absolute waarde van een $\R$.
        \item Bewijs: $\mid x+y \mid \; \leq \; \mid x \mid + \mid y \mid$ 
        \item Defineer: vervolledigde reële rechte. Wat zijn de eigenschappen hiervan?
    \end{enumerate}
\end{document}

\documentclass[12pt]{article}

\usepackage[dutch]{babel}
\usepackage{amsmath}

\title{Mechanica 2}
\author{Dries Van den Brande \and Andreas Declerck}

\begin{document}
\date{}    
\maketitle
    \section*{Hoofdstuk 4: Determinanten}
    Vetgedrukte vragen zijn vragen gesteld in vorige examens.
    \setcounter{section}{4}
    \subsection*{Te bewijzen stellingen}
    \begin{enumerate}
        \item \textbf{(4.1.1)} Voor elke verzameling $A$ is $(S(A), \circ)$ een groep. Deze groep wordt de symmetrische groep van de verzameling $A$ genoemd.
        \item \textbf{(4.1.2)} Onderstel dat $A$ een eindige verzameling is, en dat $\#(A)=n$. Dan is $\#(S(A))=n!$.
        \item \textbf{(4.1.4)} Onderstel dat $A$ een eindige verzameling is. Dan geldt: $$\forall \sigma \in S(A), \forall a \in A, \exists m \geq 1: \sigma^m(a)=a$$
        \item \textbf{(4.1.5)} Elke permutatie van een eindige verzameling A kan geschreven worden als een samenstelling van verwisselingen.
        \item \textbf{(4.1.7)} Onderstel dat $a,b \in A$ en $\sigma \in S(A)$. Dan geldt dat $$\varepsilon([a,b]\circ \sigma) = -\varepsilon(\sigma)$$
        \item \textbf{(4.1.8)} Als $\sigma \in S(A)$ kan geschreven worden als de samestelling van $p$ verwisselingen, dan is $$\varepsilon(\sigma)=(-1)^p$$ De afbeelding $\varepsilon:S(A) \rightarrow \{-1,1\}$ voldoet aan volgende eigenschap: $$\varepsilon(\sigma)=\varepsilon(\sigma)\varepsilon(\tau)$$ voor elke $\sigma,\tau \in S(A)$.
        \item \textbf{(4.2.2)} Onderstel dat de afbeelding $d:M_{n,n}(\mathbb{K}) \rightarrow \mathbb{K}$ multilineair en alternerend is. Dan gelden de volgende eigenschappen: \begin{enumerate}
            \item $d(A)$ verandert van teken als we twee kolommen met elkaar verwisselen: $$d(A_1 \ A_2 \ \dots \ A_j \ \dots \ A_n) = -d( A_1 \ A_2 \ \dots \ A_j \ \dots \ A_i \ \dots \ A_n)$$
            \item $d(A)$ verandert niet als we bij een kolom een lineaire combinatie van de andere kolommen optellen: $$d(A_1 \ A_2 \ \dots A_i+\sum_{j\neq 1}\alpha_j A_j \ \dots \ A_n)=d(A_1 \ A_2 \ \dots \ A_i \ \dots A_n)$$
            \item $d(A)=0$ als één van de kolommen van $A$ gelijk is aan $0$: $$d(A_1 \ A_2 \ \dots \ 0 \ \dots \ A_n)=0$$
            \item Als we de kolommen van $A$ permuteren, dan wordt $d(A)$ vermenigvuldigd met de pariteit van de toegepaste permutatie: $$d(A_{\sigma(1)} \ A_{\sigma(2)} \ \dots \ A_{\sigma(n)})= \varepsilon(\sigma)d(A_1 \ A_2 \ \dots \ A_n)$$ voor elke $\sigma \in S_n$.
            \item Als $rg(1)<n$ dan is $d(A)=0$.
        \end{enumerate}
        \item \textbf{\textbf{(4.2.4)} Er bestaat juist één determinantafbeelding $M_{n,n}(\mathbb{K}) \rightarrow \mathbb{K}$, en deze wordt gegeven door formule $$det(A)= \sum_{\sigma \in S_n} \varepsilon(\sigma)a_{\sigma(1)1}a_{\sigma(2)2} \dots a_{\sigma(n)n}$$}
        \item \textbf{\textbf{(4.2.5)} De determinant van een matrix is gelijk aan die van zijn getransponeerde: $$det(A)=det(A^t)$$}
        \item \textbf{(4.2.6)} De volgende eigenschappen gelden voor de rijen van een matrix: \begin{enumerate}
            \item De determinant is multilineair als functie van de rijen van $A$;
            \item $det(A)$ verandert van teken als we twee rijen met elkaar verwisselen;
            \item $det(A)$ verandert niet als we bij een rij een lineaire combinatie van de andere rijen optellen;
            \item $det(A)=0$ als een van de rijen van $A$ gelijk is aan $0$;
            \item Als we de rijen van $A$ permuteren, dan wordt $det(A)$ vermenigvuldigd met de pariteit van de toegepaste permutatie. 
        \end{enumerate}
        \item \textbf{(4.2.7)} Als $d:M_{n,n}(\mathbb{K}) \rightarrow \mathbb{K}$ een multilineaire alternerende afbeelding is, dan is $d=d(I_n)$ determinant.
        \item \textbf{\textbf{(4.2.8)} Voor $A,B \in M_{n,n}(\mathbb{K})$ geldt dat $$det(AB)=det(A)det(B)$$}
        \item \textbf{(4.2.9)} Voor een $n \times n$-matrix $A$ zijn de volgende eigenschappen equivalent: \begin{enumerate}
            \item $rg(A)=n$;
            \item $A$ is regulier;
            \item $det(A) \neq 0$.
        \end{enumerate}
        \item \textbf{(4.2.10)} Onderstel dat $f$ een lineaire afbeelding is van een eindigdimensionale vectorruimte $V$ naar zichzelf. Als $E$ en $F$ twee basissen zijn van $V$, dan is $$det([f]_{E,E})=det([f]_{F,F})$$
        \item \textbf{(4.3.1)} Voor elke $B \in M_{n,n}(\mathbb{K})$ geldt dat $$det \begin{pmatrix}
            1 & 0 \\
            0 & B
        \end{pmatrix}=det(B)$$
        \item \textbf{\textbf{(4.3.2)} Neem een $n \times n$-matrix $A$. De minor $A_{i,j}$ is dan de determinant van de matrix die ontstaat door uit $A$ de i-de rij en de j-de kolom weg te laten: $$A_{i,j}=(-1)^{i+j} \begin{vmatrix}
            a_{11} & ... & a_{1,j-1} & a_{1,j+1} & \dots & a_{1n} \\
            \vdots & & \vdots & \vdots & & \vdots \\
            a_{i-1,1} & ... & a_{i-1,j-1} & a_{i-1,j+1} & \dots & a_{i-1,n} \\
            a_{i+1,1} & ... & a_{i+1,j-1} & a_{i+1,j+1} & \dots & a_{i+1,n} \\
            \vdots & & \vdots & \vdots & & \vdots \\
            a_{n,1} & ... & a_{n,j-1} & a_{n,j+1} & \dots & a_{nn} 
        \end{vmatrix}$$ De determinant van de matrix $A$ wordt gegeven door de formule $$det(A)=\sum_{i=1}^n a_{ij}A_{ij}= a_{1j}A_{1j}+\dots+a_{nj}A_{nj}$$ Deze formule wordt de ontwikkeling van $det(A)$ volgens de j-de kolom genoemd.}
        \item \textbf{(4.3.3)} Voor een $n \times n$-matrix $A$ geldt: $$det(A)=\sum_{i=1}^n a_{ij}A_{ij}= a_{i1}A_{i1}+\dots+a_{in}A_{in}$$ voor elke $i=1,\dots,n$. Deze formule wordt de ontwikkeling van $det(A)$ volgens de i-de rij genoemd.
        \item \textbf{\textbf{(4.3.4)} Voor elke $n \times n$-matrix $A$ geldt: $$adj(A)A=Aadj(A)=det(A)I_n$$}
        \item \textbf{(4.3.5)} Als $det(a)\neq 0$, dan is $$A^{-1}= \frac{1}{det(A)}adj(A)$$
        \item \textbf{\textbf{(4.3.7)} Als $A$ een reguliere $n \times n$-matrix is, dan wordt de unieke oplossing van het lineaire stelsel $AX=B$ gegeven door de formule $$x_i= \frac{1}{det(A)} \begin{vmatrix}
            a_{11} & ... & a_{1,i-1} & b_1 &a_{1,i+1} & \dots & a_{1n} \\
            a_{21} & ... & a_{2,i-1} & b_2 &a_{2,i+1} & \dots & a_{2n} \\
            \vdots & & \vdots & \vdots & & & \vdots \\
            a_{n,1} & ... & a_{n,i-1} & b_n & a_{n,i+1} & \dots & a_{nn} 
        \end{vmatrix}$$}
        \item \textbf{(4.3.9)} Beschouw een vierkante matrix $A \in M_{n,n}(\mathbb{K})$, en noteer $m_A$ voor de lineaire afbeelding $\mathbb{K}^n \rightarrow \mathbb{K}^n:X \mapsto AX$. Dan zijn de volgende eigenschappen equivalent: \begin{enumerate}
            \item $A$ is regulier;
            \item $A$ heeft een linksinverse $A':A'A=I_n$;
            \item $A$ heeft een rechtsinverse $A'':AA''=I_n$;
            \item $m_A$ is bijectief;
            \item $m_A$ is injectief;
            \item $m_A$ is surjectief;
            \item voor elke $B \in \mathbb{K}^n$ heeft het stelsel $AX=B$ een unieke oplossing;
            \item $AX=0$ heeft enkel $X=0$ als oplossing;
            \item voor elke $B \in \mathbb{K}^n$ heeft het stelsel $AX=B$ minstens één oplossing;
            \item $rg(A)=n$;
            \item de kolommen van $A$ zijn lineair afhankelijk;
            \item $rg(A^t)=n$;
            \item de rijen van $A$ zijn lineair onafhankelijk;
            \item $det(A) \neq 0$.
        \end{enumerate}
        \item \textbf{(4.3.10)} Als \begin{equation}
            \begin{cases} 
                a_{11}x_1+a_{12}x_2+\dots + a_{1n}x_n = b_1 \\ 
                a_{21}x_1+a_{22}x_2+\dots + a_{2n}x_n = b_2 \\ 
                \vdots \\
                a_{n1}x_1+a_{n2}x_2+\dots + a_{nn}x_n = b_n \\
                a_{n+1,1}x_1+a_{n+1,2}x_2+\dots+ a_{n+1,n}x_n = b_{n+1} 
            \end{cases}
            \end{equation}
            een oplossing heeft, dan is $$D=\begin{vmatrix}
                a_{11} & a_{12}... & a_{1n} & b_1 \\
                a_{21} & a_{22}... & a_{2n} & b_2 \\
                \vdots & \vdots & & \vdots & \vdots \\
                a_{n1} & a_{n2}... & a_{nn} & b_n \\ 
                a_{n+1,1} & a_{n+1,2}... & a_{n+1,n} & b_{n+1}
            \end{vmatrix}=0$$ Omgekeerd, indien $det(A) \neq 0$ en $D=0$, dan heeft (1) een unieke oplossing.
    \end{enumerate}
    \subsection{Permutaties}
    \begin{enumerate}
        \item Defineer de permutatie van een verzameling $A$. Defineer hierbij ook $S(A)$.
        \item Defineer de symmetrische groep. Welke stelling is hier aan verbonden?
        \item Hoeveel elementen zitten in de verzameling van de permutaties? Veralgemeen dit in een stelling.
        \item Hoe noteert men een permutatie en de verzameling van permutaties?
        \item Neem $A=\{1,2,3,4,5,6,7,8\}$ en permutatie $$\sigma \begin{bmatrix}
            1 & 2 & 3 & 4 & 5 & 6 & 7 & 8 \\
            3 & 4 & 5 & 6 & 7 & 2 & 1 & 8 
        \end{bmatrix}$$ Wat zijn de banen en fixpunten van deze permutaite?
        \item Welke stelling veralgemeent het fenomeen dat elke baan sluit?
        \item Hoe weten we dat een verzameling zich splitst in partities van banen van $\sigma \in S(A)$?
        \item Defineer Partitie van $A$.
        \item Defineer cyclische permutaties.
        \item Defineer verwisseling van $A$. Hoe kunnen we dit gebruiken in de definitie van de permutatie?
        \item Als een permutatie $\sigma$ kan geschreven worden als de samenstelling van een even/oneven aantal verwisselingen dan ... ? Wat is hiermee het probleem? Kunnen we de pariteit beter defineren?
        \item Wat gebeurt er met de pariteit na een verwisseling? Veralgemeen deze eigenschap in een stelling.
        \item Als $\sigma \in S(A)$ kan geschreven worden als de samestelling van $p$ verwisselingen, dan is ... ? De afbeelding $\varepsilon:S(A) \rightarrow \{-1,1\}$ voldoet aan volgende eigenschap: ... ?
        \item Defineer de alternerende groep.
    \end{enumerate}
    \subsection{De determinant van een vierkante matrix}
    \begin{enumerate}
        \setcounter{enumi}{14}
        \item Wat zijn de eigenschappen van $3\times 3$-matrices?
        \item Leid de formule van de determinant van een $3 \times 3$-matrix af.
        \item Defineer multilineaire en alternerende afbeeldingen.
        \item Welke eigenschappen gelden als de afbeelding $d:M_{n,n}(\mathbb{K})\rightarrow \mathbb{K}$ multilineair en alternerende is?
        \item \textbf{Defineer de determinantafbeelding.}
        \item Door welke formule wordt de determinantafbeelding gegeven? Hoeveel determinantafbeeldingen bestaan er?
        \item Wat is het verband van de determinant van een matrix en zijn getransponeerde? Welke eigenschappen kan je met dit verband bewijzen?
        \item $det(A\cdot B)= ?$ Welke lemma heeft men nodig voor dit bewijs?
        \item $\forall A \in M_{nn}(\mathbb{K}) gelden de eigenschappen: ... ?$ (3)
        \item Defineer de determinant van een lineaire afbeelding. Welke stelling heeft men hiervoor nodig?
    \end{enumerate}
    \subsection{De ontwikkeling van de determinant volgens een rij of een kolom}
    \begin{enumerate}
        \setcounter{enumi}{24}
        \item \textbf{Hoe bepaal je een determinant aan de hand van een ontwikkeling volgens de kolom? Geef hierbij ook het nodige Lemma.}
        \item Hoe bepaal je een determinant aan de hand van een ontwikkeling volgens de rij?
        \item \textbf{Defineer de geadjungeerde van een matrix.}
        \item \textbf{Wat is het verband tussen de determinant en de geadjungeerde van een matrix? Hoe kan men hiermee de inverse berekenen?}
        \item \textbf{Hoe bereken je $A^{-1}$? Waarom werkt deze methode?}
        \item Leg de regel van Cramer uit. Wanneer is dit niet handig?
        \item Geef de 14 karakteriserende eigenschappen voor reguliere matrices.
        \item Als \begin{equation}
            \begin{cases} 
                a_{11}x_1+a_{12}x_2+\dots + a_{1n}x_n = b_1 \\ 
                a_{21}x_1+a_{22}x_2+\dots + a_{2n}x_n = b_2 \\ 
                \vdots \\
                a_{n1}x_1+a_{n2}x_2+\dots + a_{nn}x_n = b_n \\
                a_{n+1,1}x_1+a_{n+1,2}x_2+\dots+ a_{n+1,n}x_n = b_{n+1} 
            \end{cases}
            \end{equation}
            een oplossing heeft, dan is ... ? Omgekeerd ... ?
    \end{enumerate}
\end{document}
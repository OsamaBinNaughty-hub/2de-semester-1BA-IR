\documentclass[12pt]{article}

\usepackage[dutch]{babel}
\usepackage{amsmath}

\title{Mechanica 2}
\author{Dries Van den Brande \and Andreas Declerck}

\begin{document}
\date{}    
\maketitle
    \section*{Hoofdstuk 3: Lineaire variëteiten en stelsels lineaire vergelijkingen}
    Vetgedrukte vragen zijn vragen gesteld in vorige examens.
    \setcounter{section}{3}
    \subsection*{Te bewijzen stellingen}
    \begin{enumerate}
        \item \textbf{(3.1.2.)} Neem $\vec{a},\vec{a}' \in V$ en $W,W' \leq V$. Dan geldt $$\vec{a}+W=\vec{a}'+W' \Leftrightarrow W=W' \,\, en \,\, \vec{a}'-\vec{a}\in W$$ "deelruimte $W$ is uniek bepaald door de lineaire variëteit L"
        \item \textbf{(3.1.3.)} Onderstel dat $f:V \rightarrow W$ een lineaire afbeelding is, en dat $\vec{b}\in Im(f)$. Dan is $f^{-1}\{\vec{b}\}$ een lineaire variëteiten in $V$.
        \item \textbf{(3.1.4.)} Onderstel dat $dim(v)=n$, en $L\subset V$ een lineaire variëteiten van dimesie $n-k$. Als $W$ een vectorruimte is van dimensie ten minste $k$, dan bestaat er een lineaire afbeelding $f:V\rightarrow W$, en $\vec{b}\in W$ zodat $L=f^{-1}(\vec{b})$.
        \item \textbf{(3.1.6.)} De samenstelling van twee affiene afbeeldingen is affien.
        \item \textbf{\textbf{(3.2.1.)} Beschouw $A\in M_{mn}(\mathbb{K})$, $B\in \mathbb{K}^m$, en het lineair stelsel $AX=B$. Dan geldt $$AX=B \,\, heeft \,\, oplossingen \Leftrightarrow rg(A)=rg(A|B)$$ $$AX=B \,\, heeft \,\, geen \,\, oplossingen \Leftrightarrow rg(A)+1=rg(A|B)$$ Als $rg(A)=rg(A|B)$ dan is $Opl(AX=B)$ een lineaire variëteit in $\mathbb{K}^n$, met dimensie $n-rg(A)$. Als $X_0$ een particuliere oplossing is, dan geldt bovendien $$Opl(AX=B)=X_0+Opl(AX=0)$$}
        \item \textbf{(3.2.2.)} Beschouw twee lineaire stelsels $AX=B$ en $CX=D$ van $m$-vergelijkingen in $n$-onbekenden. Indien de matrices $(A|B)$ en $(C|D)$ rijequivalent zijn. Dan is $$Opl(AX=B)=Opl(CX=D)$$
    \end{enumerate}
    \subsection{Lineaire variëteiten}
    \begin{enumerate}
        \item \textbf{Defineer de lineaire variëteit $L$.}
        \item Met welke stelling bewijs je dat deelruimte $W$ uniek bepaald is door de lineaire variëteiten $L$?
        \item Wat is de dimensie van een lineaire variëteit? Wat betekent het als deze dimensie 1 heeft?
        \item Defineer een hypervlak.
        \item Defineer de vectorvergelijking van een lineaire variëteit. Wat als $dim(W)=1$?
        \item Wat is de vergelijking van een vlak in een driedimensionale ruimte?
        \item Welke stellingen hebben we nodig om de vergelijking van het hypervlak te construeren?
        \item Defineer de vergelijking van het hypervlak.
        \item Defineer evenwijdige lineaire variëteiten.
        \item Defineer affiene afbeeldingen.
        \item \textbf{Is de kern van een lineaire variëteiten een lineaire variëteiten? Omgekeerd?}
        \item Wat gebeurt er met de affiene eigenschap als we 2 affiene afbeeldingen samenstellen?
    \end{enumerate}
    \subsection{Stelsels lineaire vergelijkingen}
    \begin{enumerate}
        \setcounter{enumi}{12}
        \item \textbf{Leg het verband tussen de rang van een matrix en de oplossing van een stelsel uit. Hoe kunnen we dit veralgemenen in een stelling?}
        \item Wat weten we van de oplossing van 2 lineare stelsels die rijequivalent zijn? Hoe kunnen we dit veralgemenen in een stelling?
        \item Leg Gauss eliminatie en Gaus-Jordan eliminatie uit.
        \item Leg uit hoe ik de inverse van een matrix bereken.
    \end{enumerate}
\end{document}
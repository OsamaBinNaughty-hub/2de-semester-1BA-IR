\documentclass[12pt]{article}

\usepackage[dutch]{babel}
\usepackage{amsmath}

\title{Mechanica 2}
\author{Dries Van den Brande \and Andreas Declerck}

\begin{document}
\date{}    
\maketitle
    \section*{Hoofdstuk 2: Lineaire Afbeeldingen en Matrices}
    Vetgedrukte vragen zijn vragen gesteld in vorige examens.
    \setcounter{section}{2}
    \subsection*{Te bewijzen stellingen}
    \begin{enumerate}
        \item \textbf{(2.1.3.)} Onderstel dat $f: V \rightarrow W$ lineair is. Dan is $f(\vec{0})=\vec{0}$.
        \item \textbf{(2.1.4.)} Onderstel dat $f: V \rightarrow W$ en $g: W \rightarrow X$ lineaire afbeeldingen zijn. Dan is $g \circ f: V \rightarrow X$ ook een lineaire afbeelding.
        \item \textbf{(2.1.5.)} Onderstel dat $f: V \rightarrow W$ lineaire afbeeldingen zijn, en dat $\alpha \in \mathbb{K}$. Dan zijn de afbeeldingen $f+g$ en $\alpha f$, gedefineerd door $$f+g: V \rightarrow W: \vec{a} \mapsto f(\vec{a}) + g(\vec{a})$$ en $$\alpha f : V \rightarrow W: \vec{a} \mapsto \alpha f(\vec{a})$$ ook lineaire afbeeldingen.
        \item \textbf{(2.1.6.)} Onderstel dat $V=V_1 \oplus V_2$ de directe som van twee deelruimten van een vectorruimte is. Dan is de afbeelding $f:V \rightarrow V_1$ gedefineerd door $$f(\vec{v})=\vec{v}_1$$ indien $\vec{v}=\vec{v}_1 + \vec{v}_2$, met $\vec{v}_1 \in V_1$ en $\vec{v}_2 \in V_2$ een lineaire afbeelding die voldoet aan de eigenschap $$f\circ f = f$$
        \item \textbf{(2.2.2.)} $Ker(f)$ is een deelruimte van $V$ en $Im(f)$ is een deelruimte van $W$.
        \item \textbf{\textbf{(2.2.3.)} Voor een lineaire afbeelding $f: V \rightarrow W$ zijn de volgende uitspraken equivalent: \begin{enumerate}
            \item f is injectief;
            \item $Ker(f) = \{\vec{0}\}$;
            \item Het beeld onder $f$ van een stel lineair onafhankelijke vectoren in $V$ is lineair onafhankelijk in $W$.
        \end{enumerate}}
        \item \textbf{\textbf{(2.2.4.)} Voor een lineaire afbeelding $f: V \rightarrow W$ zijn de volgende uitspraken equivalent: \begin{enumerate}
            \item f is surjectief;
            \item $Im(f) = W$;
            \item Als $vect(A)=V$, dan is $vect(f(A)) = W$.
        \end{enumerate}}
        \item \textbf{\textbf{(2.2.6.)} Als $f: V \rightarrow W$ een isomorfisme is, dan is het beeld van elke basis van $V$ een basis van $W$. Bijgevolg hebben (eigendimensionale) isomorfe vectorruimten dezelfde dimensie.}
        \item \textbf{\textbf{(2.2.7.)} Als $f: V \rightarrow W$ een isomorfisme is, dan is ook de inverse afbeelding $f^{-1}: W \rightarrow V$ een isomorfisme.}
        \item \textbf{(2.2.8.)} Als $f: V \rightarrow W$ en $g: W \rightarrow X$ isomorfismen zijn, dan is ook $g \circ f: V \rightarrow X$ een isomorfisme.
        \item \textbf{(2.2.9.)} Voor elke vectorruimte $V$ is $1_V : V \rightarrow V$ een isomorfisme.
        \item \textbf{(2.2.10.)} Voor elke n-dimensionale vectorruimte V met geordende basis $E$ geldt dat $$[\bullet]_E:V \rightarrow \mathbb{K}^n : \vec{v}\mapsto [\vec{v}]_E$$ een isomorfisme is. 
        \item \textbf{(2.2.11.)} Als $dim(V) = dim(W)=n$, dan is $$V \simeq W \simeq \mathbb{K}^n.$$
        \item \textbf{\textbf{(2.2.12.)} Tweede dimensiestelling.}
        \item \textbf{(2.2.13.)} (Stelling van het alternatief) Onderstel dat $V$ en $W$ vectorruimten zijn met dezelfde dimensie, en dat $f:V \rightarrow W$ een lineaire afbeelding is. Volgende eigenschappen zijn equivalent: \begin{enumerate}
            \item $f$ is een isomorfisme;
            \item $f$ is injectief;
            \item $f$ is surjectief.
        \end{enumerate}
        \item \textbf{(2.3.1.)} $Hom_{\mathbb{K}}(V,V)$ is een deelruimte van $W^V$.
        \item \textbf{(2.4.1.)} $M_{m,n}(\mathbb{K})$ is ee vectorruimte, en de afbeelding $$[\bullet]_{F,E}: Hom_{\mathbb{K}}(V,W) \rightarrow M_{m,n}(\mathbb{K})$$ is een isomorfisme. Bovendien geldt dat $$dim(M_{m,n}(\mathbb{K})) = dim(Hom_{\mathbb{K}}(V,W))=mn.$$ 
        \item \textbf{(2.5.2.)} Als $V$,$W$ en $X$ eindigdimensionale vectorruimten zijn basissen respectievelijk $E$, $F$ en $G$, en $f:V \rightarrow W$, $g:W \rightarrow X$ lineaire afbeeldingen zijn, dan geldt dat de matrix van de samenstelling $g \circ f$ gegeven wordt door het product van de matrices van $g$ en van $f$: $$[g \circ f]_{G,E} = [g]_{G,F}[f]_{F,E}$$
        \item \textbf{(2.5.3.)} Het product van matrices is associatief.
        \item \textbf{(2.5.6.)} Als $A\, , \, B \in M{m,n}(\mathbb{K})$ regulier zijn, dan is ook $AB$ regulier. Verder is $$(AB)^{-1} = B^{-1}A^{-1}$$
        \item \textbf{(2.5.7.)} Als $A \in M{m,n}(\mathbb{K})$ regulier is en $B$, $C \in M{m,n}(\mathbb{K})$, dan geldt $$AB=AC \Rightarrow B=C$$ Wat gebeurt er als $A$ singulier is?
        \item \textbf{(2.6.1.)} Onderstel dat $E$ en $E'$ twee basissen zijn voor een eindigdimensionale vectirruimte $V$, en dat de elementen $m_{ij}$ van de matrix $M$ gegeven worden door de formule $$\vec{e}_j\,' = \sum^n_{i=1} m_{ij}\vec{e}_i$$ Dan is $M$ regulier en voor elke $\vec{v}\in V$ hebben we dat $$[\vec{v}]_E = M[\vec{v}]_{E'} \,\, en \,\, [\vec{v}]_{E'}= M^{-1}[\vec{v}]_E$$
        \item \textbf{(2.6.3.)} Beschouw een lineaire afbeelding $f:V \rightarrow W$, en de bassisen $E$ en $E'$ van $V$ en $F$ en $F'$ van $W$. Als de matrices $M$ en $N$ gegeven zijn door de formules $$\vec{e}_j\,' = \sum^n_{i=1} m_{ij}\vec{e}_i \,\, en \,\, \vec{f}_k\,' = \sum^m_{l=1} n_{lk}\vec{f}_l$$
        
        \item \textbf{(2.7.2.)} De rang van een matrix $A$ is het maximaal aantal lineair onafhankelijke kolommen van $A$.
        \item \textbf{(2.7.3.)} Onderstel dat $f:V \rightarrow W$ een lineaire afbeelding is en dat $E$, $F$ basissen zijn voor respectievelijk $V$ en $W$. Als $A=[f]_{F,E}$, dan is $$rg(A)=rg(f)$$
        \item \textbf{(2.7.4.)} Beschouw een $m \times n-matrix \, \, A$, een reguliere $n\times n-matrix \, \, M$, en een reguliere $m\times m-matix \,\, N$. Dan geldt $$rg(N^{-1}AM)=rg(A)$$
        \item \textbf{(2.7.6.)} voor elke $m \times n-matrix \,\, A$ en $n \times p - matrix \,\, B$ geldt: $$(AB)^t = B^tA^t$$
        \item \textbf{(2.7.7.)} De getransponeerde van een reguliere matrix is ook regulier.
        \item \textbf{\textbf{(2.7.8.)} De rang van elke matrix $A$ is gelijk aan die van zijn getransponeerde.}
        \item \textbf{(2.7.9.)} Onderstel dat $A\in M_{m,n}(\mathbb{K})$. Dan is $A$ regulier als en slechts als $rg(A)=n$.
        \item \textbf{(2.7.12.)} De rang van een matrix in rij (kolom) echelon vorm is gelijk aan het aantal van nul verschillende rijen (kolommen).
        \item \textbf{(2.7.14.)} Onderstel dat de matrix $B$ met rijen $S_1,\dots,S_n$ zijequivalent is met de matrix $A$ met rijen $R_1,\dots,R_n$. Dan is $$vect(S_1,\dots,S_n)=vect(R_1,\dots,R_n)$$ en bijgevolg is $$rg(A)=rg(B)$$ Geldt dit ook voor kolomequivalente matrices?
        \item \textbf{(2.7.15.)} Elke matrix $A$ is rijequivalent (kolomequivalent) met een matrix in rij (kolom) echelon vorm.
        \item \textbf{(2.7.16.)} Elke matrix A is rijequivalent (kolomequivalent) met een matrix in gereduceerde rij (kolom) echelon vorm.
    \end{enumerate}
    \subsection*{Te begrijpen stellingen}
    \begin{enumerate}
        \item \textbf{(2.6.6.)} Beschouw een lineaire afbeelding $f:V \rightarrow W$, en basissen $E$ en $E'$ van $V$. Als de matrix $M$ gegeven wordt door de formule $$\vec{e}_j\,' = \sum^n_{i=1} m_{ij}\vec{e}_i$$ dan is $$[f]_{E',E'} = M^{-1}[f]_{E,E}M$$
    \end{enumerate}
        
    \subsection{Lineaire afbeeldingen}
    \begin{enumerate}
        \item Defineer de lineaire afbeelding en de homomorfisme. Geef de formele en 'eigen woorden' definitie.
        \item Welke voorwaarden moeten gelden zodat dit geldt: $$f(\vec{0})=\vec{0}$$
        \item Welke voorwaarden moeten gelden zodat dit geldt: $$g \circ f: V \rightarrow X \,\, is \,\, een \,\, lineaire \,\, afbeelding$$
        \item Onderstel dat $f: V \rightarrow W$ lineaire afbeeldingen zijn, en dat $\alpha \in \mathbb{K}$. Dan zijn de afbeeldingen $f+g$ en $\alpha f$, gedefineerd door ... ?
        \item Onderstel dat $V=V_1 \oplus V_2$ de directe som van twee deelruimten van een vectorruimte is. Dan is de afbeelding $f:V \rightarrow V_1$ gedefineerd door $$...?$$ indien $\vec{v}=\vec{v}_1 + \vec{v}_2$, met $\vec{v}_1 \in V_1$ en $\vec{v}_2 \in V_2$ een lineaire afbeelding die voldoet aan de eigenschap $$...?$$ Welke nieuwe termen kunnen we invoegen wat hier gebeurt?
    \end{enumerate}
    \subsection{Kern en beeld van een lineaire afbeelding}
    \begin{enumerate}
        \setcounter{enumi}{5}
        \item \textbf{Defineer de kern en het beeld van een lineaire afbeelding $f$.}
        \item Waarvan zijn de kern en het beeld de deelruimtes?
        \item Wat is een injectieve en surjectieve afbeelding?
        \item Geef het verband met injectie en surjecties met de kern en het beeld.
        \item Wat is een bijectieve afbeelding?
        \item \textbf{Defineer een isomorfisme. Wat betekent isomorf?}
        \item Welke soort vectorruimten hebben dezelfde dimensie?
        \item Is de inverse afbeelding van een afbeelding een isomorfisme?
        \item Wat zijn de twee eigenschappen van isomorfisme?
        \item Defineer reflectiviteit en transitiviteit.
        \item Defineer: \begin{enumerate}
            \item geordende basissen
            \item coördinaten van ... tegen over ... 
            \item coördinatenafbeelding tegen over ... 
        \end{enumerate}
        \item Wat is een eigenschap van de coördinatenafbeelding?
        \item Wat is een voldoende voorwaarde om te gelden: $V\cong W \cong \mathbb{K}^n$
        \item Wat zegt de tweede dimensiestelling?
        \item Wat zegt de stelling van het alternatief?
        \item Defineer: \begin{enumerate}
            \item epimorfisme
            \item monomorfisme
            \item endomorfisme
            \item automorfisme
            \item lineaire vorm 
        \end{enumerate}
    \end{enumerate}
    \subsection{De vectirruimte van de lineaire afbeeldingen}
    \begin{enumerate}
        \setcounter{enumi}{21}
        \item Defineer de vectorruimte van de lineaire afbeelding.
        \item Wat is de 3e bewerking van de vectorruimte van de lineaire afbeelding.
        \item Wat zijn de eigenschappen van de vectorruimte van de lineaire afbeelding? Hoe kan men deze eigenschappen samenvatten in 1 term?
    \end{enumerate}
    \subsection{Matrices}
    \begin{enumerate}
        \setcounter{enumi}{24}
        \item \textbf{Bouw de gedachtegang van een $m \times n$-matrix op.}
        \item Wat zijn de eigenschappen van een $m \times n$-matrix?
        \item Schrijf de matrices van de volgende lineaire afbeeldingen: \begin{enumerate}
            \item $f: \mathbb{R} \rightarrow \mathbb{R}^n$
            \item $f: \mathbb{R}^n \rightarrow \mathbb{R}$
            \item $f: \mathbb{R}^2 \rightarrow \mathbb{R}^2$
        \end{enumerate}
    \end{enumerate}
    \subsection{Het product van matrices}
    \begin{enumerate}
        \setcounter{enumi}{27}
        \item Defineer het product van matrices
        \item Wat is de voorwaarde om 2 matrices te vermenigvuldigen?
        \item Wanneer is BA en AB gedefineerd? (als A en B matrices zijn)
        \item Defineer de eenheidsmatrix.
        \item Defineer de inverse matrix.
        \item Defineer de reguliere en singuliere matrix.
        \item Is elke matrix regulier? Waarom wel/niet ?
        \item Wat zijn de eigenschappen van reguliere matrices?
    \end{enumerate}
    \subsection{Verandering van basis}
    \begin{enumerate}
        \setcounter{enumi}{35}
        \item \textbf{Bouw de gedachtegang van een verband tussen de coördinaten van een vector tegen over 2 verschillende basissen van dezelfde vectorruimte. Welke stelling heb je hiermee bewezen?}
        \item Kunnen we de verandering van een basis ook anders bekijken?
        \item \textbf{Bouw de overgangsformules voor de matrix van een lineaire afbeelding op. Welke stelling hebben we hiermee bewezen? (AKA Hoe verandert de matrix als de basis verandert?)}
        \item Wat gebeurt er met de overgansformule voor de matrix van een endomorfisme?
    \end{enumerate}
    \subsection{De rang van een matrix}
    \begin{enumerate}
        \setcounter{enumi}{39}
        \item \textbf{Defineer de rang van een lineaire afbeelding}
        \item \textbf{Defineer de rang van een $m \times n$-matrix A. Kan je dit ook anders defineren?}
        \item Hoe kan dit kloppen? $$rg(A)=rg(f)$$
        \item Hoe kan dit kloppen? $$rg(N^{-1}AM)=rg(A)$$
        \item $(AB)^t = ?$
        \item Als ik een reguliere matrix transponeer, wat is er interessant aan deze matrix?
        \item $rg(A^t)= ?$
        \item Hoe kan dit kloppen? $$A \,\, regulier \Leftrightarrow rg(A)=n$$
        \item Defineer (gereduceerde) rij echelon vorm.
        \item De rang van een matrix in rij(kolom) echelon vorm is gelijk aan ... ?
        \item Defineer een elementaire rij(kolom) operatie.
        \item Defineer rijequivalentie (kolomequivalentie).
        \item Wat is er interessant aan de rang van 2 matrixen die rijequivalent zijn?
        \item Elke matrix A is rijequivalent (kolomequivalent) met ... ? 
    \end{enumerate}
\end{document}
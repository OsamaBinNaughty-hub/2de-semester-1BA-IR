\documentclass[a4paper,12pt]{article}

\usepackage[dutch]{babel}
\usepackage{amsmath}
\usepackage[version=4]{mhchem}

\begin{document}
    \title{Chemie I deel 2}
    \author{Dries Van den Brande \and Andreas Declerck}

    \maketitle
    \setcounter{section}{6}
    \section{Aggregatietoestanden en toestandsovergangen. Fasenleer}

    \subsection*{Aggregaattoestanden en toestandovergangen bij zuivere stoffen}
    \begin{enumerate}
        \item Wat zijn de 3 voornaamste aggregaattoestanden?
        \item Leg uit: gecondenceerde toestand.
        \item Warmte-inbreng in een zuivere stof resulteerd in?
        \item Faseovergang van een zuivere stof gebeurt bij een constante ... ? 
        \item Faseovergang van een zuivere stof is afhankelijk van?
        \item Wat zijn alle mogelijke toestandsovergangen?
        \item Waarom is de energie nodig om te sublimeren groter dan koken en veel groter dan smelten?
        \item Teken een T-enthalpie inbreng fasediagram van ijs-water-damp.
        \item Wat bepaalt de fasenregel van gibbs?
        \item Wat is een vrijhijdsgraad?
        \item Wat bedoeld men met het aantal onafhankelijke chemische componenten?
        \item De toestand in een chemisch systeem is in functie van?
        \item Wat is het aantal onafhankelijke chemische componenten bij heterogeen mengsel van \ce{H2O} en \ce{Hg}?
        \item Wat is het aantal onafhankelijke chemische componenten bij homogeen mengsel van \ce{H2O},\ce{NaCl} en ethanol?
        \item Wat is het aantal onafhankelijke chemische componenten bij \ce{N2 + 3H2O <=> 2NH3}?
        \item Geef de formule voor de faseregel van Gibbs en leg alle parameters uit.
        \item Wat is de vrijhijdsgraad voor zuiver kokend water?
        \item Wat is de vrijhijdsgraad voor kokend homogeen mensel van \ce{H20} en \ce{ethanol}?
        \item Wat is de vrijhijdsgraad voor smeltend hetoreen mengsel van \ce{Na} , \ce{K} en homogene smelt?
        \item Wat is dampdruk? Geef de notatie voor dampdruk.
        \item Hoe wordt vluchtigheid gedefini\"eerd?
        \item Wat is verzadigde damp?
        \item De snelheid van verdamping is ... ? Maak tekening.
        \item De snelheid van condensatie is ... ? Maak tekening.
        \item Wanneer is de snelheid van verdamping gelijk aan de snelheidd van condensatie?
        \item Waarom neemt de dampdruk toe na verhoging van de temperatuur? Maak tekening.
        \item Geef de verschillen tussen damp en koken op thermodynamisch niveau (2 zaken).
        \item Hoe heet de functie die dampdruk uitdrukt als functie van de temperatuur? Wat is hier opmerkelijk aan?
        \item Wat is het tripelpunt? Maak tekening. Wat is hier opmerkelijk aan?
        \item Wat is het kritisch punt? Maak tekening. Wat is hier opmerkelijk aan?
        \item Wat is subkritische vloeistof? Maak tekening. Wat is hier opmerkijk aan? Wat is een toepassing van dit spul, leg uit.
        \item Wat is de wet van Clapeyron-Clausius? Wat zijn de beperkingen hierop? 
        \item Hoe kan men experimenteel de kook enthalpie vinden?
        \item Waarom is de wet van Clapeyron-Clausius een semi-logaritmische grafiek?
        \item Leg uit: Wet van Trouton.
        \item Teken een fasediagram van zuivere stoffen. Waarom is \ce{H2O} een uitzondering?
        \item Teken een fasediagram van zuivere stoffen met allotopen. Geef een voorbeeld.
        \item Wat is het grote verschil tussen allotropen en gewone enkelvoudige stoffen m.b.t. het fasediagram?
        \item Wat bedoeld men met een meta-stabiele toestand? Geef een voorbeeld.
    \end{enumerate}
    \subsection*{Faseovergangen bij homogene mengsels van vloeistoffen}
    \begin{enumerate}
        \item Wat is het verschil tussen ideale en niet-ideale mengsels?
        \item Wat wordt er bedoeld met een limietgeval?
        \item Wat zijn colligatieve eigenschappen? Noem 4 verschillende colligatieve eigenschappen.
        \item Wat is colligatieve molaliteit? Geef de definitie en formule.
        \item Als de molariteit van \ce{NaCl} $0.01M$ is. Wat is de colligatieve molaliteit dan?
        \item Geef en leidt af: Wet van Raoult.
        \item Wat is de voorwaarde omdat men de wet van Raoult mag gebruiken?
        \item Leid formule af voor de dampdruk van zuivere oplossing A door aanwezigheid solutans B.
        \item Geef een wiskunige formule voor het berekenen van de molaliteit van B.
        \item Leg uit aan de hand van de wet van Raoult waarom men zout strooit in de winter?
        \item Leg osmotische druk uit. Geef voorbeelden.
        \item Wat gebeurt er als je zout of zuiver water zou drinken?
        \item Waarom willen we weten hoe kookprocessen werken van niet zuivere mengsels? Geef een voorbeeld.
        \item Leg uit hoe men aardolie destilleerd.
        \item Leid de formule van de dampdruk van een homogeen ideaal mengsel af + tekening ($p=f(x^{vl}_A)$)
        \item Stel vanuit de wet van Raoult de functie $ p = f(x_A^d) $ op. (homogeen ideaal mengsel)
        \item Wat is het verschil tussen $p=f(x^{vl}_A)$ en $ p = f(x_A^d) $. Teken deze 2 functies. Duid $x_A^d$ en $x^{vl}_A$ aan op de grafiek.
        \item Hoe ga je van een \{P,X\} naar een \{T,X\} diagram?
        \item Wat zijn de karakteristieken van een \{T,X\} diagram?
        \item Teken en leg uit wat er gebeurt bij het koken van een ideaal mengsel van 2 vloeistoffen zonder scheiding.
        \item Teken en leg uit wat er gebeurt bij een gefractioneerde destillatie van een ideaal homogeen mengsel van 2 vloeistoffen. Hoe verloopt de temperatuur bovenaan de destillatiekolom?
        \item Wat bedoelt men met de niet-idealiteit in een homogeen mengsel van 2 vloeistoffen? Wat is het resultaat dan is de \{T,X\} diagrammen dan?
        \item Wat is een azeotropisch mengsel? Geef een tekening. Wat is de vrijheidsgraad hiervan?
        \item Teken en leg uit wat er gebeurt bij een gefractioneerde destillatie van een azeotropisch mengsel.
    \end{enumerate}

    \subsection*{Faseovergangen bij heterogene mengsels}
    \begin{enumerate}
        \item Geef een voorbeeld van beperkte mengbaarheid. (vluchtige en minder vluchtige stoffen)
        \item Geef een voorbeeld van beperkte mengbaarheid met een endotherme vermenging. Leg uit + tekening. (Niet vluchtige stoffen)
        \item Leg uit waarom ,bij een gegeven temperatuur, de fasesamenstelling onafhankelijk is van de mengselsamenstelling (endotherme vermenging, Niet vluchtige stoffen). Maar...
        \item Leid de formules voor $X_A^{F1}$ en $X_A^{F2}$ af. (Niet vluchtige stoffen)
        \item Wat is de hefboomregel? Wat kan men hieruit afleiden? (Niet vluchtige stoffen)
        \item Stel dat $n_A^0 = 2.00mol$ en $n_B^0 = 2.50mol$ gemengd worden. Stel dat bij evenwicht, de fasen F1 en F2 de fasensamenstellingen $X_A^{F1} = 0.220$ en $X_A^{F2} = 0.890$. Wat zijn de molhoeveelheden van component A en B op de twee fasen F1 en F2. (Niet vluchtige stoffen)
        \item Teken het \{T,X\} diagram voor het limietgeval dat 2 vloeistoffen helemaal niet mengbaar zijn. (Niet vluchtige stoffen)
        \item Geef een voorbeeld van beperkte mengbaarheid met een exotherme vermenging. (Niet vluchtige stoffen)
        \item Geef een voorbeeld van een mengsel met gesloten 2 fasengebied. (Niet vluchtige stoffen)
        \item Wat is de formule voor de dampdruk van 2 volledig niet mengbare vloetstoffen? + tekening (Vluchtige stoffen)
        \item Wat is er opmerkelijk aan de kooktemperatuur van een heterogeen mengsel van A en B? (Vluchtige stoffen) Waarom is dit?
        \item Wat is een stoomdestillatie? Wanneer is het relevant? Wat is het nut ervan?
        \item Wat is een eutectium of eutectisch mengsel? Geef een toepassing.
        \item Wat is smelt?
        \item Wat is het eutectisch punt?
        \item Leg uit: smeltexperiment in mengsel van 2 helemaal niet mengbare vaste stoffen
        \item Leg uit: stolexperiment in mengsel van 2 helemaal niet mengbare vaste stoffen
        \item Wanneer ontstaan meerdere eutectia? Teken.
    \end{enumerate}

    \section{De elektronenconfiguratie van atomen}
    \subsection*{Inleiding}
    \begin{enumerate}
        \item Waarom kan je debateren dat deze 'statement' niet klopt: Atomen zijn ondeelbare deeltjes in chemische reacties.
        \item Uit wat bestaat materie zelden in standaardomstandigheden? Zijn er uitzonderingen? Hoe kunnen we deze staat krijgen?
        \item Wat zijn vrije atomen?
        \item Wat zijn moleculen?
        \item Uit welke deeltjes bestaan polymeren, metalen en keramische materialen?
    \end{enumerate}
    \subsection*{Ionisatie energie van atomen}
    \begin{enumerate}
        \item Wat is de ionisatie-energie van een atoom? Aan wat is dit gelijk in absolute waarden?
        \item Je hebt eerste/tweede/... ionisatie-energie. Hoe ver kan je gaan?
        \item Waarom is de tweede ionisatie-energie kleiner dan de eerste ionisatie-energie?
        \item Hoe verlopen de ionisatie-energieën in functie van het periodiek systeem?
        \item Bewijs dat elektronenconfiguratie bestaat in lagen (schillen). Doe dit met Natrium.
        \item Wat zijn de besluiten genomen in verband met de elektronenconfiguratie?
    \end{enumerate}
    \subsection*{Interactie elektromagnetische golf-materie}
    \begin{enumerate}
        \item Wat bevestigt de studie van de interactie van elektromagnetische straling? Leg uit.
        \item Wat bedoelt men met 'FINGERPRINT' van een atoom?
        \item Wat is een golf?
        \item Wat is een elektromagnetische golf?
        \item Wat is elektromagnetische straling?
        \item Wat is het gemeenschappelijk kenmerk van de elektromagnetische golven? Maak een tekening.
        \item Wat zijn de belangrijke parameters van elektromagnetische golven?
        \item Geef de 'brede waaier' van elektromagnetische golven. Hoe verloopt de energie/frequentie/golflengte in deze waaier?
        \item Wat is de golflengte van zichtbaar licht?
        \item Waarom is de energie recht evenredig aan de frequentie bij elektromagnetische golven?
        \item Wat bedoelt men met witspectrumlicht?
        \item Hoe creëert men licht van geatomiseerd gas in een ontladingsbuis? Waarom is dit geëmiteerd licht niet continue? Hoe noemt men dit spectrum?
        \item Wat is atomenemissie? Geef een tekening. Geef voorbeelden.
        \item Wat is de emperische gelijkheid van Balmer?
        \item Leg uit: Atomenemissie spectrum is het element van de 'FINGERPRINT'.
        \item Leg uit wat het absorptiespectrum is en het verschil met het emissiespectrum. Wat leert men hier uit?
        \item Wat zijn de uiteindelijke conclusies van de energieverdeling in de materie? Welke ontdekkingen complementeren deze energieverdeling?
        \item Geef de stelling van Planck.
        \item Waarop geeft de stelling van Planck antwoord?
        \item Wat zijn kwanta?
        \item Wat is kwantum?
        \item Wat is de constante van Planck?
        \item Hoe kan men met submicroscopische deeltjes en macroscopische voorwerpen een verband leggen in de klassieke mechanica?
    \end{enumerate}
    \subsection*{Dualiteit deelte-golfkarakter van een elektron in de materie}
    \begin{enumerate}
        \item Wat kun je vertellen over de interferentie franjes van Young?
        \item Wat zijn X-stralen en geef een toepassing?
        \item Wat is een diffractiepatroon?
        \item Waarom is het diffractiepatroon van aluminiumfolie zo speciaal? (voorbeeld)
        \item Wat is het uiteindelijk besluit uit de interferentiefranjes van Young? Is hier een belangrijke voorwaarde aan gehecht? (hint: ja)
        \item Wat is een toepassing van het dualiteitprincipe?
        \item Wat zegt de wet van De Broglie?
        \item Wat zijn de voorwaarden van een De Broglie-golf?
        \item Wat is de relevantie bij submicroscopische deeltjes met de wet van De Broglie?
        \item Wat is de niet-relevantie bij macroscopische deeltjes met de wet van De Broglie?
        \item Leg de planetaire visie en zijn beperkingen uit.
        \item Geef de formule voor de bindingsenergie in het Bohr-model. Wat leer je hier uit? Waarom is hier de absorbtie- en emissiespectra perfect te verklaren?
        \item Wanneer is een energieovergang mogelijk?
        \item Wat is de stralingsfrequentie?
        \item Bereken $\lambda_{4 \rightarrow 2}$. 
        \item Bewijs de emperische relatie van Balmer.
        \item Geef de 4 lijnenreeksen. Wat leer je hieruit?
        \item Wat is de laagste energietoestand?
        \item Bewijs dat de ionisatie-energie de positieve laagste energietoestand is.
        \item Wat zijn de tekortkomingen van het Bohr-model?
    \end{enumerate}
    \subsection*{Atoomorbitalen en elektronenconfiguratie in meerelektrone atomen}
    \begin{enumerate}
        \item Leg uit: onzekerheidsprincipe van Heisenberg.
        \item Wat is de wiskundige formuleren van het onzekerheidsprincipe van Heisenberg. Wat leer je hier uit?
        \item Pas het onzekerheidsprincipe van Heisenberg toe op een elektron ($x=1 A  , \Delta x = 0.3 A$). Wat leer je hier uit?
        \item Pas het onzekerheidsprincipe van Heisenberg toe op een golfbal ($m=45.9g , v=200 \frac{m}{s}, \Delta x = 1mm$). Wat leer je hier uit?
        \item Hoe worden orbitalen wiskundig beschreven?
        \item Geef en leg de vergelijkingen van Schrödinger uit.
        \item Geef alle kwantumgetallen, hun betekenis en hun mogelijke theoretische waarden.
        \item Leg het experiment uit dat tot de ontdekking van het 4de kwantumgetal heeft geleid. Waarom is dit kwantumgetal het buitenbeetje van de kwantumgetallen? Waarom is deze dan ingevoerd?
        \item Geef alle kwantumgetal combinaties tot $n=4$.
        \item Wat zijn de 2 soorten grafische voorstellingen van de orbitalen? Leg ze ook uit.
        \item Bereken de onzekerheid op de impuls $\Delta p$ van een golfbal met massa $m = 4.59e-2$ en
        snelheid $v = 55.56 m/s$ waarvan we de positie tot op 1mm nauwkeurig kennen en trek conclusies uit de gevonde waarde.
        \item Leg het onzekerheidsprincipe van Heisenberg uit + formule.
        \item Geef alle kwantumgetallen, hun betekenis en hun mogelijke theoretische waarden.
        \item Leg het experiment uit dat tot de ontdekking van het 4de kwantumgetal heeft geleid.
        \item Wat zijn knooppunten in orbitalen (schets een grafiek).
        \item Beschrijf alle orbitalen.
        \item Wat is het uitsluitingsprincipe van Pauli?
        \item Wat is de regel van Hund? Zijn er uitzonderingen?
        \item Waarom volgt Chroom (Cr) niet de theoretische verdeling van elektronen?
        \item Wat is de definitie van valentie?
        \item Geef de 2 mogelijke definities voor de straal van een atoom.
        \item Leg het verband tussen de covalente straal, het periodiek systeem en de ionisatie-energie uit.
    \end{enumerate}

    \section{De chemische binding}

    \begin{enumerate}
        \item Waarom gaan atomen bindingen aan?
        \item Bespreek alle mogelijke bindingen.
        \item Wat is een ionaire stof?
        \item Geef de definitie van stabiele ionen.
        \item Wat is elekto affiniteit (EA)?
        \item Wat is ionisatie-energie?
        \item Beschrijf de Born-Habercyclus voor keukenzout.
        \item Teken de Lewisstructuur van \ce{OF2} en \ce{NH4}.
        \item Geef de definitie van een formele lading.
        \item Wat is de formele lading van \ce{NH4+} en \ce{ClO3-}?
        \item Wat is resonantie?
        \item Teken de correcte Lewisstructuur van \ce{CO3^2-}
        \item Wat is een radicaal?
        \item Schrijf de correcte Lewisstructuur van \ce{NO2}
        \item Schrijf de correcte Lewisstructuur van \ce{BeH2}
        \item Schrijf de correcte Lewisstructuur van \ce{BF3}
        \item Wat is hypervalentie?
        \item Schrijf de correcte Lewisstructuur van \ce{SF6}
        \item Schrijf de correcte Lewisstructuur van \ce{SO3}
    \end{enumerate}

    \section{Ruimtelijke- en elektronenstructuur van moleculen}

    \begin{enumerate}
        \item Wat is de VSEPR-theorie?
        \item Wat is een elektron domein?
        \item Geef 4 verschillende voorstellingen van ammoniak nodig om de structuur te begrijpen.
        \item Waarom heb je verschillende hoeken bij \ce{CH4}, \ce{NH3}, \ce{H2O}?
        \item Wanneer is een binding covalent, polair covalent of ionair?
        \item Wat is een dipoolmoment en bindingsmoment in de chemie?
        \item Zet 1 D(ebye) om naar SI-eenheden.
        \item Wat is het dipoolmoment van \ce{CO2}?
        \item Geef de definitie van intermoleculaire krachgten.
        \item Beschrijf de dipool-dipool interactie en zijn effect op het kookpunt.
        \item Beschrijf de ge\"induceerde krachten of gedisperceerde London krachten.
        \item Geef 3 parameters die de sterkte van de intermoleculaire krachten bij apolaire stoffen bepalen.
        \item Beschrijf waterstofbruggen.
        \item Geef de definitie van oppervlaktespanning.
        \item Geef de definitie van capillariteit.
    \end{enumerate}

    \section{Moleculaire orbitalen en structuren}

    \begin{enumerate}
        \item Wat zijn de limieten van de VSEPR-theorie?
        \item Wat is overlap in de kwantum chemie?
        \item Wat is een bindend elektronenpaar?
        \item Waarom onstaat er steeds bij atoomoverlap een moleculaire orbitaal?
        \item Beschrijf de gevormde orbitalen bij de vorming van \ce{H2} uit 2 \ce{H}-atomen.
        \item Wat is hybridisatie van atoomorbitalen en geef de wiskundige betekenis.
        \item Beschrijf orbitalair het ganse structuurvormingsproces van \ce{BeH2}.
        \item Beschrijf orbitalair het ganse structuurvormingsproces van \ce{BF3}.
        \item Beschrijf orbitalair het ganse structuurvormingsproces van \ce{CH4}.
        \item Beschrijf orbitalair het ganse structuurvormingsproces van \ce{PF5}. 
        \item Beschrijf orbitalair het ganse structuurvormingsproces van \ce{SF6}.
        \item Wat is de voorwaarde omdat men hybridisatie kan toepassen?
        \item Wat is een $\sigma$-binding?
        \item Wat is een $\sigma$-bindings geraamte?
        \item Wanneer kun je afwijkingen hebben bij de hybrides en hoe zou je dit kunnen verklaren?
        \item Wat zijn $\pi$-orbitalen?
        \item Beschrijf orbitalair het ganse structuurvormingsproces van \ce{CH2} (2 $C$-atomen en 4 $H$-atomen).
        \item Kunnen $\pi$-bindingen draaien en zo ja, in welke omstandigheden?
        \item Geef alle mogelijke combinaties orbitalen die samen een  $\pi$-binding kunnen vormen.
        \item Beschrijf  $\pi$-elektronendelokalisatie.
        \item Als n atoomorbitalen overlappen, hoeveel molecuuloribtalen worden er dan gevormd?
        \item Wat zijn K\'ekul\'estructuren?
        \item Beschrijf met de molecuul orbitalen theorie de structuur van een benzeen-ring.
        \item Geef een nodige voorwaarde opdat een molecule aromatisch kan genoemd worden.
        \item De stof \ce{NO3-} heeft een gedelokaliseerd $\pi$-elektronenpaar. Welke eigenschappen zou je uit hieruit kunnen afleiden? 
        \item Geef een definitie van een elektrofiel.
        \item Geef een definitie van een nucleofiel.
        \item Wat is een elektrodefici\"ent atoom?
        \item Bespreek de reactie tussen een elektrofiel en een nucleofiel op thermodynamisch niveau.
        \item Wat is een \textbf{heterolytische reactie}?
        \item Wat is een \textbf{homolytische reactie}?
        \item Wat is een radicaal?
        \item In welk soort milieu worden heterolytische en homolytische reacties bevorderd?
        \item Wat wordt er bedoelt met inductieve effecten?
        \item Tot hoe ver hebben de meeste inductieve effecten hun effect?
        \item Geef een gevolg van mesomere effecten.
    \end{enumerate}

    \section{Organische chemie: structen en naamgeving}
    \begin{enumerate}
        \item Wanneer is een molecule organisch?
        \item Wat wordt er bedoeld met een \textbf{verzadigd C-atoom}?
        \item Geef de molecuulformule van een volledig verzadigde koolwaterstof.
        \item Wat zijn alkanen?
        \item Wat wordt er bedoeld met \textbf{conformationele flexibiliteit}?
        \item Geef alle \textbf{multiplicerende voorvoegsels} voor substituenten.
        \item Geef alle \textbf{multiplicerende voorvoegsels} voor samengestelde substituenten.
        \item Teken de structuur van 5,6-bis(1,1-dimethylpropyl)decaan.
        \item Teken de structuur van 2-hexeen.
        \item Teken de structuur van isopreen.
        \item Teken de structuur van cyclooctatetra\"een.
    \end{enumerate}

\end{document}

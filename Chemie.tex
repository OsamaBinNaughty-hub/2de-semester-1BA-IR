\documentclass[a4paper,12pt]{article}

\usepackage[dutch]{babel}
\usepackage{amsmath}
\usepackage[version=4]{mhchem}

\begin{document}
    \title{Chemie I deel 2}
    \author{Dries Van den Brande \and Andreas Declerck}

    \maketitle
    \section{Hfst 1.7: Aggregatietoestanden en toestandsovergangen. Fasenleer}

    \subsection{Algemene formules en begrippen}
    \begin{enumerate}
        \item Geef de formule voor de faseregel van Gibbs en leg alle parameters uit.
        \item Geef de formule voor kook-/ condensatielijn van Clapeyron-Claussius.
        \item Hoe kan men experimenteel de kook enthalpie vinden?
        \item Hoe wordt vluchtigheid gedefini\"eerd?
        \item Leg uit: gecondenceerde toestand.
        \item Geef de verschillen tussen damp en koken op thermodynamisch niveau (2 zaken).
        \item Geef de definitie van een ideaal mengsel.
        \item Wat zijn colligatieve eigenschappen?
        \item Wat is colligatieve molaliteit (niet fout geschreven)?
        \item Wat is een stoomdiagram?
        \item Wat is smelt?
    \end{enumerate}

    \subsubsection{Wet van Raoult}
    \begin{enumerate}
        \item Geef en leidt af: Wet van Raoult.
        \item Wat is de voorwaarde omdat men de wet van Raoult mag gebruiken?
        \item Geef een wiskunige formule voor het berekenen van de molaliteit van B.
        \item Leid uit de wet van Raoult de formule van de dampdruk af.
        \item Leg uit aan de hand van de wet van Raoult waarom men zout strooit in de winter?
        \item Noem 4 verschillende colligatieve eigenschappen.
        \item Stel vanuit de wet van Raoult de functie $ p = f(x_A^d) $ op.
    \end{enumerate}

    \subsection{Uitzonderingen en speciale gevallen}
    \begin{enumerate}
        \item Leg uit: Wet van Trouton.
        \item Leg uit: onderkoelde vloeistof.
        \item Geef een toepassing van een superkritisch flu\"idum.
        \item Wat is het grote verschil tussen allotropen en gewone enkelvoudige stoffen m.b.t. het fasediagram?
        \item Wat is er zo speciaal aan het fasediagram van water en waarom is dit verschillend?
    \end{enumerate}

    \subsection{Fasediagrammen}
    \begin{enumerate}
        \item Som de belangrijke punten in het fasediagram op en leg uit wat er hier gebeurt.
        \item Teken een fasediagram \{T,x\}, \{T,t\}, \{x,t\} en leg alle stappen uit en gebruik de faseregel van Gibbs.
        \item Wat is een azeotroop?
        \item Teken het fasediagram van een azeotroop uit en gebruik de faseregel van Gibbs.
        \item Waarom lijkt in het \{T,x\}-deel van een fasediagram de menglijn op een parabool bij heterogene mengsels?
    \end{enumerate}

    \subsection{Eutecticum (\emph{Zeer waarschijnlijk op examen})}
    \begin{enumerate}
        \item Wat is een eutecticum?
        \item Teken een fasediagram van heterogene vaste stoffen die smelten en leg uit waar elk gebied voor staat.
        \item Wanneer ontstaan meerdere eutectica?
    \end{enumerate}

    \section{De elektronenconfiguratie van atomen}

    \subsection{Algemene begrippen en formules}

    \begin{enumerate}
        \item Wat is een elektronenconfiguratie?
        \item Geef de definitie van \emph{ionisatie-energie}.
        \item Wat zijn vrije atomen?
        \item Hoe kan men vrije atomen in gasfase krijgen?
        \item Wat zijn moleculen?
        \item Geef de stelling van Planck.
        \item Wat is de constante van Planck?
    \end{enumerate}

    \subsection{Bohr-model}

    \begin{enumerate}
        \item Wat wordt er bedoeld met het wit lichtspectrum?
        \item Leg het experiment rond de geëmiteerde kleurenspectra van geatomiseerd gas uit.
        \item Leg uit wat men bedoelt met de \emph{FINGERPRINT van een atoom} en geef enkele toepassingen.
        \item Leg uit wat het absorptiespectrum is en het verschil met het emissiespectrum.
        \item Wat kan men besluiten uit de experimenten met het absorptiespectrum en het emissiespectrum?
        \item Wat kun je vertellen over de interferentie franjes van Young?
        \item Wat zijn X-stralen en geef een toepassing?
        \item Wat is een difractiepatroon?
        \item Kan een deeltje ook tegelijk een golf zijn?
        \item Geef de wet van de Broglie.
        \item Waarom kunnen we X-stralen gebruiken voor elektronendifractie?
        \item Bereken de golflengte van een golfbal met massa $m = 4.59e-2$ en
        snelheid $v = 55.56 m/s$ en trek conclusies uit de gevonde waarde.
        \item Leg de planetaire visie en zijn beperkingen uit.
        \item Geef de formule voor de bindingsenergie in het Bohr-model.
        \item Bewijs de relatie van Balmer mbv de wet van Planck en de bindingsenergie in het model van Bohr.
        \item Bereken de energie van het dichtste elektron bij een H-kern in J/mol en trek conclusies rond het gevonden resultaat.
    \end{enumerate}

    \subsection{De verbetering van het Bohr-model en de structuur van atomen met meerdere elektronen}

    \begin{enumerate}
        \item Bereken de onzekerheid op de impuls $\Delta p$ van een golfbal met massa $m = 4.59e-2$ en
        snelheid $v = 55.56 m/s$ waarvan we de positie tot op 1mm nauwkeurig kennen en trek conclusies uit de gevonde waarde.
        \item Leg het onzekerheidsprincipe van Heisenberg uit + formule.
        \item Geef alle kwantumgetallen, hun betekenis en hun mogelijke theoretische waarden.
        \item Leg het experiment uit dat tot de ontdekking van het 4de kwantumgetal heeft geleid.
        \item Wat zijn knooppunten in orbitalen (schets een grafiek).
        \item Beschrijf alle orbitalen.
        \item Wat is het uitsluitingsprincipe van Pauli?
        \item Wat is de regel van Hund? Zijn er uitzonderingen?
        \item Waarom volgt Chroom (Cr) niet de theoretische verdeling van elektronen?
        \item Wat is de definitie van valentie?
        \item Geef de 2 mogelijke definities voor de straal van een atoom.
        \item Leg het verband tussen de covalente straal, het periodiek systeem en de ionisatie-energie uit.
    \end{enumerate}

    \section{De chemische binding}

    \begin{enumerate}
        \item Waarom gaan atomen bindingen aan?
        \item Bespreek alle mogelijke bindingen.
        \item Wat is een ionaire stof?
        \item Geef de definitie van stabiele ionen.
        \item Wat is elekto affiniteit (EA)?
        \item Wat is ionisatie-energie?
        \item Beschrijf de Born-Habercyclus voor keukenzout.
        \item Teken de Lewisstructuur van \ce{OF2} en \ce{NH4}.
        \item Geef de definitie van een formele lading.
        \item Wat is de formele lading van \ce{NH4+} en \ce{ClO3-}?
        \item Wat is resonantie?
        \item Teken de correcte Lewisstructuur van \ce{CO3^2-}
        \item Wat is een radicaal?
        \item Schrijf de correcte Lewisstructuur van \ce{NO2}
        \item Schrijf de correcte Lewisstructuur van \ce{BeH2}
        \item Schrijf de correcte Lewisstructuur van \ce{BF3}
        \item Wat is hypervalentie?
        \item Schrijf de correcte Lewisstructuur van \ce{SF6}
        \item Schrijf de correcte Lewisstructuur van \ce{SO3}
    \end{enumerate}

\end{document}

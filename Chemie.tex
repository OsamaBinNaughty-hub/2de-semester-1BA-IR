\documentclass[12pt]{article}

\begin{document}
    \title{Chemie I deel 2}
    \author{Dries Van den Brande \and Andreas Declerck}

    \maketitle
    \section*{Hoofdstuk 1.7: Aggregatietoestanden en -toestandsovergangen. Fasenleer}
    \begin{enumerate}
        \item Geef de formule voor de faseregel van Gibbs en leg alle parameters uit.
        \item Geef de formule voor kook-/ condensatielijn van Clapeyron-Claussius.
        \item Hoe kan men experimenteel de kook enthalpie vinden?
        \item Leg uit: Wet van Trouton.
        \item Leg uit: onderkoelde vloeistof.
        \item Geef een toepassing van een superkritisch flu\"idum.
        \item Wat is het grote verschil tussen allotropen en gewone enkelvoudige stoffen m.b.t. het fasediagram?
        \item Wat is er zo speciaal aan het fasediagram van water en waarom is dit verschillend?
        \item Hoe wordt vluchtigheid gedefini\"eerd?
        \item Leg uit: gecondenceerde toestand.
        \item Geef de verschillen tussen damp en koken op thermodynamisch niveau (2 zaken).
        \item Som de belangrijke punten in het fasediagram op en leg uit wat er hier gebeurt.
    \end{enumerate}
\end{document}

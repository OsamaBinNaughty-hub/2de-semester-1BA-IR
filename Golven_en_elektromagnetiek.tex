\documentclass[12pt]{article}

\usepackage[dutch]{babel}
\usepackage{amsmath}

\begin{document}
    
    \title{Golven en elektromagnetisme}
    \author{Dries Van den Brande \and Andreas Declerck}

    \maketitle
    \begin{enumerate}
        \item Geef de differentiaal vergelijking nodig voor de afleiding van de algemene golfbetrekking.
        \item Leidt de algemene golfbetrekking af.
        \item Wat zijn veronderstelling die we maken voor het opstellen van de algemene golvergelijking?
        \item Waarom defini\"eren we $\alpha = x - vt$ en $\beta = x + vt$ op deze manier in de afleding voor de algemene golfbetrekking?
        \item Geef de klassificatie en de subklassificatie van golven.
        \item Som de verschillende golven op aanwezig bij een aardbeving en zeg onder welke noemer we ze kunnen klassificeren;
        \item Verklaar waarom bij aardbevingen de verschillende golven een verschillende snelheid hebben terwijl we weten dat de snelheid wordt bepaald door het milieu en alle deze golven de aarde als milieu hebben.
        \item Geef en leidt de betrekking ruimte-tijd in de golfvergelijkingen af.
        \item Wat is een monochromatische golf? 
    \end{enumerate}

    \section{ALGEMENE INLEIDING}
    \subsection{Het contract "Golven en elektromagnetische"}
    Niet te kennen
    \subsection{Wat is fysica?}
    Niet te kennen
    \subsection{Het standaardmodel als illustratie van reductionisme}
    Niet te kennen



    \section{WERKTUIGEN}
    \subsection{Wiskundige afspraken}
    Niet te kennen
    \subsection{Eenheden}
    Niet te kennen
    \subsection{Grootteordes en machten van tien}
    Niet te kennen
    \subsection{Dimensieanalyse}
    Niet te kennen
    \subsection{De harmonische oscillator}
    Niet te kennen
    \subsection{De Galileo-transformaties}
    Niet te kennen



    \section{GOLVEN EN TRILLINGEN}
    \subsection{Inleidende begrippen en fundamentele concepten}
    \begin{enumerate}
        \item 
    \end{enumerate}
    \subsection{Harmonische, lineaire, vlakke, scalaire golven}
    \begin{enumerate}
        \item 
    \end{enumerate}
    \subsection{De golfbetrekking of golfvergelijking}
    \begin{enumerate}
        \item 
    \end{enumerate}
    \subsection{Staande golven}
    \begin{enumerate}
        \item 
    \end{enumerate}
    \subsection{Zwevingen in de tijd en ruimte}
    \begin{enumerate}
        \item 
    \end{enumerate}
    \subsection{De gespannen snaar}
    \begin{enumerate}
        \item 
    \end{enumerate}
    \subsection{De sferische golf}
    \begin{enumerate}
        \item 
    \end{enumerate}
    \subsection{Geluid}
    \begin{enumerate}
        \item 
    \end{enumerate}
    \subsection{Het Doppler effect}
    \begin{enumerate}
        \item 
    \end{enumerate}



    \section{ELEKTROMAGNETISME}
    \subsection{Inleiding}
    \begin{enumerate}
        \item Wie is Maxwell? Wat doen de vergelijkingen van Maxwell? Voor welke grote geschiedenis evenement had dit de basis gelegt?
        \item Geef de eenheden en notatie van de volgende concepten: (Elektrische lading, Ladingsdichtheid, Elektrisch Veld, Magnetisch inductieveld, Elektische stroom, Stroomdichtheid).
        \item Geef de volgende natuurconstanten: (Permittiviteit van vacuüm, Permeabiliteit van vacuüm, snelheid van het licht).
        \item Geef de Maxwell vergelijkingen in globale/integrale en lokale/differentiale vorm.
    \end{enumerate}
    \subsection{Werktuigen / Toolbox}
    \begin{enumerate}
        \item Wat is het doel van dit hoofdstuk? Welke onderwerpen gaan we over gaan? (belangrijk voor structuur te houden)
        \item \textbf{Geef de intrinsieke definitie van de divergentie van een vectorveld, bespreek de betekenis ervan en geef minimum twee voorbeelden van een natuurwet waarin de divergentie een belangrijke rol speelt. Idem voor de rotatie van een vectorveld.} (HOOFDVRAAG)
        \item \textbf{Maak de overgang tussen de globale en lokale vorm van de divergentie van een vectorveld en de rotatie van een vectorveld (en/of omgekeerd)} (HOOFDVRAAG)
        \item \textbf{Geef de intrinsieke definitie van de gradiënt van een scalair veld, bespreek de betekenis ervan en geef minimum twee voorbeelden van een natuurwet waarin de gradiënt een belangrijke rol speelt. Geef en bewijs de eigenschappen van een gradiënt.} (HOOFDVRAAG)
        \item Wat bedoeld men met een veld? Waaraan is een veld afhankelijk? Geef voorbeelden.
        \item Wanneer kan je zeggen dat iets in het ander haar veld ligt? Geef een voorbeeld.
        \item Defineer: Scalair veld. Beschijf scalaire velden in globale en lokale vorm. Geef voorbeelden van scaliare velden.
        \item bewijs dat $\delta v = \vec{grad}v(\vec{r}) \cdot \vec{dr}$. Bekijk hiervoor $v(\vec{r} + \vec{\delta r})$
        \item Geef de belangrijke voorbeelden voor de toepassing van de gradiënt. (5)
        \item Defineer: Vectorveld. Beschrijf vectorvelden in globale en lokale vorm. Geef een voorbeeld van een vectorveld.
        \item Defineer: Flux van een vectorveld $\Phi_E$ .
        \item Defineer: Flux voor een open oppervlakte.
        \item Defineer: Flux voor een gesloten oppervlakte.
        \item Defineer: Flux van een uniform veld $\vec{E}$ doorheen een vlak met een hoek.
        \item Defineer: Flux van een algemeen veld $\vec{E}$ doorheen een willekeurig oppervlakte.
        \item Bereken het waterdebiet doorheen een oppervlakte. Check dit ook met een dimensieanalyse.
        \item Bereken de elektrische stroom door middel van een fluxintegraal. Check dit ook met een dimensieanalyse.
        \item Defineer: Flux van een uniform veld $\vec{E}$ doorheen een gesloten oppervlakte.
        \item Zijn er andere notaties voor de divergentie? ( Tip: ja )
        \item Defineer: Algemene vorm van een Bilan wet. Geef voorbeelden.
        \item Geef de differentiaaloperatorgymnastiek? Wat zijn de gevolgen hiervan?
    \end{enumerate}
    \subsection{Electrostatica}
    \begin{enumerate}
        \item \textbf{Geef de Wet van Coulomb voor de kracht die twee puntladingen op elkaar uitoefenen.} (HOOFDVRAAG)
        \item \textbf{Toon kwantitatief aan dat elektrische wisselwerkingen, als ze bestaan, altijd veeeel groter zijn dan de effecten van gravitationele wisselwerkingen.} (HOOFDVRAAG)
        \item Wat is elektrostatica? Wat is het gevolg van deze definitie? Voor wat vormt dit een basis? 
        \item Wat is een lading? (tricky question)
        \item Wat zijn de soorten ladingen?
        \item Wat bedoeld men met: Behoud van lading. Druk dit ook wiskundig uit.
        \item Hoe heeft men de lading gequantiseerd? Wat bevestigde dit?
        \item Hoe gaan we voorwerpen een lading geven? Leg uit. Geef voorbeelden.
        \item Aan wat is de elektrische classificatie van materialen afhankelijk?
        \item Wat zijn de soorten materialen? Leg uit. geef voorbeelden.
        \item Welke waarnemingen zorgded voor de Wet van Coulomb.
        \item Geef de Wet van Coulomb voor de kracht die meerdere puntladingen op elkaar uitoefenen.
        \item Geef de Wet van Coulomb voor de kracht van een continue lading. Geef hierbij ook de formules voor lineare en een oppervlakte ladingsverdeling.
        \item Leg uit: Basic principal of why photocopy machines use electrostatics.
        \item Een menselijk lichaam heeft 65 kg water. Bereken de positieve lading van deze persoon.
    \end{enumerate}
    \subsection{Het elektrisch veld in de elektrostatica}
    \begin{enumerate}
        \item \textbf{Bereken het elektrisch veld van een dipool (op de as van de dipool en op het middelloodvlak). Geef ook de algemene formule.} (HOOFDVRAAG)
        \item \textbf{Definieer een elektrische dipool.} (HOOFDVRAAG)
        \item \textbf{Bereken het krachtmoment (synoniem “koppel”) en de potentiële energie van een elek­trische dipool in een extern (uniform) elektrisch veld. Leg een toepassing hierop ook uit.} (HOOFDVRAAG)
        \item \textbf{Schets ook kwalitatief wat er gebeurt met een dipool in een niet-uniform elektrisch veld.} (HOOFDVRAAG)
        \item \textbf{Verklaar kwalitatief hoe via de “Van der Waals” interactie twee ongeladen moleculen elkaar toch kunnen aantrekken.} (HOOFDVRAAG)
        \item Defineer het elektrisch veld. Wat is de oorzaak van het elektrisch veld? Geef de algemene definitie van het elektrisch veld. Hoe kan ik dit interpreteren? Waarom is dit fout? : $\lim{q_0 \to 0} = \frac{\vec{F}(\vec{r})}{q_0}$
        \item Geef de 4 eigenschappen van het elektrisch veld. Leg uit waarom / bewijs eventueel / gevolgen (je weet wel)
        \item Geef de verschillende formules van het elektrisch veld met continue ladingsverdelingen.
        \item Teken de elektrische veldlijnen van een positieve en negatieve lading.
        \item Geef de eigenschappen van elektrische veldlijnen.
        \item Teken de elektische veldlijnen van 2 stilstaande ladingen met: +Q,-Q / +Q,+Q / -Q,+2Q
        \item Wat is het verband tussen de elektrische velden en divergentie?                                                                             
        \item Defineer: dipoolmoment + geef voorbeeld
        \item Wat zijn de voorwaarden van een gebonden lading? Wat kan je hier uit concluderen voor het elektrisch veld?
        \item Systemen in de natuur streven naar een minimum potentiële energie. Wanneer is dit?
        \item Wat is de netto kracht van een elektrisch veld tegen over een punt lading? (pos en neg) Wat kan men hieruit concluderen?
    \end{enumerate}
    \subsection{De wet van Gauss}
    \begin{enumerate}
        \item \textbf{Geef en bespreek de Wet van Gauss in integrale en lokale vorm, en hoe je overgaat van de ene naar de andere.} (HOOFDVRAAG)
        \item \textbf{Gebruik de Wet van Gauss om het elektrisch veld te berekenen binnen én buiten een diëlektrische sfeer met homogeen verdeelde ladingsdichtheid} (HOOFDVRAAG)
        \item \textbf{Gebruik deze om het elektrisch veld te berekenen binnen én buiten een diëlektrische sfeer met homogeen verdeelde ladingsdichtheid.} (HOOFDVRAAG)
        \item \textbf{Bewijs dat een sferisch symmetrische ladings­verdeling, in de ruimte buiten de ladingsverdeling, een elektrisch veld opwekt zoals een puntlading.} (HOOFDVRAAG)
        \item Wat brengt de Wet van Gauss in verband? Over welke concepten wordt er gesproken bij de Wet van Gauss?
        \item Wat zegt de Wet van Gauss? Leid deze af met een eenvoudig model.
        \item Defineer: Gauss oppervlakte
        \item Wat is de toepassing van de Wet van Gauss? Wat is de strategie voor deze toepassing?
    \end{enumerate}
    \subsection*{Wiskundige aanvulling - Drie-voudige integraal in sferische coördinaten}
    \begin{itemize}
        \item Geef de algemene transformatie van coördinaten. Geef de transformatieformules.
        \item Welke stelling gaan we gebruiken voor de transformatie van bolcoördinaten?
        \item Teken de transformatie van bolcoördinaten. Geef de transformatieformules.
        \item Bereken de Jacobiaanse determinant voor bolcoördinaten.
        \item Integreer op een boloppervlakte.
    \end{itemize}
    \subsection{De elektrische potentiaal - elektrische spanning}
    \begin{enumerate}
        \item \textbf{Wat is de elektrostatische potentiaal (geef de definitie en de fysische betekenis)?} (HOOFDVRAAG)
        \item \textbf{Hoe kan je uit de potentiaal terug het elektrisch veld berekenen?} (HOOFDVRAAG)
        \item \textbf{Vertrekken­de van de Wet van Coulomb, bereken de elektrostatische potentiaal opgewekt door een punt­lading Q} (HOOFDVRAAG)
        \item \textbf{Leg uit hoe je de elektrostatische potentiaal opgewekt door een elektrische dipool kan berekenen. Vergelijk.  } (HOOFDVRAAG)
        \item Defineer: De potentiële energie uit de klassieke mechanica.
        \item Defineer: De totale energie:
        \item Defineer: Veralgemeende definitie van de potentiële energie
        \item Defineer: elektrische spanning $\Delta V$
        \item Geef de eigenschappen van de elektrische potentiaal en leg uit.
        \item Defineer: elektronvolt $eV$
        \item Bereken het elektrisch veld van een H-kern / H-atoom en een chemische reactie.
        \item Bereken de elektrische potentiaal van een uniform geladen bolcoördinaten. Leg uit.
        \item Bereken de elektrische potentiaal van een uniform geladen oneindig grote plaat. Leg uit.
    \end{enumerate}
    \subsection{De condensator}
    \begin{enumerate}
        \item 
    \end{enumerate}
    \subsection{Geleiders en isolatoren in elektrostatische omstandigheden}
    \begin{enumerate}
        \item 
    \end{enumerate}
    \subsection{Inleiding tot het magnetisme}
    \begin{enumerate}
        \item 
    \end{enumerate}
    \subsection{Beweging van een lading in een constant magnetisch inductieveld}
    \begin{enumerate}
        \item 
    \end{enumerate}
    \subsection{Magnetostatica - de fysica van constante elektrische stromen}
    \begin{enumerate}
        \item 
    \end{enumerate}
    \subsection{De wet van Biot-Savart - Constante stromen en magnetische velden}
    \begin{enumerate}
        \item 
    \end{enumerate}
    \subsection{De wet van Ampère}
    \begin{enumerate}
        \item 
    \end{enumerate}
    \subsection{Samenvatting van de elektrostatica - magnetostatica}
    \begin{enumerate}
        \item 
    \end{enumerate}
    \subsection{De wet van Faraday-Lenz - Elektromagnetische inductie}
    \begin{enumerate}
        \item 
    \end{enumerate}
    \subsection{Inductantie en elektrische kringen}
    \begin{enumerate}
        \item 
    \end{enumerate}
    \subsection{INTERMEZZO - Magnetische materialen}
    \begin{enumerate}
        \item 
    \end{enumerate}
    \subsection{De Maxwell-vergelijkingen - Samenvatting van de klassieke fysica}
    \begin{enumerate}
        \item 
    \end{enumerate}
    \subsection{De theorie - Eigenschappen van elektromagnetische straling in vacuüm}
    \begin{enumerate}
        \item 
    \end{enumerate}
    \subsection{Energie in elektromagnetische straling - De stelling van Poynting}
    \begin{enumerate}
        \item 
    \end{enumerate}
    \subsection{Het experiment - Golfkarakter van elektromagnetische straling}
    \begin{enumerate}
        \item 
    \end{enumerate}



    \section{HET DEELTJESKARAKTER VAN ELEKTROMAGNETISCHE STRALING}
    \subsection{Zwarte straling}
    \begin{enumerate}
        \item 
    \end{enumerate}
    \subsection{Het foto-elektrisch effect}
    \begin{enumerate}
        \item 
    \end{enumerate}
    \subsection{Het Compton effect}
    \begin{enumerate}
        \item 
    \end{enumerate}
    \subsection{Gravitational redshift}
    \begin{enumerate}
        \item 
    \end{enumerate}



    \section{INLEIDING TOT DE SPECIALE RELATIVITEIT}
    \subsection{Probleemstelling - De Lorentz-transformaties}
    \begin{enumerate}
        \item 
    \end{enumerate}
    \subsection{Het experiment van Michelson en Morley}
    \begin{enumerate}
        \item 
    \end{enumerate}
    \subsection{De speciale relativiteitstheorie van Einstein}
    \begin{enumerate}
        \item 
    \end{enumerate}
    \subsection{Gevolgen van de speciale relativiteitstheorie van Einstein}
    \begin{enumerate}
        \item 
    \end{enumerate}
    \subsection{$E=m \cdot c^2$}
    \begin{enumerate}
        \item 
    \end{enumerate}



    \section{LAAG ENERGETISCHE KERNFYSICA EN BEGINSELEN VAN DE STRALINGSBESCHERMING}
    \subsection{Basisbeginselen}
    \begin{enumerate}
        \item 
    \end{enumerate}
    \subsection{Halfwaardetijd en radioactiviteit}
    \begin{enumerate}
        \item 
    \end{enumerate}


\end{document}
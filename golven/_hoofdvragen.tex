
\documentclass[12pt]{article}

\usepackage[dutch]{babel}
\usepackage{amsmath}

\title{Mechanica 2}
\author{Dries Van den Brande \and Andreas Declerck}

\begin{document}
    \maketitle

    \begin{enumerate}
        \item Definieer een “golf” en klasseer golven in die twee grote families met welke we in de cursus hebben gewerkt – bespreek de verschillen tussen beide. Hoe definieer je een vlakke monochromatische golf? Schrijf de algemene uitdrukking van zulke golf op en toon aan dat deze voldoet aan de golf­betrekking. Zoek de algemene oplossing van deze golfbetrekking en bespreek.
        \item Wat zijn vlakke golven en wat zijn sferische golven? Vind de algemene uitdrukking van vlakke golven en van sferische golven (de uitdrukking van de Laplaciaan in sferische coördinaten hoef je niet uit het hoofd te kennen). Bespreek.
        \item Bestudeer staande golven en resonante golven.  Wat zijn harmonieken? Wat zijn zwevingen? Wat is de zwevingsfrequentie of “beat-frequency”?
        \item Bepaal de voortplantingssnelheid van een transversale perturbatie van een gespannen snaar. Doe dit op twee manieren. Bepaal eveneens de energiedichtheid gedragen door een golf op deze snaar. Bespreek.
        \item Bespreek het Doppler effect. Geef toepassingen in de hedendaagse technologie en in de kosmologie: wat is het verschil tussen het “gewone” Doppler effect en het “relativistisch” Doppler effect. Wat is een schokgolf en wat stelt het Mach getal voor?
        \item Geef de intrinsieke definitie van de divergentie van een vectorveld, bespreek de betekenis ervan en geef minimum twee voorbeelden van een natuurwet waarin de divergentie een belangrijke speelt. Idem voor rotatie van een vectorveld. Maak de overgang van de globale naar de lokale vorm van die wetten (en/of omgekeerd). Geef de intrinsieke definitie van de gradiënt van een scalair veld, bespreek de betekenis en geef min. twee voorbeelden van natuurwetten waarin de gradiënt een belangrijke rol speelt. Geef de eigenschappen van een gradiënt.
        \item Geef de Wet van Coulomb voor de kracht die twee puntladingen op elkaar uitoefenen. Toon kwantitatief aan dat elektrische wisselwerkingen, als ze bestaan, altijd véél groter zijn dan de effecten van gravitationele wisselwerkingen. Bereken het elektrisch veld van een dipool (op de as van de dipool en op het middelloodvlak). Geef en bespreek de Wet van Gauss in integrale en lokale vorm, en hoe je overgaat van de ene naar de andere. 
        \item Bespreek de Wet van Gauss in integrale en lokale vorm, en hoe je overgaat van de ene naar de andere. Gebruik deze om het elektrisch veld te berekenen binnen én buiten een diëlektrische sfeer met homogeen verdeelde ladingsdichtheid. Bewijs dat een sferisch symmetrische ladings­verdeling, in de ruimte buiten de ladingsverdeling, een elektrisch veld opwekt zoals een puntlading. Bespreek ook het gedrag van de ladingsverdeling van de vrije ladingen in een geleider in elektrostatisch evenwicht.  Welke zijn de gevolgen voor het hiermee gepaarde elektrisch veld?
        \item Bespreek de Wet van Gauss in integrale en lokale vorm, en hoe je overgaat van de ene naar de andere. Gebruik deze om het elektrisch veld te berekenen opgewekt door een oneindig lange rechte draad met homogeen verdeelde ladingsdichtheid. Wat is een condensator? Bereken het elektrisch veld, het potentiaalverschil en de capaciteit van een lange, coaxiale kabel.
        \item Wat is de elektrostatische potentiaal (geef de definitie en de fysische betekenis)? Hoe kan je uit de potentiaal terug het elektrisch veld berekenen? Vertrekken­de van de Wet van Coulomb, bereken de elektrostatische potentiaal opgewekt door een punt­lading Q. Leg uit hoe je de elektrostatische potentiaal opgewekt door een elektrische dipool kan berekenen. Vergelijk. 
        \item Definieer een elektrische dipool. Bereken het krachtmoment (synoniem “koppel”) en de potentiële energie van een elek­trische dipool in een extern (uniform) elektrisch veld. Schets ook kwalitatief wat er gebeurt met een dipool in een niet-uniform elektrisch veld. Verklaar kwalitatief hoe via de “Van der Waals” interactie twee ongeladen moleculen elkaar toch kunnen aantrekken.
        \item Definieer een elektrische dipool. Bereken het krachtmoment (synoniem “koppel”) en de potentiële energie van een elektrische dipool in een uniform extern elektrisch veld. Idem voor een magnetische dipool (of “magnetisch moment”) in een uniform extern magnetisch veld. Geef toepassingen.
        \item Wat is een condensator? Bereken het elektrisch veld, het potentiaalverschil en de capaciteit van een vlakke condensator. Bereken ook de energie opgeslagen in een vlakke condensator. Wat gebeurt er als er een diëlektricum tussen de platen van een condensator is? Is er een verschil tussen de Wet van Gauss in vacuüm en in diëlektrica? Verklaar en bespreek.
        \item Bespreek de beweging van een lading in een constant magnetisch veld. Onderscheid verschillende gevallen (snelheid loodrecht op het magnetisch inductieveld, of niet). Bespreek de beweging van een lading in een (constant) magnetisch en elektrisch veld die loodrecht op elkaar staan. Geef van alle gevallen voorbeelden en/of toepassingen.
        \item Wat is elektrische stroom? Wat zegt de wet van Ohm? Geef en bespreek het Drude-model voor de geleidbaarheid (of “conductiviteit”). Geef het verband tussen de weerstand van een lange rechte draad en de “resistiviteit” (definieer “resistiviteit”) van het materiaal.
        \item Hoe zijn we tot de wet van Biot-Savart gekomen? Pas deze wet toe om het magnetisch inductieveld opgewekt door een oneindig lange rechte stroomdoorlopen draad te berekenen. Hoe zijn we verder tot de kringwet van Ampère gekomen (lokale en integrale vorm, en overgang tussen beide)? Pas hem toe om het magnetisch inductieveld opgewekt door een solenoïde te berekenen.
        \item Leid vanuit de Lorentz kracht de kracht af op een stroomdoorlopen draad. Bereken, vanuit de uitdrukking voor het magnetisch inductieveld van een lange rechte draad, de kracht tussen twee stroomdoorlopen draden en de definitie van de eenheid “Ampère”. Bereken de krachten op een stroomdoorlopen kring in een homogeen magnetisch inductieveld. Definieer het magnetisch moment, en bereken het krachtmoment (synoniem “koppel”) en de potentiële energie van zulk een magnetisch moment in een uniform magnetisch inductieveld. Bespreek de werking van een gelijk- en wissel­stroommotor.
        \item Geef en bespreek de inductiewet van Faraday-Lenz. Beschrijf minstens drie experimenten die door deze wet verklaard worden. Bespreek toepassingen. Ga over naar de lokale formulering ervan. Bespreek de werking van een wissel­stroom­generator.
        \item Definieer wederzijdse inductantie en zelfinductantie. Bereken de zelfinductantie L van een spoel. Bespreek de werking van de RL-kring, en leid daaruit verder een uitdrukking voor de energie opgeslagen in een stroomdoorlopen spoel af. Bespreek de analogie tussen de LC-kring en de harmonische oscillator.
        \item Bespreek hoe Maxwell er toe kwam om een term toe te voegen aan de kringwet van Ampère (geef de redeneringen in integrale en in differentiële vorm). Toon dan aan dat het elektromagnetisch veld in vacuüm aan de golf­betrekking voldoet. 
        \item Toon aan dat het elektromagnetisch veld in vacuüm aan de golfbetrekking voldoet.  Leid de algemene oplossing af voor vlakke mono­chroma­tische golven en sferische mono­chroma­tische golven. Interpreteer. Leid de eigenschappen van vlakke monochromatische golven in vacuüm af.
        \item Stel de algemene behoudsvergelijking op van elektromagnetische energie (stelling van Poynting).  Interpreteer de verschillende termen en geef hun dimensie. Definieer de “gemiddelde irradiantie” en bespreek de fysische betekenis ervan. Geef andere voorbeelden van behoudswetten.
        \item Bespreek het interferentie-experiment van Young. Leid de voorwaarden af voor constructieve en destructieve interferentie. Doe dit ook in het algemene geval van interferentie van vlakke monochromatische golven. Bespreek ook het verschil tussen coherente en incoherente superpositie.
        \item Bespreek de waarnemingen en experimenten die aantonen dat de uitwisseling van energie tussen elektromagnetische straling en materie in “energie-pakketjes” gebeurt. Welke universele wetten ken je?  Wat is de werkfunctie van een metaal, wat is de stoppotentiaal? Bereken deze laatste.
        \item Bespreek het experiment (inclusief de betrokken waarnemingen) dat aantoont dat fotonen een impuls hebben. Bereken het verband tussen impuls en golflengte en verklaar de waarnemingen.
        \item Bespreek het experiment van Michelson en Morley. Bereken wat je verwacht als resultaat van het experiment. Welk resultaat krijg je uiteindelijk?
        \item Geef de probleemstelling van de speciale relativiteit van Einstein. Leid de formules af van de Lorentz-transformaties. Wat is het verband met de Minkowski ruimte?  Toon aan in welk geval de Lorentz-transformaties equivalent zijn met de Galileotransformaties. Som kort de gevolgen van de speciale relativiteitstheorie op.
        \item Bespreek uitgebreid de gevolgen van de speciale relativiteitstheorie van Einstein. Bespreek ook “relativistische energie” en het gedachtenexperiment van Einstein om te komen tot E=mc2.
        \item Bespreek de voornaamste eigenschappen van atoomkernen en de begrippen massadefect en bindingsenergie. Hoe kan men uit atoomkernen energie vrijmaken. Bespreek de formule van Bethe-Weizsäcker voor de massa van een kern (de formule hoef je niet van buiten te kennen) en bespreek alle termen die erin voorkomen.
        \item Verklaar het concept werkzame doorsnede voor kernreacties. Geef het verband met de Wet van Beer-Lambert. Waarvoor kan deze wet gebruikt worden?
        \item Leg uit wat radioactief verval is en welke vervalwijzen je kent. Hoe bekijk je verval op de nuclidekaart. Geef eigenschappen van de verschillende soorten straling. Geef de vervalwet voor nucleair verval. Wat zijn halfwaardetijd en activiteit”? Leg uit hoe je hiermee de ouderdom van een archeologische vondst kan bepalen via de 14C-methode.
        \item Pas de wetten van Kirchhoff toe om elektrische schakelingen op te lossen. Pas ook de superpositiestelling toe om elektrische netwerken op te lossen en vind het Norton en Thévenin equivalent van schakelingen. Hoe kan je de methodes van de knooppuntpotentialen en van de maasstromen toepassen om stromen en spanningen in een elektrisch netwerk te vinden door gebruik te maken van een minimum aan vergelijkingen.
        \item Bepaal de sinusoïdale regimerespons en de overgangsverschijnselen in een RLC-netwerk.
    \end{enumerate}
    {\bf Specifieke eindcompetenties 26 tot 31 zijn alleen van toepassing voor de studenten 1 Ba Ingenieurswetenschappen en niet voor de studenten 1Ba Fysica en Sterrenkunde. Eindcompetenties 32 en 33 zijn alleen van toepassing voor de studenten 1Ba Fysica en Sterrenkunde en niet voor de studenten 1Ba Ingenieurswetenschappen, noch voor de studenten Wiskunde. Eindcompetenties 32 en 33 wordt enkel geëvalueerd tijdens het schriftelijke examen.}
\end{document}


\documentclass[12pt]{article}

\usepackage[dutch]{babel}
\usepackage{amsmath}

\title{Mechanica 2}
\author{Dries Van den Brande \and Andreas Declerck}

\begin{document}
    \maketitle

    \section{LAAG ENERGETISCHE KERNFYSICA EN BEGINSELEN VAN DE STRALINGSBESCHERMING}
    \subsection{Basisbeginselen}
    \begin{enumerate}
        \item \textbf{Bespreek de voornaamste eigenschappen van atoomkernen en de begrippen massadefect en bindingsenergie.} (HOOFDVRAAG)
        \item \textbf{Hoe kan men uit atoomkernen energie vrijmaken.} (HOOFDVRAAG)
        \item \textbf{Bespreek de formule van Bethe-Weizsäcker voor de massa van een kern en bespreek alle termen die erin voorkomen.} (HOOFDVRAAG)
        \item \textbf{Verklaar het concept werkzame doorsnede voor kernreacties. Geef het verband met de Wet van Beer-Lambert. Waarvoor kan deze wet gebruikt worden?} (HOOFDVRAAG)
        \item \textbf{Leg uit wat radioactief verval is en welke vervalwijzen je kent. Hoe bekijk je verval op de nuclidekaart. Geef eigenschappen van de verschillende soorten straling. Geef de vervalwet voor nucleair verval.} (HOOFDVRAAG)
        \item Defineer: molecule, atoomgetal, massagetal.
        \item Hoe worden moleculen gerangschikt?
        \item Defineer: periode, groep.
        \item Defineer: Ionisatieenergie.
        \item Defineer: atomic mass unit. Wat is de relatie tussen mol en u?
        \item Defineer: Thermonucleaire fusie.
        \item Wat is de oorzaak van bindingsenergie?
        \item Bespreek de kerndiameter.
        \item Defineer: Radioactiviteit, nuclide, isotopen.
        \item Geef de 2 isotopen van waterstof.
        \item Defineer: Fractionele isotopische samenstelling.
        \item Wat is een nuclidekaart? Wat leer je hieruit? Leg de kleurencode uit.
        \item Wat is de stabiliteitslijn? Waarom heb je voor een atoomgetal groter dan 20 extra neutronen nodig?
        \item Wat is de zwaarste stabiele kern? 
        \item Welke 3 effecten kunnen plaatsvinden door interactie van een foton met materie?
        \item Bespreek K-vangst, Nucleaire isomerie, spontane kernsplijting.
        \item Bespreek de 4 natuurlijke vervalreeksen.
        \item Bespreek de doordringingsvermogen van de verschillende deeltjes.
        \item Maak een onderscheid tussen bestraling en besmetting.
    \end{enumerate}
    \subsection{Halfwaardetijd en radioactiviteit}                                                                                                                                                                     
    \begin{enumerate}
        \item \textbf{Wat zijn halfwaardetijd en activiteit? Leg uit hoe je hiermee de ouderdom van een archeologische vondst kan bepalen via de 14C-methode} (HOOFDVRAAG)
        \item Bespreek specifieke activiteit.
        \item Wat is het verschil tussen Bq en Hz?
    \end{enumerate}
    
\end{document}


\documentclass[12pt]{article}

\usepackage[dutch]{babel}
\usepackage{amsmath}

\title{Mechanica 2}
\author{Dries Van den Brande \and Andreas Declerck}

\begin{document}
    \maketitle

    \section{INLEIDING TOT DE SPECIALE RELATIVITEIT}
    \subsection{Probleemstelling - De Lorentz-transformaties}
    \begin{enumerate}
        \item Welke gegevens wisten we in aan het einde van de 19e eeuw? Wat waren de problemen? Leg uit.
        \item Waarom hangt de fysica schijnbaar af van het perspectief?
        \item Wat stond er in de: "Einstein in Annalen der physik, 1905".
        \item Vat de grote vragen samen tot het ontdekken van de speciale relativiteitstheorie.
    \end{enumerate}
    \subsection{Het experiment van Michelson en Morley}
    \begin{enumerate}
        \item \textbf{Bespreek het experiment van Michelson en Morley. Bereken wat je verwacht als resultaat van het experiment. Welk resultaat krijg je uiteindelijk?} (HOOFDVRAAG)
        \item Bespreek het concept van het Michelson-Morley experiment.
        \item Leg uit hoe de Michelson interferometer Werktuigen
        \item Wat bewijst het experiment van Michelson-Morley?
    \end{enumerate}
    \subsection{De speciale relativiteitstheorie van Einstein}
    \begin{enumerate}
        \item \textbf{Geef de probleemstelling van de speciale relativiteit van Einstein.} (HOOFDVRAAG)
        \item \textbf{Leid de formules af van de Lorentz-transformaties. Wat is het verband met de Minkowski ruimte?} (HOOFDVRAAG)
        \item \textbf{Toon aan in welk geval de Lorentz-transformaties equivalent zijn met de Galileotransformaties.} (HOOFDVRAAG)
        \item \textbf{Som kort de gevolgen van de speciale relativiteitstheorie op.} (HOOFDVRAAG)
        \item Defineer: event.
        \item Defineer: observer.
        \item Hoe ijk ik de coördinaten en synchroniseer ik klokken in de speciale relativiteit?
        \item Geef en bespreek de eigenschappen van de Lorentz-transformaties
    \end{enumerate}
    \subsection{Gevolgen van de speciale relativiteitstheorie van Einstein}
    \begin{enumerate}
        \item \textbf{Bespreek uitgebreid de gevolgen van de speciale relativiteitstheorie van Einstein} (HOOFDVRAAG)
    \end{enumerate}
    \subsection{$E=m \cdot c^2$}
    \begin{enumerate}
        \item \textbf{Bespreek “relativistische energie” en het gedachtenexperiment van Einstein om te komen tot E=mc2.} (HOOFDVRAAG)
    \end{enumerate}
    
\end{document}

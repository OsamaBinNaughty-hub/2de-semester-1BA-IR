
\documentclass[12pt]{article}

\usepackage[dutch]{babel}
\usepackage{amsmath}

\title{Mechanica 2}
\author{Dries Van den Brande \and Andreas Declerck}

\begin{document}
    \maketitle

    \section{HET DEELTJESKARAKTER VAN ELEKTROMAGNETISCHE STRALING}
    \subsection{Zwarte straling}
    \begin{enumerate}
        \item \textbf{Bespreek de waarnemingen en experimenten die aantonen dat de uitwisseling van energie tussen elektromagnetische straling en materie in “energie-pakketjes” gebeurt. Welke universele wetten ken je?} (HOOFDVRAAG)
        \item Welk experiment toont aan dat er iets moet bestaan zoals zwarte straling?
        \item Defineer: Zwart Lichaam / Zwarte straling
        \item Wat waren de vaststellingen van Kirchoff over deze zwarte straling? Geef en leg uit: Model van Kirchoff.
        \item Defineer: Spectrale energiedichtheid / Spectrale irrandiantie. Wat is de relatie tussen deze twee?
        \item Wat zijn de eigenschappen en mysteries van de zwarte straling?
        \item Leg de hypotheses van Planck uit voor deze mysteries. Hoe kan je uit deze hypotheses de eigenschappen van de zwarte straling bevestigen? Wat waren de problemen van Planck over deze hypotheses? Wie en hoe werd dit opgelost?
        \item Leg uit hoe zwarte straling gebruikt wordt om temperatuur te meten op een afstand.
        \item Leg uit hoe zwarte straling gebruikt werd om de Big Bang en uitdeiding van het heelal te bevestigen.
        \item Wat is het algemeen besluit over zwarte straling?

        \item Bereken de maximale golflengte van:
            \begin{enumerate}
                \item ijs bij 273K
                \item lamp bij 3500K
                \item helium bij 4.273K
                \item universum bij T:2.725K
            \end{enumerate}
            Ga er van uit dat er zwarte straling is. In welke regio van het EM-spectrum bevinden deze golflengtes zich?
    \end{enumerate}
    \subsection{Het foto-elektrisch effect}
    \begin{enumerate}
        \item \textbf{Wat is de werkfunctie van een metaal, wat is de stoppotentiaal? Bereken deze laatste} (HOOFDVRAAG)
        \item Wat is het experiment dat leid to de ontdekking van het foto-elektrisch effect? Wat zijn de waarnemingen?
        \item Wat zijn de eigenschappen/mysteries van het foto-elektrisch effect? Wat zijn de vragen en klassieke redenaties hiervan?
        \item Wat is de hypothese van Einstein over het foto-elektrisch effect? Verklaar de mysteries. Bewijs de foto-elektrische vergelijking.
        \item Wat is het besluit van het foto-elektrisch effect?
    \end{enumerate}
    \subsection{Het Compton effect}
    \begin{enumerate}
        \item \textbf{Bespreek het experiment (inclusief de betrokken waarnemingen) dat aantoont dat fotonen een impuls hebben. Bereken het verband tussen impuls en golflengte en verklaar de waarnemingen.} (HOOFDVRAAG) 
        \item Wat is het algemeen besluit van het Compton effect?
    \end{enumerate}
    \subsection{Gravitational redshift}
    \begin{enumerate}
        \item Hoe kwamen ze op de gedachte dat fotonen een massa hebben? Waar ga je een antwoord vinden?
        \item Defineer de verschillende soorten massa's
        \item Leg uit: Equivalentieprincipe van Einstein
        \item Algemene relativiteitstheorie van herleid zich tot de wetten van Newton indien?
        \item Bewijs dat een foton een massa heeft.
        \item Waarom "valt" licht?
        \item leg uit: Gravitational redshift.
        \item Waarom heeft licht een duaal karakter? Heeft materie ook een duaal karakter?
    \end{enumerate}
    
\end{document}

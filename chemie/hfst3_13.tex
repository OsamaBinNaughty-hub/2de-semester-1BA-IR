\documentclass[a4paper,12pt]{article}

\usepackage[dutch]{babel}
\usepackage{amsmath}
\usepackage{amssymb}
\usepackage{physics}

\title{Formularium}
\author{Dries Van den Brande \and Andreas Declerck}

\newcommand{\R}{\mathbb{R}}
\newcommand{\Z}{\mathbb{Z}}


\begin{document}

    % Only things after \maketitle are added to master, all other things should be put in preamble
    \maketitle

    \section{Organische chemie: Reactiviteit}
    \subsection*{Substitutiereacties op alifatische, verzadigde C-atomen}
    \begin{enumerate}
        \item Geef en teken de algemene Substitutieregels. Wat zijn de 2 soorten reacties?
        \item Geef en teken de Nucleofiele / Elektrofiele substitutiereactie. Wat is de voorwaarde dat deze zou doorgaan?
        \item Wat maakt een goed nuceofiel uittredende structuureenheid? Geef voorbeelden van goede en slechte.
        \item Waarom is de bereiding alkylchloride uit alcohol met \ce{NaCl} onmogelijk? Wat is de oplossing. Pas deze oplossing toe op het bereiden van alkylchloride.
        \item Defineer: Elektrofiele preactivering.
        \item Defineer: $S_N1$-reactie. Geef ook de gedachtegang. Geef de omzetting van t-butylchloride in t-butyljodide. Wat is de snelheidsvergelijking van deze omzetting?
        \item Wanneer gebeuren $S_N1$-reacties? Leg alle types uit.
        \item Waarom gebeuren $S_N1$-reacties steeds met racemisatie? Leg uit en teken.
        \item Defineer: $S_N2$-reactie. Geef ook de gedachtegang. Geef de omzetting van methyljodide tot methylalcohol.
        \item Defineer: Walden inversie.
        \item Wanneer worden reacties stereospecifiek genoemd? Geef de types. Wat bewijst dit? Met wat treden $S_N2$-reacties op? Waarom?
        \item Geef toepassingen en voorbeelden van nucleofiele reacties.
        \item \textbf{De 2 bij $S_N2$-reacties slaat op? Bij een $S_N1$-reactie gebeurt steeds of nooit racemisatie?} (PPT-VRAAG)
        \item Defineer: Organometalen. Geef de 2 belangrijkste organometalen. Hoe worden deze meestal gesynthetiseerd? 
        \item Defineer: Grignard reagentia. 
        \item Hoe worden andere organometalen (tetraalkyltin-,tertaalkyllood-,dialkylkwikderivaten) gesynthetiseerd?
        \item Wat is de voorwaarde om organometalen te synthetiseren? Waarom?
        \item Defineer: $S_E1$-reacties. Waaraan zijn deze te wijten? Geef en leg uit de algemene methode. Pas toe op:
        \begin{enumerate}
            \item methyllithium met een waterproton.
            \item $R-MgCl$ + $H-Cl$ 
            \item $R-BeR$ + $H-NEt_2$
            \item $R-ZnR$ + $H-NH_2$
            \item $\ce{CH3}-\ce{G(CH3)_2}$ + $I-I$
        \end{enumerate}
        \item Waarom vinden $S_E1$-reacties plaats met racemisatie. Geef voorbeelden.
        \item Defineer: $S_E1$-reacties.  Pas toe op:
        \begin{enumerate}
            \item methyllithium met een waterproton.
            \item $CH_3-Pb(CH_3)_3$ + $Ag^+NO_3^-$ 
            \item $CH_3CH_2-Hg-I$ + $I-I$
            \item $CH_3-Sn(CH_3)_3$ + $H-Br$
        \end{enumerate}
    \end{enumerate}
    \subsection*{Elektrofiele addities op dubbele $C=C$ bindingen}
    \begin{enumerate}
        \item Defineer de elektrofiele addities van dihalogenen op de dubbele $C=C$ binding.  Wat gebeurt er als het alkeen niet symmetrisch gesubstitueerd is? Voor wat is dit nuttig?
        \item Bewijs experimenteel het bestaan van trans stereospecifiteit.
        \item Defineer de additie van waterstofhalogeniden op alkenen.
        \item Leg de Markovnikov-regel / Anti-Markovnikov-regel uit.
    \end{enumerate}
    \subsection*{Eliminatie reacties}
    \begin{enumerate}
        \item Defineer E1-eliminaties.
        \item Defineer $\beta-$ / $\alpha-$eliminaties.
        \item Defineer E2-eliminaties. Wat is de voorwaarde voor een E2-eliminatie?
        \item Wat zijn organische oxidaties? Leg de oxidatie van een primair alcohol naar aldehyde uit. Geef schematische voorstellingen van:
        \begin{enumerate}
            \item alcohol $\rightarrow$ aldehyde
            \item aldehyde $\rightarrow$ carbonzuur
            \item alcohol $\rightarrow$ keton
        \end{enumerate}
        \item Kunnen tertiaire alcoholen oxideren? Waarom wel/niet ?
    \end{enumerate}
    \subsection*{Additiereacties op $C=O$ bindingen}
    \begin{enumerate}
        \item Wat is de algemene eigenschap van de carbonylfunctie in aldehyden en ketonen?
        \item Defineer grignardreacties op aldehyden en ketonen.
        \item Defineer grignardreacties op kooldioxide tot carbonzuren.
        \item Wat zijn aldoreacties. Toon aan met aceton.
    \end{enumerate}
    \subsection*{Reacties met carbonzuurderivaten}
    \begin{enumerate}
        \item Wat is de algemene eigenschap van de carbonylfunctie in carbonzuur en carbonzuurderivaten?
        \item Hoe worden esters gesynthetiseerd?
        \item Hoe worden carbonzuur chloriden gesynthetiseerd?
        \item Hoe worden amiden gesynthetiseerd? Wat gebeurt er met de nevenproducten? Wat is hier een gevolg van?
        \item Hoe worden esters gehydrolyseerd? Geef de ook de globale stoichiometrie. Wat zijn de gevolgen?
        \item Defineer: verzeping / zeep / vetzuren / detergenten.
        \item Hoe werkt zeep?
        \item Hoe worden zepen gesynthetiseerd?
    \end{enumerate}
    \subsection*{Aromatische Substitutiereacties}
    \begin{enumerate}
        \item Defineer: Aromaciteit. Wat is een gevolg van Aromaciteit? Wat verklaart dit?
        \item Hoe brommeer je benzeen?
        \item Maak TNT.
        \item Wat is de Friedel-Craft reactie? Leg beide toepassingen uit.
        \item Wat zijn elektrofiele substituties op monogesubstitueerde derivaten? Geef een voorbeelden + uitleg voor alle vormen.
        \item Waarom zijn aromatische nucleofiele substituties op benzeenderivaten niet mogelijk? Geef een uitzondering.
        \item Zet benzeen diazoniumkation om tot fenol met water.
        \item Wanneer gebeurt aromatische nucleofiele substitutie als additie-eliminatie? 
        \item Wanneer gebeurt nucleofilie verlies op de substitutiesite?
        \item Wat gebeurt er na het stabiliseren van een negatieve lading van een intermediair carbonion? Geef een voorbeeld.
        \item Hoe wordt methyloranje gesynthetiseerd?
    \end{enumerate}
\end{document}
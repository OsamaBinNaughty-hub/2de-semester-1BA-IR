\documentclass[a4paper,12pt]{article}

\usepackage[dutch]{babel}
\usepackage{amsmath}

\title{Mechanica 2}
\author{Dries Van den Brande \and Andreas Declerck}

\begin{document}

    % Only things after \maketitle are added to master, all other things should be put in preamble
    \maketitle

    \section{Organische chemie: structuren en naamgeving}
    \subsection*{Inleiding}
    \begin{enumerate}
        \item Defineer: Organische chemie.
        \item Welke verbindingen kan een C-atoom aangaan? Wat is een direct vervolg hiervan? 
        \item Wanneer wordt het C-atoom \textbf{verzadigd} of \textbf{onverzadigd} genoemd?
        \item Defineer: koolwaterstoffen. Uit wat worden deze ontgonnen? Wat wordt er afgezondert bij een destilatie van dit product.
        \item Defineer: verzadigde koolwaterstoffen.
    \end{enumerate}

    \subsection*{Koolwaterstoffen}
    \begin{enumerate}
        \item Defineer: Alkanen. 
        \item Geef de namen van de lineaire alkanen (1-10). 
        \item Geef de verschillende 'old school' grafische voorstellingen van \ce{C4H10}. Waarom zijn deze onpraktisch?
        \item Defineer: conformationele flexibiliteit. 
        \item Teken en benoem de mogelijke vertakte isomeren van butaan en pentaan. 
        \item Wanneer daalt het kookpunt/smeltpunt bij organische moleculen? 
        \item Wat is de algemene molecuulformule van een cyclische alkaan (1 ringstructuur)? Waarom is deze anders?
        \item Defineer: Onverzadiging van een cyclische alkaan. Wat moet men doen per ingevoerde onverzadiging?
        \item Geef alle mogelijke bindingspatronen van een verzadigd hexaan isomeer dat de formule \ce{C6H8} heeft. Hoeveel onverzadigingen heeft dit isomeer?
        \item Wat zijn de voorwaarden om het voorvoegsel 'cyclo-' te gebruiken? Teken cyclopropaan /cyclobutaan /cyclopentaan /cyclohexaan.
        \item Hoe bekomt men een substituenten? Hoe stelt men de KWS voor met een substituent.
        \item Hoe wordt de basisnaam van een substituent bepaald?
        \item Teken en benoem de substituenten van:
                \begin{enumerate}
                    \item methaan / ethaan
                    \item propaan / cyclopropaan
                    \item butaan / isobutaan / cyclobutaan
                \end{enumerate}
        \item Wat zijn de voorwaarden om het voorvoegsel 'sec-' en 'tert-' te gebruiken?
        \item Defineer: primair en quaternair C-atoom. 
        \item Waarom moet de hoofdketen genummerd worden?
        \item Wanneer gebruikt men multiplicerende voorvoegsels?
        \item Hoe vernoem je radicalen afgeleid uit alkanen?
        \item Hoe worden meerdere gelijke samengestelde substituenten aangeduid? 
        \item Hoe duid men het aantal dubbele koolstof bindingen aan in de hoofdketen van een alkaan?
        \item Hoe duid men het aantal meervoudige bindingen aan in de hoofdketen van een alkaan? 
        \item Defineer: Aromatische koolwaterstoffen. 
        \item Defineer: Digesubstitueerde benzeenringen. 
        \item Wat zijn de voorwaarden om de voorvoegsels 'o-' / 'm-' / 'p-' te gebruiken?
        \item Teken de organische structuren: (oplossingen staan in volgorde in het boek)
            \begin{itemize}
                \item 3-methylpentaan
                \item methylcyclohexaan
                \item 2,3,3,4-tetramethylpentaan
                \item 5,5-di-t-butylnonaan
                \item 2-methylpentaan
                \item 2,3,5-trimethylhexaan
                \item 2,7,8-trimethyldecaan
                \item 4-isopropyl-5-porpyloctaan
                \item 3-ethyl-2-methylheptaan
                \item 5-methyl-4-propylnonaan
                \item 1-methylpentyl
                \item 2-methylpentyl
                \item 3-methylpentyl
                \item 4-methylpentyl
                \item 4,4-diethyloctaan
                \item 5,6-bis(1,1-dimethylpropyl)decaan
                \item 2-hexeen
                \item 1,4-hexadieen
                \item alleen 
                \item isopreen
                \item cyclohexeen
                \item vinyl
                \item allyl
                \item butenyn
                \item acetyleen
                \item ethynyl
                \item 3-penteen-1-yn                                                                                                                                                   
                \item 1,4-cyclohexadieen
                \item cyclooctaterta"een
                \item 1,3-hexadieen-5-yn
                \item 1-penteen-4-yn
                \item benzeen
                \item naftaleen
                \item anthraceen
                \item fenantreen
                \item o-benzeen / m-benzeen / p-benzeen
                \item 1-ethyl-4-methylbenzeen
                \item 1,2,3-trimethylbenzeen
                \item styreen
                \item tolueen
                \item o-xyleen
                \item benzyl
                \item fenyl
                \item o-tolyl
            \end{itemize}
        \subsection*{Organische functies of functionele groepen}
        \begin{enumerate}
            \item Defineer: Organische functie, Hetero-atomen, Mono-functionele verbindingen
            \item Geef de 2 soorten substituenten. Leg ze ook uit.
            \item Wat zijn carbonylgroep en carboxylgroep?
            \item Waarom is Nitril een carbonzuur derivaat?
            \item Geef de formule en alle mogelijken aarden van de substituenten van de volgende functionele groepen:
                \begin{itemize}
                    \item alifatisch halogenide
                    \item aromatisch halogenide
                    \item alcohol
                    \item enol
                    \item fenol
                    \item ether
                    \item primaire / secundaire / tertiaire amine 
                    \item nitroverbinding
                    \item aldehyde
                    \item keton 
                    \item carbonzuur
                    \item ester
                    \item carbonzuur anhydride
                    \item carbonzuur halogenide
                    \item amide en primaire / secundaire / tertiaire amide
                    \item nitril
                \end{itemize}
            \item Geef de 2 types nomenclatuur. Leg ze ook uit.
            \item Leg uit: substitutieve naamgeving van organische halogeniden
            \item Leg uit: radicofunctionele naamgeving van organische halogeniden
            \item Leg uit: substitutieve naamgeving van alcoholen + speciaal geval
            \item Geef de voorangregels van de substituenten
            \item Leg uit: radicofunctionele naamgeving van alcoholen
            \item Leg uit: naamgeving van fenolen
            \item Leg uit: substitutieve naamgeving van ethers
            \item Leg uit: radicofunctionele naamgeving van ethers
            \item Leg uit: radicofunctionele naamgeving van primaire aminen 
            \item Leg uit: substitutieve naamgeving van primaire aminen en polyaminen + speciaal geval
            \item Leg uit: naamgeving van secundaire en tertiaire aminen 
            \item Leg uit: naamgeving van nitroderivaten
            \item Leg uit: naamgeving van aldehyden + speciaal geval
            \item Leg uit: naamgeving van ketonen
            \item Wanneer mag het voorvoegsel 'oxo' gebruikt worden?
            \item Leg uit: naamgeving van carbonzuren en carbonzuurderivaten + speciaal geval
            \item Leg uit: naamgeving van esters
            \item Leg uit: naamgeving van N-gesubstitueerde amiden
            \item Teken de organische structuren: (oplossingen staan in volgorde in het boek)
                \begin{itemize}
                    \item 2-chloorbutaan
                    \item p-broomchloorbenzeen
                    \item 1,2-dibroommethaan
                    \item 1,4-dijoodcyclohexaan
                    \item 1,1-dichloorethaan
                    \item 1,2-dichloorethaan
                    \item tetrachloormethaan
                    \item methylchloride
                    \item benzyljodide
                    \item tert-butylbromide
                    \item chloroform
                    \item koolstoftetrachloride
                    \item 2-propanol
                    \item 1,4-butaandiol
                    \item 2-propyl-2-penteen-1-ol
                    \item 2-cyclohexeen-1-ol
                    \item 2-hydroxymethyl-1,5-pentaandiol
                    \item ethylalcohol
                    \item tert-butylalcohol
                    \item allylalcohol
                    \item benzylalcohol
                    \item 4-methyl-1,2-benzeendiol
                    \item 1,2,4-benzeentriol
                    \item fenol
                    \item 2-naftol
                    \item ethoxyetheen
                    \item 1-broom-2-propoxyethaan
                    \item 1,2,4-trimethoxybenzeen
                    \item 1-isopropoxybutaan
                    \item ethylmethylether
                    \item diethylether
                    \item bis(1-chloorethyl)ether
                    \item isopropylamine
                    \item cyclohexylamine
                    \item 2-naftylamine
                    \item aniline
                    \item 2-propaanamine / 2-aminopropaan
                    \item 1,3-benzeendiamine / 1,3-diaminobenzeen
                    \item 2-aminomethyl-1,5-pentaandiamine
                    \item 1-(N,N-dimethylamino)propaan
                    \item N-methylaminocyclopentaan
                    \item N-methyl-N-propylaniline
                    \item 2-(N-methylamino)naftaleen
                    \item 3-(N-ethyl-N-methylamino)hexaan
                    \item nitromethaan
                    \item 1,3,5-trinitrobenzeen
                    \item ethanal
                    \item hexaandial
                    \item 5-allyl-2-hexeendial
                    \item formaldehyde
                    \item acetaldehyde
                    \item propionaldehyde
                    \item butyraldehyde
                    \item benzaldehyde
                    \item formylcyclohexaan
                    \item 3-formylpentaandial
                    \item 2-pentanon
                    \item 2,4-pentaandion
                    \item 5-hexeen-2-onpraktisch3-allyl-2,4-pentaandion
                    \item fenyl-1-butanon
                    \item cyclopentanon
                    \item cyclohexyl-1-propanol
                    \item aceton
                    \item acetofenon
                    \item benzofenon
                    \item 3-methyl-1-pentaancarbonzuur
                    \item 2-butaancarbonzuur
                    \item 1,2-cyclohexaandicarbonzuur
                    \item 3-pyrideinecarbonzuur
                    \item 2-methyl-2-butaancarboxylaat
                    \item 3-hydroxy-1-propaancarbonylchloride
                    \item 3-amino-cyclopentaancarbonamide
                    \item 3,5-dioxo-1-pentaancarbonzuur
                    \item cyclohexaancarbonitril
                    \item ethyl 2-methyl-2-butaancarboxylaat
                    \item n-propyl 4-aminocyclohexaancarboxylaat
                    \item ethyl acetaat
                    \item isopropyl formiaat
                    \item N,N-dimethylbutyramide
                    \item N-ethyl-N-methyl-3-hydroxy-1cyclopentaancarbonamide
                    \item N-methylacetamide
                    \item 2-carboxymethyl-1,5-pentaandicarbonzuur
                    \item 3-acetyloxypropionzuur
                    \item 4-cyaanboterzuur
                \end{itemize}
        \end{enumerate}
        \subsection*{Isomerie en isomeren in de organische chemie}
        \begin{enumerate}
            \item Defineer: isomeren
            \item Wat bepaald het type isomerie?
            \item Defineer: Constitutie van een moleculen / Constitutionele isomeren.
            \item Geef alle mogelijke isomeren voor alle verbindingen met molecuulformule \ce{C2H4O} / \ce{C5H12} / \ce{C4H9Cl}
            \item Defineer: stereoisomeren. 
            \item Wat zijn de verschillenden soorten stereoisomeren? 
            \item Wat zijn cis/trans isomeren? Uit wat is dit een gevolg?
            \item Defineer: Enantiomeren. geef voorbeelden. geef toepassing.
            \item Waaraan zijn enantiomerie te wijten?
            \item Defineer: Chiraliteit. 
            \item Hoe kan je grafisch een enantiomeer onderscheiden?
            \item Defineer: optische activiteit. Waarom hebben Enantiomeren andere optische activiteit?
            \item Defineer: Achirale molecule / Racemisch mengsel
            \item Defineer: Diastereomeren. geef voorbeelden.
            \item Defineer: configuratie / conformatie
            \item Leg de isometrie in dichloorcyclopentaan uit.
            \item Wat zijn de algemene regels voor reactiviteit met enantiomeren?
            \item Hoe gaat men enantiomeren scheiden?
            \item Conformeer ethaan als een Newmanprojectie.
        \end{enumerate}
    \end{enumerate}

    \subsection*{Isomeren}
    \begin{enumerate}
	    \item Wat is de definitie van een isomeer?
	    \item Wat is de definitie van een constitutionele isomeer?
	    \item Wat is de definitie van een stereoisomeer?
	    \item Leg de CIS/TRANS nomenclatuur uit (en geef ook de nieuwe benamingen.
	    \item Wat zijn enantiomeren?
	    \item Wat is chiraliteit?
	    \item Wat is achiraliteit?
	    \item Wat is een racemisch mengsel?
    \end{enumerate}
\end{document}

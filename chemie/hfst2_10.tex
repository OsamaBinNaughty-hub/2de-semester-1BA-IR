\documentclass[a4paper,12pt]{article}

\usepackage[dutch]{babel}
\usepackage{amsmath}

\title{Mechanica 2}
\author{Dries Van den Brande \and Andreas Declerck}

\begin{document}

    % Only things after \maketitle are added to master, all other things should be put in preamble
    \maketitle

    \section{Ruimtelijke- en elektronenstructuur van moleculen}

    \begin{enumerate}
        \item Wat is de VSEPR-theorie?
        \item Wat is een elektron domein?
        \item Geef 4 verschillende voorstellingen van ammoniak nodig om de structuur te begrijpen.
        \item Waarom heb je verschillende hoeken bij \ce{CH4}, \ce{NH3}, \ce{H2O}?
        \item Wanneer is een binding covalent, polair covalent of ionair?
        \item Wat is een dipoolmoment en bindingsmoment in de chemie?
        \item Zet 1 D(ebye) om naar SI-eenheden.
        \item Wat is het dipoolmoment van \ce{CO2}?
        \item Geef de definitie van intermoleculaire krachgten.
        \item Beschrijf de dipool-dipool interactie en zijn effect op het kookpunt.
        \item Beschrijf de ge\"induceerde krachten of gedisperceerde London krachten.
        \item Geef 3 parameters die de sterkte van de intermoleculaire krachten bij apolaire stoffen bepalen.
        \item Beschrijf waterstofbruggen.
        \item Geef de definitie van oppervlaktespanning.
        \item Geef de definitie van capillariteit.
    \end{enumerate}
\end{document}
\documentclass[a4paper,12pt]{article}

\usepackage[dutch]{babel}
\usepackage{amsmath}

\title{Mechanica 2}
\author{Dries Van den Brande \and Andreas Declerck}

\begin{document}

    % Only things after \maketitle are added to master, all other things should be put in preamble
    \maketitle

    \section{Polymeren}
    \subsection*{Definities}
    \begin{enumerate}
        \item Defineer: Polymeer / mesomeer.
        \item Welke polymeren onderschijden we? Defineer elke subcategorie.
    \end{enumerate}
    \subsection*{Plastics of polymeren?}
    \begin{enumerate}
        \item Wat is het verschil tussen plastic en een polymeer?
    \end{enumerate}
    \subsection*{Classificatie van polymeren}
    \begin{enumerate}
        \item Uit wat resulteren kunsstoffen?
        \item Wat is onderverdeling van polymeren?
        \item Defineer: Morfologie. Wat zijn de verschillende soorten? Defineer de soorten?
        \item Wat zijn de eigenschappen van de verschillende soorten chemische structuren bij polymeren?
        \item Defineer: Oligomeer. Geef een voorbeeld.
        \item Welk concept moeten we invoeren om de thermische eigenschappen van polymeren te bestuderen?
        \item Defineer: Thermoplasten / Thermoharder / Elastomeer / Elastoplast.
        \item Leg gebruiktclassificatie uit. Waarom slecht?
        \item Defineer de 2 grote synthesemethodes voor polymeren.
    \end{enumerate}
    \subsection*{Ketengroei polymeren}
    \begin{enumerate}
        \item Defineer:
        \begin{enumerate}
            \item polyethyleen
            \item polpropyleen
            \item polyvinylchloride (PVC)
            \item polystyreen
            \item poly(fluorethyleen)     Teflon \textregistered
            \item poly(methyl methacrylaat)    Plexiglas \textregistered
            \item poly(acrylnitril)
            \item poly(vinylacetaat)
            \item poly(ethylacrylaat)
        \end{enumerate}
        \item Wat is radicalaire polymerisatie van alkenen? Toon dit aan met de synthese van LDPE. Naar wat streven gesubstitueerde alkenen? Wat gebeurt er met niet-symmetrische alkenen?
        \item Wat is kationische polymerisatie van alkenen? Toon dit aan. Wat gebeurt er bij niet stabiliseerbare intermediaire kationen? Geef een voorbeeld met de synthese van polyisobuteen.
        \item Wat is anionische polymerisatie van alkenen? Toon dit aan theoretisch en met een voorbeeld.
        \item Hoe worden ketenvertakkingen gevormd? Geef de twee manieren. Wat is het verschil tussen deze twee?
        \item Wat is stereochemie? Hoe wordt deze bepaald op een stereogene koolstofatomen? 
        \item Defineer de verschillende soorten stereochemie van polymeren.
        \item Waarom hebben polymeren met verschillende tactiviteit verschillende chemische / fysische eigenschappen?
        \item Waarom zijn atactische polyalkenen ook vertakt?
        \item Waarvoor worden de Ziegler-Natta katalysatoren gebruikt? Wat zijn de voordelen van deze katalysatoren? Waarom zijn deze stijver/sterker. Geef het vereenvoudigd mechanisme.
        \item Wat is een gevolg na polymerisatie van alkadiënen? Teken alle mogelijke dingetjes.
        \item Defineer: Natuurlijke rubbers. Wat is de andere isomerie van deze rubber? Hoe maak je de kunstmatige rubber Neopreen? 
        \item Wat is vulcanisatie van rubber? In welk opzicht verandert het rubber?
    \end{enumerate}
    \subsection*{Copolymeren}
    \begin{enumerate}
        \item Defineer: Copolymeren / Homopolymeer. Geef en teken de soorten copolymeren.
        \item Maak styreen-butadieenrubber. Wat zijn de eigenschappen afzonderlijk en als copolymeer?
        \item Maak verf. Wat zijn de eigenschappen afzonderlijk en als copolymeer?
        \item Wat zijn toepassingen van anker copolymeren?
        \item Geef de bereidingsmethoden voor elke soort copolymeren.
    \end{enumerate}
    \subsection*{Stapgroei polymeren}
    \begin{enumerate}
        \item Defineer: Stapgroei polymeren. Geef de algemene vorsteling. Wat wordt er bedoelt met ideale alternerende copolymeren?
        \item Hoe worden polyamiden gesynthetiseerd? Welke isomerie kunnen deze aannemen? Waarom?
        \item Maak Nylon6.
        \item Defineer: Proteïnen / peptide bindingen. Geef de soorten peptide bindingen. Wat zijn opmerkelijke eigenschappen van natuurlijke proteïnen?
        \item Hoe worden polyesters gesynthetiseerd? Geef de algemene voorstelling. Waarom is deze reactie aflopend?
        \item Hoe worden polycarbonaten gesynthetiseerd? Stel voor door synthese van Lexan \textregistered.
        \item Hoe worden polyurethanen gesynthetiseerd? Hoe wordt urethaan gesynthetiseerd? Waarom is polyurethaan wel/geen condensatiepolymeer? Toon aan.
        \item Geef gedachtegang voor PU gebruikt als schuimrubbers. Geef ook de chemische vergelijking.
        \item Defineer: Covalente netwerkpolymeren. Maak Bakeliet.
        \item Hoe wordt de molmassa van polymeren berekent? Defineer de nieuw ingevoerde eenheid.
    \end{enumerate}
    \subsection*{Covalente netwerkpolymeren}
    \begin{enumerate}
        \item 
    \end{enumerate}
    \subsection*{Molmassa's van polymeren}
    \begin{enumerate}
        \item 
    \end{enumerate}
\end{document}
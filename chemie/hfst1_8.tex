
\documentclass[a4paper,12pt]{article}

\usepackage[dutch]{babel}
\usepackage{amsmath}

\title{Mechanica 2}
\author{Dries Van den Brande \and Andreas Declerck}

\begin{document}

    % Only things after \maketitle are added to master, all other things should be put in preamble
    \maketitle

    \section{De elektronenconfiguratie van atomen}
    \subsection{Inleiding}
    \begin{enumerate}
        \item Waarom kan je debateren dat deze 'statement' niet klopt: Atomen zijn ondeelbare deeltjes in chemische reacties.
        \item Uit wat bestaat materie zelden in standaardomstandigheden? Zijn er uitzonderingen? Hoe kunnen we deze staat krijgen?
        \item Wat zijn vrije atomen?
        \item Wat zijn moleculen?
        \item Uit welke deeltjes bestaan polymeren, metalen en keramische materialen?
    \end{enumerate}

    \subsection{Ionisatie energie van atomen}
    \begin{enumerate}
        \item Wat is de ionisatie-energie van een atoom? Aan wat is dit gelijk in absolute waarden?
        \item Je hebt eerste/tweede/... ionisatie-energie. Hoe ver kan je gaan?
        \item Waarom is de tweede ionisatie-energie kleiner dan de eerste ionisatie-energie?
        \item Hoe verlopen de ionisatie-energieën in functie van het periodiek systeem?
        \item Bewijs dat elektronenconfiguratie bestaat in lagen (schillen). Doe dit met Natrium.
        \item Wat zijn de besluiten genomen in verband met de elektronenconfiguratie?
    \end{enumerate}

    \subsection{Interactie elektromagnetische golf-materie}
    \begin{enumerate}
        \item Wat bevestigt de studie van de interactie van elektromagnetische straling? Leg uit.
        \item Wat bedoelt men met 'FINGERPRINT' van een atoom?
        \item Wat is een golf?
        \item Wat is een elektromagnetische golf?
        \item Wat is elektromagnetische straling?
        \item Wat is het gemeenschappelijk kenmerk van de elektromagnetische golven? Maak een tekening.
        \item Wat zijn de belangrijke parameters van elektromagnetische golven?
        \item Geef de 'brede waaier' van elektromagnetische golven. Hoe verloopt de energie/frequentie/golflengte in deze waaier?
        \item Wat is de golflengte van zichtbaar licht?
        \item Waarom is de energie recht evenredig aan de frequentie bij elektromagnetische golven?
        \item Wat bedoelt men met witspectrumlicht?
        \item Hoe creëert men licht van geatomiseerd gas in een ontladingsbuis? Waarom is dit geëmiteerd licht niet continue? Hoe noemt men dit spectrum?
        \item Wat is atomenemissie? Geef een tekening. Geef voorbeelden.
        \item Wat is de emperische gelijkheid van Balmer?
        \item Leg uit: Atomenemissie spectrum is het element van de 'FINGERPRINT'.
        \item Leg uit wat het absorptiespectrum is en het verschil met het emissiespectrum. Wat leert men hier uit?
        \item Wat zijn de uiteindelijke conclusies van de energieverdeling in de materie? Welke ontdekkingen complementeren deze energieverdeling?
        \item Geef de stelling van Planck.
        \item Waarop geeft de stelling van Planck antwoord?
        \item Wat zijn kwanta?
        \item Wat is kwantum?
        \item Wat is de constante van Planck?
        \item Hoe kan men met submicroscopische deeltjes en macroscopische voorwerpen een verband leggen in de klassieke mechanica?
    \end{enumerate}

    \subsection{Dualiteit deelte-golfkarakter van een elektron in de materie}
    \begin{enumerate}
        \item Wat kun je vertellen over de interferentie franjes van Young?
        \item Wat zijn X-stralen en geef een toepassing?
        \item Wat is een diffractiepatroon?
        \item Waarom is het diffractiepatroon van aluminiumfolie zo speciaal? (voorbeeld)
        \item Wat is het uiteindelijk besluit uit de interferentiefranjes van Young? Is hier een belangrijke voorwaarde aan gehecht? (hint: ja)
        \item Wat is een toepassing van het dualiteitprincipe?
        \item Wat zegt de wet van De Broglie?
        \item Wat zijn de voorwaarden van een De Broglie-golf?
        \item Wat is de relevantie bij submicroscopische deeltjes met de wet van De Broglie?
        \item Wat is de niet-relevantie bij macroscopische deeltjes met de wet van De Broglie?
        \item Leg de planetaire visie en zijn beperkingen uit.
        \item Geef de formule voor de bindingsenergie in het Bohr-model. Wat leer je hier uit? Waarom is hier de absorbtie- en emissiespectra perfect te verklaren?
        \item Wanneer is een energieovergang mogelijk?
        \item Wat is de stralingsfrequentie?
        \item Bereken $\lambda_{4 \rightarrow 2}$. 
        \item Bewijs de emperische relatie van Balmer.
        \item Geef de 4 lijnenreeksen. Wat leer je hieruit?
        \item Wat is de laagste energietoestand?
        \item Bewijs dat de ionisatie-energie de positieve laagste energietoestand is.
        \item Wat zijn de tekortkomingen van het Bohr-model?
    \end{enumerate}

    \subsection{Atoomorbitalen en elektronenconfiguratie in meerelektrone atomen}
    \begin{enumerate}
        \item Leg uit: onzekerheidsprincipe van Heisenberg.
        \item Wat is de wiskundige formuleren van het onzekerheidsprincipe van Heisenberg. Wat leer je hier uit?
        \item Pas het onzekerheidsprincipe van Heisenberg toe op een elektron ($x=1 A  , \Delta x = 0.3 A$). Wat leer je hier uit?
        \item Pas het onzekerheidsprincipe van Heisenberg toe op een golfbal ($m=45.9g , v=200 \frac{m}{s}, \Delta x = 1mm$). Wat leer je hier uit?
        \item Hoe worden orbitalen wiskundig beschreven?
        \item Geef en leg de vergelijkingen van Schrödinger uit.
        \item Geef alle kwantumgetallen, hun betekenis en hun mogelijke theoretische waarden.
        \item Leg het experiment uit dat tot de ontdekking van het 4de kwantumgetal heeft geleid. Waarom is dit kwantumgetal het buitenbeetje van de kwantumgetallen? Waarom is deze dan ingevoerd?
        \item Geef alle kwantumgetal combinaties tot $n=4$.
        \item Wat zijn de 2 soorten grafische voorstellingen van de orbitalen? Leg ze ook uit.
        \item Bereken de onzekerheid op de impuls $\Delta p$ van een golfbal met massa $m = 4.59e-2$ en
        snelheid $v = 55.56 m/s$ waarvan we de positie tot op 1mm nauwkeurig kennen en trek conclusies uit de gevonde waarde.
        \item Leg het onzekerheidsprincipe van Heisenberg uit + formule.
        \item Geef alle kwantumgetallen, hun betekenis en hun mogelijke theoretische waarden.
        \item Leg het experiment uit dat tot de ontdekking van het 4de kwantumgetal heeft geleid.
        \item Wat zijn knooppunten in orbitalen (schets een grafiek).
        \item Beschrijf alle orbitalen.
        \item Wat is het uitsluitingsprincipe van Pauli?
        \item Wat is de regel van Hund? Zijn er uitzonderingen?
        \item Waarom volgt Chroom (Cr) niet de theoretische verdeling van elektronen?
        \item Wat is de definitie van valentie?
        \item Geef de 2 mogelijke definities voor de straal van een atoom.
        \item Leg het verband tussen de covalente straal, het periodiek systeem en de ionisatie-energie uit.
    \end{enumerate}

\end{document}
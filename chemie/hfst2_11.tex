\documentclass[a4paper,12pt]{article}

\usepackage[dutch]{babel}
\usepackage{amsmath}
\usepackage{amssymb}
\usepackage{physics}

\title{Formularium}
\author{Dries Van den Brande \and Andreas Declerck}

\newcommand{\R}{\mathbb{R}}
\newcommand{\Z}{\mathbb{Z}}


\begin{document}

    % Only things after \maketitle are added to master, all other things should be put in preamble
    \maketitle

    \section{Moleculaire orbitalen en structuren}

    \begin{enumerate}
        \item Wat zijn de limieten van de VSEPR-theorie?
        \item Wat is overlap in de kwantum chemie?
        \item Wat is een bindend elektronenpaar?
        \item Waarom onstaat er steeds bij atoomoverlap een moleculaire orbitaal?
        \item Beschrijf de gevormde orbitalen bij de vorming van \ce{H2} uit 2 \ce{H}-atomen.
        \item Wat is hybridisatie van atoomorbitalen en geef de wiskundige betekenis.
        \item Beschrijf orbitalair het ganse structuurvormingsproces van \ce{BeH2}.
        \item Beschrijf orbitalair het ganse structuurvormingsproces van \ce{BF3}.
        \item Beschrijf orbitalair het ganse structuurvormingsproces van \ce{CH4}.
        \item Beschrijf orbitalair het ganse structuurvormingsproces van \ce{PF5}. 
        \item Beschrijf orbitalair het ganse structuurvormingsproces van \ce{SF6}.
        \item Wat is de voorwaarde omdat men hybridisatie kan toepassen?
        \item Wat is een $\sigma$-binding?
        \item Wat is een $\sigma$-bindings geraamte?
        \item Wanneer kun je afwijkingen hebben bij de hybrides en hoe zou je dit kunnen verklaren?
        \item Wat zijn $\pi$-orbitalen?
        \item Beschrijf orbitalair het ganse structuurvormingsproces van \ce{CH2} (2 $C$-atomen en 4 $H$-atomen).
        \item Kunnen $\pi$-bindingen draaien en zo ja, in welke omstandigheden?
        \item Geef alle mogelijke combinaties orbitalen die samen een  $\pi$-binding kunnen vormen.
        \item Beschrijf  $\pi$-elektronendelokalisatie.
        \item Als n atoomorbitalen overlappen, hoeveel molecuuloribtalen worden er dan gevormd?
        \item Wat zijn K\'ekul\'estructuren?
        \item Beschrijf met de molecuul orbitalen theorie de structuur van een benzeen-ring.
        \item Geef een nodige voorwaarde opdat een molecule aromatisch kan genoemd worden.
        \item De stof \ce{NO3-} heeft een gedelokaliseerd $\pi$-elektronenpaar. Welke eigenschappen zou je uit hieruit kunnen afleiden? 
        \item Geef een definitie van een elektrofiel.
        \item Geef een definitie van een nucleofiel.
        \item Wat is een elektrodefici\"ent atoom?
        \item Bespreek de reactie tussen een elektrofiel en een nucleofiel op thermodynamisch niveau.
        \item Wat is een \textbf{heterolytische reactie}?
        \item Wat is een \textbf{homolytische reactie}?
        \item Wat is een radicaal?
        \item In welk soort milieu worden heterolytische en homolytische reacties bevorderd?
        \item Wat wordt er bedoelt met inductieve effecten?
        \item Tot hoe ver hebben de meeste inductieve effecten hun effect?
        \item Geef een gevolg van mesomere effecten.
    \end{enumerate}
\end{document}
\documentclass[a4paper,12pt]{article}

\usepackage[dutch]{babel}
\usepackage{amsmath}
\usepackage{amssymb}
\usepackage{physics}

\title{Formularium}
\author{Dries Van den Brande \and Andreas Declerck}

\newcommand{\R}{\mathbb{R}}
\newcommand{\Z}{\mathbb{Z}}


\setcounter{section}{6}

% Everything in this region will be overriden
\begin{document}

    \maketitle
    
    \section{Aggregatietoestanden en toestandsovergangen. Fasenleer}

    \subsection{Aggregaattoestanden en toestandovergangen bij zuivere stoffen}
    \begin{enumerate}
        \item Wat zijn de 3 voornaamste aggregaattoestanden?
        \item Leg uit: gecondenceerde toestand.
        \item Warmte-inbreng in een zuivere stof resulteerd in?
        \item Faseovergang van een zuivere stof gebeurt bij een constante ... ? 
        \item Faseovergang van een zuivere stof is afhankelijk van?
        \item Wat zijn alle mogelijke toestandsovergangen?
        \item Waarom is de energie nodig om te sublimeren groter dan koken en veel groter dan smelten?
        \item Teken een T-enthalpie inbreng fasediagram van ijs-water-damp.
        \item Wat bepaalt de fasenregel van gibbs?
        \item Wat is een vrijhijdsgraad?
        \item Wat bedoeld men met het aantal onafhankelijke chemische componenten?
        \item De toestand in een chemisch systeem is in functie van?
        \item Wat is het aantal onafhankelijke chemische componenten bij heterogeen mengsel van \ce{H2O} en \ce{Hg}?
        \item Wat is het aantal onafhankelijke chemische componenten bij homogeen mengsel van \ce{H2O},\ce{NaCl} en ethanol?
        \item Wat is het aantal onafhankelijke chemische componenten bij \ce{N2 + 3H2O <=> 2NH3}?
        \item Geef de formule voor de faseregel van Gibbs en leg alle parameters uit.
        \item Wat is de vrijhijdsgraad voor zuiver kokend water?
        \item Wat is de vrijhijdsgraad voor kokend homogeen mensel van \ce{H20} en \ce{ethanol}?
        \item Wat is de vrijhijdsgraad voor smeltend hetoreen mengsel van \ce{Na} , \ce{K} en homogene smelt?
        \item Wat is dampdruk? Geef de notatie voor dampdruk.
        \item Hoe wordt vluchtigheid gedefini\"eerd?
        \item Wat is verzadigde damp?
        \item De snelheid van verdamping is ... ? Maak tekening.
        \item De snelheid van condensatie is ... ? Maak tekening.
        \item Wanneer is de snelheid van verdamping gelijk aan de snelheidd van condensatie?
        \item Waarom neemt de dampdruk toe na verhoging van de temperatuur? Maak tekening.
        \item Geef de verschillen tussen damp en koken op thermodynamisch niveau (2 zaken).
        \item Hoe heet de functie die dampdruk uitdrukt als functie van de temperatuur? Wat is hier opmerkelijk aan?
        \item Wat is het tripelpunt? Maak tekening. Wat is hier opmerkelijk aan?
        \item Wat is het kritisch punt? Maak tekening. Wat is hier opmerkelijk aan?
        \item Wat is subkritische vloeistof? Maak tekening. Wat is hier opmerkijk aan? Wat is een toepassing van dit spul, leg uit.
        \item Wat is de wet van Clapeyron-Clausius? Wat zijn de beperkingen hierop? 
        \item Hoe kan men experimenteel de kook enthalpie vinden?
        \item Waarom is de wet van Clapeyron-Clausius een semi-logaritmische grafiek?
        \item Leg uit: Wet van Trouton.
        \item Teken een fasediagram van zuivere stoffen. Waarom is \ce{H2O} een uitzondering?
        \item Teken een fasediagram van zuivere stoffen met allotopen. Geef een voorbeeld.
        \item Wat is het grote verschil tussen allotropen en gewone enkelvoudige stoffen m.b.t. het fasediagram?
        \item Wat bedoeld men met een meta-stabiele toestand? Geef een voorbeeld.
    \end{enumerate}

    \subsection{Faseovergangen bij homogene mengsels van vloeistoffen}
    \begin{enumerate}
        \item Wat is het verschil tussen ideale en niet-ideale mengsels?
        \item Wat wordt er bedoeld met een limietgeval?
        \item Wat zijn colligatieve eigenschappen? Noem 4 verschillende colligatieve eigenschappen.
        \item Wat is colligatieve molaliteit? Geef de definitie en formule.
        \item Als de molariteit van \ce{NaCl} $0.01M$ is. Wat is de colligatieve molaliteit dan?
        \item Geef en leidt af: Wet van Raoult.
        \item Wat is de voorwaarde omdat men de wet van Raoult mag gebruiken?
        \item Leid formule af voor de dampdruk van zuivere oplossing A door aanwezigheid solutans B.
        \item Geef een wiskunige formule voor het berekenen van de molaliteit van B.
        \item Leg uit aan de hand van de wet van Raoult waarom men zout strooit in de winter?
        \item Leg osmotische druk uit. Geef voorbeelden.
        \item Wat gebeurt er als je zout of zuiver water zou drinken?
        \item Waarom willen we weten hoe kookprocessen werken van niet zuivere mengsels? Geef een voorbeeld.
        \item Leg uit hoe men aardolie destilleerd.
        \item Leid de formule van de dampdruk van een homogeen ideaal mengsel af + tekening ($p=f(x^{vl}_A)$)
        \item Stel vanuit de wet van Raoult de functie $ p = f(x_A^d) $ op. (homogeen ideaal mengsel)
        \item Wat is het verschil tussen $p=f(x^{vl}_A)$ en $ p = f(x_A^d) $. Teken deze 2 functies. Duid $x_A^d$ en $x^{vl}_A$ aan op de grafiek.
        \item Hoe ga je van een \{P,X\} naar een \{T,X\} diagram?
        \item Wat zijn de karakteristieken van een \{T,X\} diagram?
        \item Teken en leg uit wat er gebeurt bij het koken van een ideaal mengsel van 2 vloeistoffen zonder scheiding.
        \item Teken en leg uit wat er gebeurt bij een gefractioneerde destillatie van een ideaal homogeen mengsel van 2 vloeistoffen. Hoe verloopt de temperatuur bovenaan de destillatiekolom?
        \item Wat bedoelt men met de niet-idealiteit in een homogeen mengsel van 2 vloeistoffen? Wat is het resultaat dan is de \{T,X\} diagrammen dan?
        \item Wat is een azeotropisch mengsel? Geef een tekening. Wat is de vrijheidsgraad hiervan?
        \item Teken en leg uit wat er gebeurt bij een gefractioneerde destillatie van een azeotropisch mengsel.
    \end{enumerate}

    \subsection{Faseovergangen bij heterogene mengsels}
    \begin{enumerate}
        \item Geef een voorbeeld van beperkte mengbaarheid. (vluchtige en minder vluchtige stoffen)
        \item Geef een voorbeeld van beperkte mengbaarheid met een endotherme vermenging. Leg uit + tekening. (Niet vluchtige stoffen)
        \item Leg uit waarom ,bij een gegeven temperatuur, de fasesamenstelling onafhankelijk is van de mengselsamenstelling (endotherme vermenging, Niet vluchtige stoffen). Maar...
        \item Leid de formules voor $X_A^{F1}$ en $X_A^{F2}$ af. (Niet vluchtige stoffen)
        \item Wat is de hefboomregel? Wat kan men hieruit afleiden? (Niet vluchtige stoffen)
        \item Stel dat $n_A^0 = 2.00mol$ en $n_B^0 = 2.50mol$ gemengd worden. Stel dat bij evenwicht, de fasen F1 en F2 de fasensamenstellingen $X_A^{F1} = 0.220$ en $X_A^{F2} = 0.890$. Wat zijn de molhoeveelheden van component A en B op de twee fasen F1 en F2. (Niet vluchtige stoffen)
        \item Teken het \{T,X\} diagram voor het limietgeval dat 2 vloeistoffen helemaal niet mengbaar zijn. (Niet vluchtige stoffen)
        \item Geef een voorbeeld van beperkte mengbaarheid met een exotherme vermenging. (Niet vluchtige stoffen)
        \item Geef een voorbeeld van een mengsel met gesloten 2 fasengebied. (Niet vluchtige stoffen)
        \item Wat is de formule voor de dampdruk van 2 volledig niet mengbare vloetstoffen? + tekening (Vluchtige stoffen)
        \item Wat is er opmerkelijk aan de kooktemperatuur van een heterogeen mengsel van A en B? (Vluchtige stoffen) Waarom is dit?
        \item Wat is een stoomdestillatie? Wanneer is het relevant? Wat is het nut ervan?
        \item Wat is een eutectium of eutectisch mengsel? Geef een toepassing.
        \item Wat is smelt?
        \item Wat is het eutectisch punt?
        \item Leg uit: smeltexperiment in mengsel van 2 helemaal niet mengbare vaste stoffen
        \item Leg uit: stolexperiment in mengsel van 2 helemaal niet mengbare vaste stoffen
        \item Wanneer ontstaan meerdere eutectia? Teken.
    \end{enumerate}


    \section{De elektronenconfiguratie van atomen}
    \subsection{Inleiding}
    \begin{enumerate}
        \item Waarom kan je debateren dat deze 'statement' niet klopt: Atomen zijn ondeelbare deeltjes in chemische reacties.
        \item Uit wat bestaat materie zelden in standaardomstandigheden? Zijn er uitzonderingen? Hoe kunnen we deze staat krijgen?
        \item Wat zijn vrije atomen?
        \item Wat zijn moleculen?
        \item Uit welke deeltjes bestaan polymeren, metalen en keramische materialen?
    \end{enumerate}

    \subsection{Ionisatie energie van atomen}
    \begin{enumerate}
        \item Wat is de ionisatie-energie van een atoom? Aan wat is dit gelijk in absolute waarden?
        \item Je hebt eerste/tweede/... ionisatie-energie. Hoe ver kan je gaan?
        \item Waarom is de tweede ionisatie-energie kleiner dan de eerste ionisatie-energie?
        \item Hoe verlopen de ionisatie-energieën in functie van het periodiek systeem?
        \item Bewijs dat elektronenconfiguratie bestaat in lagen (schillen). Doe dit met Natrium.
        \item Wat zijn de besluiten genomen in verband met de elektronenconfiguratie?
    \end{enumerate}

    \subsection{Interactie elektromagnetische golf-materie}
    \begin{enumerate}
        \item Wat bevestigt de studie van de interactie van elektromagnetische straling? Leg uit.
        \item Wat bedoelt men met 'FINGERPRINT' van een atoom?
        \item Wat is een golf?
        \item Wat is een elektromagnetische golf?
        \item Wat is elektromagnetische straling?
        \item Wat is het gemeenschappelijk kenmerk van de elektromagnetische golven? Maak een tekening.
        \item Wat zijn de belangrijke parameters van elektromagnetische golven?
        \item Geef de 'brede waaier' van elektromagnetische golven. Hoe verloopt de energie/frequentie/golflengte in deze waaier?
        \item Wat is de golflengte van zichtbaar licht?
        \item Waarom is de energie recht evenredig aan de frequentie bij elektromagnetische golven?
        \item Wat bedoelt men met witspectrumlicht?
        \item Hoe creëert men licht van geatomiseerd gas in een ontladingsbuis? Waarom is dit geëmiteerd licht niet continue? Hoe noemt men dit spectrum?
        \item Wat is atomenemissie? Geef een tekening. Geef voorbeelden.
        \item Wat is de emperische gelijkheid van Balmer?
        \item Leg uit: Atomenemissie spectrum is het element van de 'FINGERPRINT'.
        \item Leg uit wat het absorptiespectrum is en het verschil met het emissiespectrum. Wat leert men hier uit?
        \item Wat zijn de uiteindelijke conclusies van de energieverdeling in de materie? Welke ontdekkingen complementeren deze energieverdeling?
        \item Geef de stelling van Planck.
        \item Waarop geeft de stelling van Planck antwoord?
        \item Wat zijn kwanta?
        \item Wat is kwantum?
        \item Wat is de constante van Planck?
        \item Hoe kan men met submicroscopische deeltjes en macroscopische voorwerpen een verband leggen in de klassieke mechanica?
    \end{enumerate}

    \subsection{Dualiteit deelte-golfkarakter van een elektron in de materie}
    \begin{enumerate}
        \item Wat kun je vertellen over de interferentie franjes van Young?
        \item Wat zijn X-stralen en geef een toepassing?
        \item Wat is een diffractiepatroon?
        \item Waarom is het diffractiepatroon van aluminiumfolie zo speciaal? (voorbeeld)
        \item Wat is het uiteindelijk besluit uit de interferentiefranjes van Young? Is hier een belangrijke voorwaarde aan gehecht? (hint: ja)
        \item Wat is een toepassing van het dualiteitprincipe?
        \item Wat zegt de wet van De Broglie?
        \item Wat zijn de voorwaarden van een De Broglie-golf?
        \item Wat is de relevantie bij submicroscopische deeltjes met de wet van De Broglie?
        \item Wat is de niet-relevantie bij macroscopische deeltjes met de wet van De Broglie?
        \item Leg de planetaire visie en zijn beperkingen uit.
        \item Geef de formule voor de bindingsenergie in het Bohr-model. Wat leer je hier uit? Waarom is hier de absorbtie- en emissiespectra perfect te verklaren?
        \item Wanneer is een energieovergang mogelijk?
        \item Wat is de stralingsfrequentie?
        \item Bereken $\lambda_{4 \rightarrow 2}$. 
        \item Bewijs de emperische relatie van Balmer.
        \item Geef de 4 lijnenreeksen. Wat leer je hieruit?
        \item Wat is de laagste energietoestand?
        \item Bewijs dat de ionisatie-energie de positieve laagste energietoestand is.
        \item Wat zijn de tekortkomingen van het Bohr-model?
    \end{enumerate}

    \subsection{Atoomorbitalen en elektronenconfiguratie in meerelektrone atomen}
    \begin{enumerate}
        \item Leg uit: onzekerheidsprincipe van Heisenberg.
        \item Wat is de wiskundige formuleren van het onzekerheidsprincipe van Heisenberg. Wat leer je hier uit?
        \item Pas het onzekerheidsprincipe van Heisenberg toe op een elektron ($x=1 A  , \Delta x = 0.3 A$). Wat leer je hier uit?
        \item Pas het onzekerheidsprincipe van Heisenberg toe op een golfbal ($m=45.9g , v=200 \frac{m}{s}, \Delta x = 1mm$). Wat leer je hier uit?
        \item Hoe worden orbitalen wiskundig beschreven?
        \item Geef en leg de vergelijkingen van Schrödinger uit.
        \item Geef alle kwantumgetallen, hun betekenis en hun mogelijke theoretische waarden.
        \item Leg het experiment uit dat tot de ontdekking van het 4de kwantumgetal heeft geleid. Waarom is dit kwantumgetal het buitenbeetje van de kwantumgetallen? Waarom is deze dan ingevoerd?
        \item Geef alle kwantumgetal combinaties tot $n=4$.
        \item Wat zijn de 2 soorten grafische voorstellingen van de orbitalen? Leg ze ook uit.
        \item Bereken de onzekerheid op de impuls $\Delta p$ van een golfbal met massa $m = 4.59e-2$ en
        snelheid $v = 55.56 m/s$ waarvan we de positie tot op 1mm nauwkeurig kennen en trek conclusies uit de gevonde waarde.
        \item Leg het onzekerheidsprincipe van Heisenberg uit + formule.
        \item Geef alle kwantumgetallen, hun betekenis en hun mogelijke theoretische waarden.
        \item Leg het experiment uit dat tot de ontdekking van het 4de kwantumgetal heeft geleid.
        \item Wat zijn knooppunten in orbitalen (schets een grafiek).
        \item Beschrijf alle orbitalen.
        \item Wat is het uitsluitingsprincipe van Pauli?
        \item Wat is de regel van Hund? Zijn er uitzonderingen?
        \item Waarom volgt Chroom (Cr) niet de theoretische verdeling van elektronen?
        \item Wat is de definitie van valentie?
        \item Geef de 2 mogelijke definities voor de straal van een atoom.
        \item Leg het verband tussen de covalente straal, het periodiek systeem en de ionisatie-energie uit.
    \end{enumerate}


    \section{Ruimtelijke- en elektronenstructuur van moleculen}

    \begin{enumerate}
        \item Wat is de VSEPR-theorie?
        \item Wat is een elektron domein?
        \item Geef 4 verschillende voorstellingen van ammoniak nodig om de structuur te begrijpen.
        \item Waarom heb je verschillende hoeken bij \ce{CH4}, \ce{NH3}, \ce{H2O}?
        \item Wanneer is een binding covalent, polair covalent of ionair?
        \item Wat is een dipoolmoment en bindingsmoment in de chemie?
        \item Zet 1 D(ebye) om naar SI-eenheden.
        \item Wat is het dipoolmoment van \ce{CO2}?
        \item Geef de definitie van intermoleculaire krachgten.
        \item Beschrijf de dipool-dipool interactie en zijn effect op het kookpunt.
        \item Beschrijf de ge\"induceerde krachten of gedisperceerde London krachten.
        \item Geef 3 parameters die de sterkte van de intermoleculaire krachten bij apolaire stoffen bepalen.
        \item Beschrijf waterstofbruggen.
        \item Geef de definitie van oppervlaktespanning.
        \item Geef de definitie van capillariteit.
    \end{enumerate}

    \section{Moleculaire orbitalen en structuren}

    \begin{enumerate}
        \item Wat zijn de limieten van de VSEPR-theorie?
        \item Wat is overlap in de kwantum chemie?
        \item Wat is een bindend elektronenpaar?
        \item Waarom onstaat er steeds bij atoomoverlap een moleculaire orbitaal?
        \item Beschrijf de gevormde orbitalen bij de vorming van \ce{H2} uit 2 \ce{H}-atomen.
        \item Wat is hybridisatie van atoomorbitalen en geef de wiskundige betekenis.
        \item Beschrijf orbitalair het ganse structuurvormingsproces van \ce{BeH2}.
        \item Beschrijf orbitalair het ganse structuurvormingsproces van \ce{BF3}.
        \item Beschrijf orbitalair het ganse structuurvormingsproces van \ce{CH4}.
        \item Beschrijf orbitalair het ganse structuurvormingsproces van \ce{PF5}. 
        \item Beschrijf orbitalair het ganse structuurvormingsproces van \ce{SF6}.
        \item Wat is de voorwaarde omdat men hybridisatie kan toepassen?
        \item Wat is een $\sigma$-binding?
        \item Wat is een $\sigma$-bindings geraamte?
        \item Wanneer kun je afwijkingen hebben bij de hybrides en hoe zou je dit kunnen verklaren?
        \item Wat zijn $\pi$-orbitalen?
        \item Beschrijf orbitalair het ganse structuurvormingsproces van \ce{CH2} (2 $C$-atomen en 4 $H$-atomen).
        \item Kunnen $\pi$-bindingen draaien en zo ja, in welke omstandigheden?
        \item Geef alle mogelijke combinaties orbitalen die samen een  $\pi$-binding kunnen vormen.
        \item Beschrijf  $\pi$-elektronendelokalisatie.
        \item Als n atoomorbitalen overlappen, hoeveel molecuuloribtalen worden er dan gevormd?
        \item Wat zijn K\'ekul\'estructuren?
        \item Beschrijf met de molecuul orbitalen theorie de structuur van een benzeen-ring.
        \item Geef een nodige voorwaarde opdat een molecule aromatisch kan genoemd worden.
        \item De stof \ce{NO3-} heeft een gedelokaliseerd $\pi$-elektronenpaar. Welke eigenschappen zou je uit hieruit kunnen afleiden? 
        \item Geef een definitie van een elektrofiel.
        \item Geef een definitie van een nucleofiel.
        \item Wat is een elektrodefici\"ent atoom?
        \item Bespreek de reactie tussen een elektrofiel en een nucleofiel op thermodynamisch niveau.
        \item Wat is een \textbf{heterolytische reactie}?
        \item Wat is een \textbf{homolytische reactie}?
        \item Wat is een radicaal?
        \item In welk soort milieu worden heterolytische en homolytische reacties bevorderd?
        \item Wat wordt er bedoelt met inductieve effecten?
        \item Tot hoe ver hebben de meeste inductieve effecten hun effect?
        \item Geef een gevolg van mesomere effecten.
    \end{enumerate}
    
    \section{De chemische binding}

    \begin{enumerate}
        \item Waarom gaan atomen bindingen aan?
        \item Bespreek alle mogelijke bindingen.
        \item Wat is een ionaire stof?
        \item Geef de definitie van stabiele ionen.
        \item Wat is elekto affiniteit (EA)?
        \item Wat is ionisatie-energie?
        \item Beschrijf de Born-Habercyclus voor keukenzout.
        \item Teken de Lewisstructuur van \ce{OF2} en \ce{NH4}.
        \item Geef de definitie van een formele lading.
        \item Wat is de formele lading van \ce{NH4+} en \ce{ClO3-}?
        \item Wat is resonantie?
        \item Teken de correcte Lewisstructuur van \ce{CO3^2-}
        \item Wat is een radicaal?
        \item Schrijf de correcte Lewisstructuur van \ce{NO2}
        \item Schrijf de correcte Lewisstructuur van \ce{BeH2}
        \item Schrijf de correcte Lewisstructuur van \ce{BF3}
        \item Wat is hypervalentie?
        \item Schrijf de correcte Lewisstructuur van \ce{SF6}
        \item Schrijf de correcte Lewisstructuur van \ce{SO3}
    \end{enumerate}

    \section{Organische chemie: structuren en naamgeving}
    \subsection*{Inleiding}
    \begin{enumerate}
        \item Defineer: Organische chemie.
        \item Welke verbindingen kan een C-atoom aangaan? Wat is een direct vervolg hiervan? 
        \item Wanneer wordt het C-atoom \textbf{verzadigd} of \textbf{onverzadigd} genoemd?
        \item Defineer: koolwaterstoffen. Uit wat worden deze ontgonnen? Wat wordt er afgezondert bij een destilatie van dit product.
        \item Defineer: verzadigde koolwaterstoffen.
    \end{enumerate}

    \subsection*{Koolwaterstoffen}
    \begin{enumerate}
        \item Defineer: Alkanen. 
        \item Geef de namen van de lineaire alkanen (1-10). 
        \item Geef de verschillende 'old school' grafische voorstellingen van \ce{C4H10}. Waarom zijn deze onpraktisch?
        \item Defineer: conformationele flexibiliteit. 
        \item Teken en benoem de mogelijke vertakte isomeren van butaan en pentaan. 
        \item Wanneer daalt het kookpunt/smeltpunt bij organische moleculen? 
        \item Wat is de algemene molecuulformule van een cyclische alkaan (1 ringstructuur)? Waarom is deze anders?
        \item Defineer: Onverzadiging van een cyclische alkaan. Wat moet men doen per ingevoerde onverzadiging?
        \item Geef alle mogelijke bindingspatronen van een verzadigd hexaan isomeer dat de formule \ce{C6H8} heeft. Hoeveel onverzadigingen heeft dit isomeer?
        \item Wat zijn de voorwaarden om het voorvoegsel 'cyclo-' te gebruiken? Teken cyclopropaan /cyclobutaan /cyclopentaan /cyclohexaan.
        \item Hoe bekomt men een substituenten? Hoe stelt men de KWS voor met een substituent.
        \item Hoe wordt de basisnaam van een substituent bepaald?
        \item Teken en benoem de substituenten van:
                \begin{enumerate}
                    \item methaan / ethaan
                    \item propaan / cyclopropaan
                    \item butaan / isobutaan / cyclobutaan
                \end{enumerate}
        \item Wat zijn de voorwaarden om het voorvoegsel 'sec-' en 'tert-' te gebruiken?
        \item Defineer: primair en quaternair C-atoom. 
        \item Waarom moet de hoofdketen genummerd worden?
        \item Wanneer gebruikt men multiplicerende voorvoegsels?
        \item Hoe vernoem je radicalen afgeleid uit alkanen?
        \item Hoe worden meerdere gelijke samengestelde substituenten aangeduid? 
        \item Hoe duid men het aantal dubbele koolstof bindingen aan in de hoofdketen van een alkaan?
        \item Hoe duid men het aantal meervoudige bindingen aan in de hoofdketen van een alkaan? 
        \item Defineer: Aromatische koolwaterstoffen. 
        \item Defineer: Digesubstitueerde benzeenringen. 
        \item Wat zijn de voorwaarden om de voorvoegsels 'o-' / 'm-' / 'p-' te gebruiken?
        \item Teken de organische structuren: (oplossingen staan in volgorde in het boek)
            \begin{itemize}
                \item 3-methylpentaan
                \item methylcyclohexaan
                \item 2,3,3,4-tetramethylpentaan
                \item 5,5-di-t-butylnonaan
                \item 2-methylpentaan
                \item 2,3,5-trimethylhexaan
                \item 2,7,8-trimethyldecaan
                \item 4-isopropyl-5-porpyloctaan
                \item 3-ethyl-2-methylheptaan
                \item 5-methyl-4-propylnonaan
                \item 1-methylpentyl
                \item 2-methylpentyl
                \item 3-methylpentyl
                \item 4-methylpentyl
                \item 4,4-diethyloctaan
                \item 5,6-bis(1,1-dimethylpropyl)decaan
                \item 2-hexeen
                \item 1,4-hexadieen
                \item alleen 
                \item isopreen
                \item cyclohexeen
                \item vinyl
                \item allyl
                \item butenyn
                \item acetyleen
                \item ethynyl
                \item 3-penteen-1-yn                                                                                                                                                   
                \item 1,4-cyclohexadieen
                \item cyclooctaterta"een
                \item 1,3-hexadieen-5-yn
                \item 1-penteen-4-yn
                \item benzeen
                \item naftaleen
                \item anthraceen
                \item fenantreen
                \item o-benzeen / m-benzeen / p-benzeen
                \item 1-ethyl-4-methylbenzeen
                \item 1,2,3-trimethylbenzeen
                \item styreen
                \item tolueen
                \item o-xyleen
                \item benzyl
                \item fenyl
                \item o-tolyl
            \end{itemize}
        \subsection*{Organische functies of functionele groepen}
        \begin{enumerate}
            \item Defineer: Organische functie, Hetero-atomen, Mono-functionele verbindingen
            \item Geef de 2 soorten substituenten. Leg ze ook uit.
            \item Wat zijn carbonylgroep en carboxylgroep?
            \item Waarom is Nitril een carbonzuur derivaat?
            \item Geef de formule en alle mogelijken aarden van de substituenten van de volgende functionele groepen:
                \begin{itemize}
                    \item alifatisch halogenide
                    \item aromatisch halogenide
                    \item alcohol
                    \item enol
                    \item fenol
                    \item ether
                    \item primaire / secundaire / tertiaire amine 
                    \item nitroverbinding
                    \item aldehyde
                    \item keton 
                    \item carbonzuur
                    \item ester
                    \item carbonzuur anhydride
                    \item carbonzuur halogenide
                    \item amide en primaire / secundaire / tertiaire amide
                    \item nitril
                \end{itemize}
            \item Geef de 2 types nomenclatuur. Leg ze ook uit.
            \item Leg uit: substitutieve naamgeving van organische halogeniden
            \item Leg uit: radicofunctionele naamgeving van organische halogeniden
            \item Leg uit: substitutieve naamgeving van alcoholen + speciaal geval
            \item Geef de voorangregels van de substituenten
            \item Leg uit: radicofunctionele naamgeving van alcoholen
            \item Leg uit: naamgeving van fenolen
            \item Leg uit: substitutieve naamgeving van ethers
            \item Leg uit: radicofunctionele naamgeving van ethers
            \item Leg uit: radicofunctionele naamgeving van primaire aminen 
            \item Leg uit: substitutieve naamgeving van primaire aminen en polyaminen + speciaal geval
            \item Leg uit: naamgeving van secundaire en tertiaire aminen 
            \item Leg uit: naamgeving van nitroderivaten
            \item Leg uit: naamgeving van aldehyden + speciaal geval
            \item Leg uit: naamgeving van ketonen
            \item Wanneer mag het voorvoegsel 'oxo' gebruikt worden?
            \item Leg uit: naamgeving van carbonzuren en carbonzuurderivaten + speciaal geval
            \item Leg uit: naamgeving van esters
            \item Leg uit: naamgeving van N-gesubstitueerde amiden
            \item Teken de organische structuren: (oplossingen staan in volgorde in het boek)
                \begin{itemize}
                    \item 2-chloorbutaan
                    \item p-broomchloorbenzeen
                    \item 1,2-dibroommethaan
                    \item 1,4-dijoodcyclohexaan
                    \item 1,1-dichloorethaan
                    \item 1,2-dichloorethaan
                    \item tetrachloormethaan
                    \item methylchloride
                    \item benzyljodide
                    \item tert-butylbromide
                    \item chloroform
                    \item koolstoftetrachloride
                    \item 2-propanol
                    \item 1,4-butaandiol
                    \item 2-propyl-2-penteen-1-ol
                    \item 2-cyclohexeen-1-ol
                    \item 2-hydroxymethyl-1,5-pentaandiol
                    \item ethylalcohol
                    \item tert-butylalcohol
                    \item allylalcohol
                    \item benzylalcohol
                    \item 4-methyl-1,2-benzeendiol
                    \item 1,2,4-benzeentriol
                    \item fenol
                    \item 2-naftol
                    \item ethoxyetheen
                    \item 1-broom-2-propoxyethaan
                    \item 1,2,4-trimethoxybenzeen
                    \item 1-isopropoxybutaan
                    \item ethylmethylether
                    \item diethylether
                    \item bis(1-chloorethyl)ether
                    \item isopropylamine
                    \item cyclohexylamine
                    \item 2-naftylamine
                    \item aniline
                    \item 2-propaanamine / 2-aminopropaan
                    \item 1,3-benzeendiamine / 1,3-diaminobenzeen
                    \item 2-aminomethyl-1,5-pentaandiamine
                    \item 1-(N,N-dimethylamino)propaan
                    \item N-methylaminocyclopentaan
                    \item N-methyl-N-propylaniline
                    \item 2-(N-methylamino)naftaleen
                    \item 3-(N-ethyl-N-methylamino)hexaan
                    \item nitromethaan
                    \item 1,3,5-trinitrobenzeen
                    \item ethanal
                    \item hexaandial
                    \item 5-allyl-2-hexeendial
                    \item formaldehyde
                    \item acetaldehyde
                    \item propionaldehyde
                    \item butyraldehyde
                    \item benzaldehyde
                    \item formylcyclohexaan
                    \item 3-formylpentaandial
                    \item 2-pentanon
                    \item 2,4-pentaandion
                    \item 5-hexeen-2-onpraktisch3-allyl-2,4-pentaandion
                    \item fenyl-1-butanon
                    \item cyclopentanon
                    \item cyclohexyl-1-propanol
                    \item aceton
                    \item acetofenon
                    \item benzofenon
                    \item 3-methyl-1-pentaancarbonzuur
                    \item 2-butaancarbonzuur
                    \item 1,2-cyclohexaandicarbonzuur
                    \item 3-pyrideinecarbonzuur
                    \item 2-methyl-2-butaancarboxylaat
                    \item 3-hydroxy-1-propaancarbonylchloride
                    \item 3-amino-cyclopentaancarbonamide
                    \item 3,5-dioxo-1-pentaancarbonzuur
                    \item cyclohexaancarbonitril
                    \item ethyl 2-methyl-2-butaancarboxylaat
                    \item n-propyl 4-aminocyclohexaancarboxylaat
                    \item ethyl acetaat
                    \item isopropyl formiaat
                    \item N,N-dimethylbutyramide
                    \item N-ethyl-N-methyl-3-hydroxy-1cyclopentaancarbonamide
                    \item N-methylacetamide
                    \item 2-carboxymethyl-1,5-pentaandicarbonzuur
                    \item 3-acetyloxypropionzuur
                    \item 4-cyaanboterzuur
                \end{itemize}
        \end{enumerate}
        \subsection*{Isomerie en isomeren in de organische chemie}
        \begin{enumerate}
            \item Defineer: isomeren
            \item Wat bepaald het type isomerie?
            \item Defineer: Constitutie van een moleculen / Constitutionele isomeren.
            \item Geef alle mogelijke isomeren voor alle verbindingen met molecuulformule \ce{C2H4O} / \ce{C5H12} / \ce{C4H9Cl}
            \item Defineer: stereoisomeren. 
            \item Wat zijn de verschillenden soorten stereoisomeren? 
            \item Wat zijn cis/trans isomeren? Uit wat is dit een gevolg?
            \item Defineer: Enantiomeren. geef voorbeelden. geef toepassing.
            \item Waaraan zijn enantiomerie te wijten?
            \item Defineer: Chiraliteit. 
            \item Hoe kan je grafisch een enantiomeer onderscheiden?
            \item Defineer: optische activiteit. Waarom hebben Enantiomeren andere optische activiteit?
            \item Defineer: Achirale molecule / Racemisch mengsel
            \item Defineer: Diastereomeren. geef voorbeelden.
            \item Defineer: configuratie / conformatie
            \item Leg de isometrie in dichloorcyclopentaan uit.
            \item Wat zijn de algemene regels voor reactiviteit met enantiomeren?
            \item Hoe gaat men enantiomeren scheiden?
            \item Conformeer ethaan als een Newmanprojectie.
        \end{enumerate}
    \end{enumerate}
    \subsection*{Koolwaterstoffen - eigennamen}
    \begin{enumerate}
        \item isobutaan $$\chemfig[angle increment=30]{-[1](-[-1])-[3]}$$
        \item isopentaan $$\chemfig[angle increment=30]{-[1](-[-1]-[1])-[3]}$$
        \item neopentaan $$\chemfig[]{-[:30](-[:-30])(<[:60])<:[:120]}$$
        \item alleen $$\chemfig{\ce{H2C}=C=\ce{CH2}}$$
        \item isopreen $$\chemfig[angle increment=30]{\ce{H2C}=C(-[1]\ce{CH3})-[-1]CH=\ce{CH2}}$$
        \item cyclohexeen $$\chemfig[]{*6(--=---)}$$
        \item vinyl $$\chemfig[]{\ce{H2C}=CH\bigcdot{}}$$
        \item allyl $$\chemfig[angle increment=30]{\ce{H2C}=CH-[1]\chemabove{\ce{CH2}}{\bigcdot}}$$
        \item acetyleen $$\chemfig[]{HC~CH}$$
        \item ethynyl $$\chemfig[]{HC~C\bigcdot}$$
        \item benzeen $$\chemfig[]{*6(-=-=-=)}$$
        \item nafteleen $$\chemfig{*6(=-*6(-=-=-)=-=-)}$$
        \item arthraceen $$\chemfig{*6(=-*6(=-*6(-=-=--)=-=)--=-)}$$
        \item fenantreen $$\chemfig[]{[:-30]*6(-=-*6(-*6(-=-=--)=-=--)=-=)}$$
        \item styreen $$\chemfig[]{*6(-=-(-=[:-30])=-=)}$$
        \item tolueen $$\chemfig[]{*6(-=-(-)=-=)}$$
        \item o-xyleen $$\chemfig[]{*6(-=-(-)=(-)-=)}$$
        \item benzyl $$\chemfig[]{*6(-=-(-\chemabove{\ce{CH2}}{\bigcdot{}})=-=)}$$
        \item fenyl $$\chemfig[]{*6(-=-(\chemabove{}{\bigcdot{}})=-=)}$$
        \item o-tolyl $$\chemfig[]{*6(-=-(-\chemabove{\ce{CH3}}{})=(\chemabove{}{\bigcdot{}})-=)}$$
    \end{enumerate}

    \subsection{Organische functies of functionele groepen - eigennamen}
    \begin{center}
        {\small 
        \begin{tabular}{ c c c c c c } 
         \hline
         $\chemfig[]{\ce{R}-}$ & $\chemfig[]{\ce{R}-COOH}$ & $\chemfig[]{\ce{R}-COO^{(-)}}$ & $\chemfig[]{\ce{R}-COX}$ & $\chemfig[]{\ce{R}-\ce{CONH2}}$ & $\chemfig[]{\ce{R}-}$ \\ 
         $\chemfig[]{\ce{H}-}$ & mierenzuur & formiaat & formyl halogenide & formamide & formonitril \\ 
         $\chemfig[]{\ce{CH3}-}$ & azijnzuur & acetaat & acetyl halogenide & acetamide & acetonitril \\ 
         $\chemfig[]{\ce{CH3CH2}-}$ & propionzuur & propionaat & propionyl halogenide & propionamide & propionitril\\
         $\chemfig[]{\ce{CH3CH2CH2}-}$ & boterzuur & butyraat & butyryl halogenide & butyramide & butyronitril \\
         $\chemfig[]{\ce{C6H5}-}$ & benzoëzuur & benzoaat & benzonyl halogenide & benzamide & benzonitril \\
         \hline
        \end{tabular}
        }
        \end{center}
    \begin{enumerate}
        \item chloroform $$\chemfig[]{Cl-[:30]C(-[:90]H)(<[:-55]Cl)<:[:-20]Cl}$$
        \item fenol $$\chemfig[]{*6(-=-(-\chemabove{\ce{OH}}{})=-=)}$$
        \item 2-naftol $$\chemfig{*6(=-*6(-=-(-[:30]OH)=-)=-=-)}$$
        \item aniline $$\chemfig[]{*6(-=-(-\chemabove{\ce{NH2}}{})=-=)}$$
        \item formaldehyde (in aq. formol) $$\chemfig[]{\ce{H2C}=O}$$
        \item acetaldehyde $$\chemfig[]{\ce{H3C}-CH=O}$$
        \item propionaldehyde $$\chemfig[]{\ce{CH3}-\ce{CH2}-CH=O}$$
        \item butyraldehyde $$\chemfig[]{\ce{CH3}-\ce{CH2}-\ce{CH2}-\ce{CH}=O}$$
        \item benzaldehyde $$\chemfig[]{*6(-=-(-\ce{C}(=[:90]O)-[:-30]H)=-=)}$$
        \item aceton $$\chemfig[]{\ce{CH3}-CO-\ce{CH3}}$$
        \item acetofenon $$\chemfig[]{*6(-=-(-\ce{C}(=[:90]O)-[:-30]\ce{CH3})=-=)}$$
        \item bezofenon $$\chemfig[]{*6(-=-(-\ce{C}(=[:90]O)-[:-30]*6(=-=-=-))=-=)}$$
    \end{enumerate}

    \section{Organische chemie: Reactiviteit}
    \subsection*{Substitutiereacties op alifatische, verzadigde C-atomen}
    \begin{enumerate}
        \item Geef en teken de algemene Substitutieregels. Wat zijn de 2 soorten reacties?
        \item Geef en teken de Nucleofiele / Elektrofiele substitutiereactie. Wat is de voorwaarde dat deze zou doorgaan?
        \item Wat maakt een goed nuceofiel uittredende structuureenheid? Geef voorbeelden van goede en slechte.
        \item Waarom is de bereiding alkylchloride uit alcohol met \ce{NaCl} onmogelijk? Wat is de oplossing. Pas deze oplossing toe op het bereiden van alkylchloride.
        \item Defineer: Elektrofiele preactivering.
        \item Defineer: $S_N1$-reactie. Geef ook de gedachtegang. Geef de omzetting van t-butylchloride in t-butyljodide. Wat is de snelheidsvergelijking van deze omzetting?
        \item Wanneer gebeuren $S_N1$-reacties? Leg alle types uit.
        \item Waarom gebeuren $S_N1$-reacties steeds met racemisatie? Leg uit en teken.
        \item Defineer: $S_N2$-reactie. Geef ook de gedachtegang. Geef de omzetting van methyljodide tot methylalcohol.
        \item Defineer: Walden inversie.
        \item Wanneer worden reacties stereospecifiek genoemd? Geef de types. Wat bewijst dit? Met wat treden $S_N2$-reacties op? Waarom?
        \item Geef toepassingen en voorbeelden van nucleofiele reacties.
        \item \textbf{De 2 bij $S_N2$-reacties slaat op? Bij een $S_N1$-reactie gebeurt steeds of nooit racemisatie?} (PPT-VRAAG)
        \item Defineer: Organometalen. Geef de 2 belangrijkste organometalen. Hoe worden deze meestal gesynthetiseerd? 
        \item Defineer: Grignard reagentia. 
        \item Hoe worden andere organometalen (tetraalkyltin-,tertaalkyllood-,dialkylkwikderivaten) gesynthetiseerd?
        \item Wat is de voorwaarde om organometalen te synthetiseren? Waarom?
        \item Defineer: $S_E1$-reacties. Waaraan zijn deze te wijten? Geef en leg uit de algemene methode. Pas toe op:
        \begin{enumerate}
            \item methyllithium met een waterproton.
            \item $R-MgCl$ + $H-Cl$ 
            \item $R-BeR$ + $H-NEt_2$
            \item $R-ZnR$ + $H-NH_2$
            \item $\ce{CH3}-\ce{G(CH3)_2}$ + $I-I$
        \end{enumerate}
        \item Waarom vinden $S_E1$-reacties plaats met racemisatie. Geef voorbeelden.
        \item Defineer: $S_E1$-reacties.  Pas toe op:
        \begin{enumerate}
            \item methyllithium met een waterproton.
            \item $CH_3-Pb(CH_3)_3$ + $Ag^+NO_3^-$ 
            \item $CH_3CH_2-Hg-I$ + $I-I$
            \item $CH_3-Sn(CH_3)_3$ + $H-Br$
        \end{enumerate}
    \end{enumerate}
    \subsection*{Elektrofiele addities op dubbele $C=C$ bindingen}
    \begin{enumerate}
        \item Defineer de elektrofiele addities van dihalogenen op de dubbele $C=C$ binding.  Wat gebeurt er als het alkeen niet symmetrisch gesubstitueerd is? Voor wat is dit nuttig?
        \item Bewijs experimenteel het bestaan van trans stereospecifiteit.
        \item Defineer de additie van waterstofhalogeniden op alkenen.
        \item Leg de Markovnikov-regel / Anti-Markovnikov-regel uit.
    \end{enumerate}
    \subsection*{Eliminatie reacties}
    \begin{enumerate}
        \item Defineer E1-eliminaties.
        \item Defineer $\beta-$ / $\alpha-$eliminaties.
        \item Defineer E2-eliminaties. Wat is de voorwaarde voor een E2-eliminatie?
        \item Wat zijn organische oxidaties? Leg de oxidatie van een primair alcohol naar aldehyde uit. Geef schematische voorstellingen van:
        \begin{enumerate}
            \item alcohol $\rightarrow$ aldehyde
            \item aldehyde $\rightarrow$ carbonzuur
            \item alcohol $\rightarrow$ keton
        \end{enumerate}
        \item Kunnen tertiaire alcoholen oxideren? Waarom wel/niet ?
    \end{enumerate}
    \subsection*{Additiereacties op $C=O$ bindingen}
    \begin{enumerate}
        \item Wat is de algemene eigenschap van de carbonylfunctie in aldehyden en ketonen?
        \item Defineer grignardreacties op aldehyden en ketonen.
        \item Defineer grignardreacties op kooldioxide tot carbonzuren.
        \item Wat zijn aldoreacties. Toon aan met aceton.
    \end{enumerate}
    \subsection*{Reacties met carbonzuurderivaten}
    \begin{enumerate}
        \item Wat is de algemene eigenschap van de carbonylfunctie in carbonzuur en carbonzuurderivaten?
        \item Hoe worden esters gesynthetiseerd?
        \item Hoe worden carbonzuur chloriden gesynthetiseerd?
        \item Hoe worden amiden gesynthetiseerd? Wat gebeurt er met de nevenproducten? Wat is hier een gevolg van?
        \item Hoe worden esters gehydrolyseerd? Geef de ook de globale stoichiometrie. Wat zijn de gevolgen?
        \item Defineer: verzeping / zeep / vetzuren / detergenten.
        \item Hoe werkt zeep?
        \item Hoe worden zepen gesynthetiseerd?
    \end{enumerate}
    \subsection*{Aromatische Substitutiereacties}
    \begin{enumerate}
        \item Defineer: Aromaciteit. Wat is een gevolg van Aromaciteit? Wat verklaart dit?
        \item Hoe brommeer je benzeen?
        \item Maak TNT.
        \item Wat is de Friedel-Craft reactie? Leg beide toepassingen uit.
        \item Wat zijn elektrofiele substituties op monogesubstitueerde derivaten? Geef een voorbeelden + uitleg voor alle vormen.
        \item Waarom zijn aromatische nucleofiele substituties op benzeenderivaten niet mogelijk? Geef een uitzondering.
        \item Zet benzeen diazoniumkation om tot fenol met water.
        \item Wanneer gebeurt aromatische nucleofiele substitutie als additie-eliminatie? 
        \item Wanneer gebeurt nucleofilie verlies op de substitutiesite?
        \item Wat gebeurt er na het stabiliseren van een negatieve lading van een intermediair carbonion? Geef een voorbeeld.
        \item Hoe wordt methyloranje gesynthetiseerd?
    \end{enumerate}

    \section{Polymeren}
    \subsection*{Definities}
    \begin{enumerate}
        \item Defineer: Polymeer / mesomeer.
        \item Welke polymeren onderschijden we? Defineer elke subcategorie.
    \end{enumerate}
    \subsection*{Plastics of polymeren?}
    \begin{enumerate}
        \item Wat is het verschil tussen plastic en een polymeer?
    \end{enumerate}
    \subsection*{Classificatie van polymeren}
    \begin{enumerate}
        \item Uit wat resulteren kunsstoffen?
        \item Wat is onderverdeling van polymeren?
        \item Defineer: Morfologie. Wat zijn de verschillende soorten? Defineer de soorten?
        \item Wat zijn de eigenschappen van de verschillende soorten chemische structuren bij polymeren?
        \item Defineer: Oligomeer. Geef een voorbeeld.
        \item Welk concept moeten we invoeren om de thermische eigenschappen van polymeren te bestuderen?
        \item Defineer: Thermoplasten / Thermoharder / Elastomeer / Elastoplast.
        \item Leg gebruiktclassificatie uit. Waarom slecht?
        \item Defineer de 2 grote synthesemethodes voor polymeren.
    \end{enumerate}
    \subsection*{Ketengroei polymeren}
    \begin{enumerate}
        \item Defineer:
        \begin{enumerate}
            \item polyethyleen
            \item polpropyleen
            \item polyvinylchloride (PVC)
            \item polystyreen
            \item poly(fluorethyleen)     Teflon \textregistered
            \item poly(methyl methacrylaat)    Plexiglas \textregistered
            \item poly(acrylnitril)
            \item poly(vinylacetaat)
            \item poly(ethylacrylaat)
        \end{enumerate}
        \item Wat is radicalaire polymerisatie van alkenen? Toon dit aan met de synthese van LDPE. Naar wat streven gesubstitueerde alkenen? Wat gebeurt er met niet-symmetrische alkenen?
        \item Wat is kationische polymerisatie van alkenen? Toon dit aan. Wat gebeurt er bij niet stabiliseerbare intermediaire kationen? Geef een voorbeeld met de synthese van polyisobuteen.
        \item Wat is anionische polymerisatie van alkenen? Toon dit aan theoretisch en met een voorbeeld.
        \item Hoe worden ketenvertakkingen gevormd? Geef de twee manieren. Wat is het verschil tussen deze twee?
        \item Wat is stereochemie? Hoe wordt deze bepaald op een stereogene koolstofatomen? 
        \item Defineer de verschillende soorten stereochemie van polymeren.
        \item Waarom hebben polymeren met verschillende tactiviteit verschillende chemische / fysische eigenschappen?
        \item Waarom zijn atactische polyalkenen ook vertakt?
        \item Waarvoor worden de Ziegler-Natta katalysatoren gebruikt? Wat zijn de voordelen van deze katalysatoren? Waarom zijn deze stijver/sterker. Geef het vereenvoudigd mechanisme.
        \item Wat is een gevolg na polymerisatie van alkadiënen? Teken alle mogelijke dingetjes.
        \item Defineer: Natuurlijke rubbers. Wat is de andere isomerie van deze rubber? Hoe maak je de kunstmatige rubber Neopreen? 
        \item Wat is vulcanisatie van rubber? In welk opzicht verandert het rubber?
    \end{enumerate}
    \subsection*{Copolymeren}
    \begin{enumerate}
        \item Defineer: Copolymeren / Homopolymeer. Geef en teken de soorten copolymeren.
        \item Maak styreen-butadieenrubber. Wat zijn de eigenschappen afzonderlijk en als copolymeer?
        \item Maak verf. Wat zijn de eigenschappen afzonderlijk en als copolymeer?
        \item Wat zijn toepassingen van anker copolymeren?
        \item Geef de bereidingsmethoden voor elke soort copolymeren.
    \end{enumerate}
    \subsection*{Stapgroei polymeren}
    \begin{enumerate}
        \item Defineer: Stapgroei polymeren. Geef de algemene vorsteling. Wat wordt er bedoelt met ideale alternerende copolymeren?
        \item Hoe worden polyamiden gesynthetiseerd? Welke isomerie kunnen deze aannemen? Waarom?
        \item Maak Nylon6.
        \item Defineer: Proteïnen / peptide bindingen. Geef de soorten peptide bindingen. Wat zijn opmerkelijke eigenschappen van natuurlijke proteïnen?
        \item Hoe worden polyesters gesynthetiseerd? Geef de algemene voorstelling. Waarom is deze reactie aflopend?
        \item Hoe worden polycarbonaten gesynthetiseerd? Stel voor door synthese van Lexan \textregistered.
        \item Hoe worden polyurethanen gesynthetiseerd? Hoe wordt urethaan gesynthetiseerd? Waarom is polyurethaan wel/geen condensatiepolymeer? Toon aan.
        \item Geef gedachtegang voor PU gebruikt als schuimrubbers. Geef ook de chemische vergelijking.
    \end{enumerate}
    \subsection*{Covalente netwerkpolymeren}
    \begin{enumerate}
        \item Defineer: Covalente netwerkpolymeren. Maak Bakeliet.
    \end{enumerate}
    \subsection*{Molmassa's van polymeren}
    \begin{enumerate}
        \item Hoe wordt de molmassa van polymeren berekent? Defineer de nieuw ingevoerde eenheid.
    \end{enumerate}
\end{document}
% Everything in this region will be overriden

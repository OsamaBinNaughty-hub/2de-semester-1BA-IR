\documentclass[a4paper,12pt]{article}

\usepackage[dutch]{babel}
\usepackage{amsmath}

\title{Mechanica 2}
\author{Dries Van den Brande \and Andreas Declerck}

\begin{document}

    % Only things after \maketitle are added to master, all other things should be put in preamble
    \maketitle
    
    \section{Aggregatietoestanden en toestandsovergangen. Fasenleer}

    \subsection{Aggregaattoestanden en toestandovergangen bij zuivere stoffen}
    \begin{enumerate}
        \item Wat zijn de 3 voornaamste aggregaattoestanden?
        \item Leg uit: gecondenceerde toestand.
        \item Warmte-inbreng in een zuivere stof resulteerd in?
        \item Faseovergang van een zuivere stof gebeurt bij een constante ... ? 
        \item Faseovergang van een zuivere stof is afhankelijk van?
        \item Wat zijn alle mogelijke toestandsovergangen?
        \item Waarom is de energie nodig om te sublimeren groter dan koken en veel groter dan smelten?
        \item Teken een T-enthalpie inbreng fasediagram van ijs-water-damp.
        \item Wat bepaalt de fasenregel van gibbs?
        \item Wat is een vrijhijdsgraad?
        \item Wat bedoeld men met het aantal onafhankelijke chemische componenten?
        \item De toestand in een chemisch systeem is in functie van?
        \item Wat is het aantal onafhankelijke chemische componenten bij heterogeen mengsel van \ce{H2O} en \ce{Hg}?
        \item Wat is het aantal onafhankelijke chemische componenten bij homogeen mengsel van \ce{H2O},\ce{NaCl} en ethanol?
        \item Wat is het aantal onafhankelijke chemische componenten bij \ce{N2 + 3H2O <=> 2NH3}?
        \item Geef de formule voor de faseregel van Gibbs en leg alle parameters uit.
        \item Wat is de vrijhijdsgraad voor zuiver kokend water?
        \item Wat is de vrijhijdsgraad voor kokend homogeen mensel van \ce{H2O} en \ce{ethanol}?
        \item Wat is de vrijhijdsgraad voor smeltend hetoreen mengsel van \ce{Na} , \ce{K} en homogene smelt?
        \item Wat is dampdruk? Geef de notatie voor dampdruk.
        \item Hoe wordt vluchtigheid gedefini\"eerd?
        \item Wat is verzadigde damp?
        \item De snelheid van verdamping is ... ? Maak tekening.
        \item De snelheid van condensatie is ... ? Maak tekening.
        \item Wanneer is de snelheid van verdamping gelijk aan de snelheidd van condensatie?
        \item Waarom neemt de dampdruk toe na verhoging van de temperatuur? Maak tekening.
        \item Geef de verschillen tussen damp en koken op thermodynamisch niveau (2 zaken).
        \item Hoe heet de functie die dampdruk uitdrukt als functie van de temperatuur? Wat is hier opmerkelijk aan?
        \item Wat is het tripelpunt? Maak tekening. Wat is hier opmerkelijk aan?
        \item Wat is het kritisch punt? Maak tekening. Wat is hier opmerkelijk aan?
        \item Wat is subkritische vloeistof? Maak tekening. Wat is hier opmerkijk aan? Wat is een toepassing van dit spul, leg uit.
        \item Wat is de wet van Clapeyron-Clausius? Wat zijn de beperkingen hierop? 
        \item Hoe kan men experimenteel de kook enthalpie vinden?
        \item Waarom is de wet van Clapeyron-Clausius een semi-logaritmische grafiek?
        \item Leg uit: Wet van Trouton.
        \item Teken een fasediagram van zuivere stoffen. Waarom is \ce{H2O} een uitzondering?
        \item Teken een fasediagram van zuivere stoffen met allotopen. Geef een voorbeeld.
        \item Wat is het grote verschil tussen allotropen en gewone enkelvoudige stoffen m.b.t. het fasediagram?
        \item Wat bedoeld men met een meta-stabiele toestand? Geef een voorbeeld.
    \end{enumerate}

    \subsection{Faseovergangen bij homogene mengsels van vloeistoffen}
    \begin{enumerate}
        \item Wat is het verschil tussen ideale en niet-ideale mengsels?
        \item Wat wordt er bedoeld met een limietgeval?
        \item Wat zijn colligatieve eigenschappen? Noem 4 verschillende colligatieve eigenschappen.
        \item Wat is colligatieve molaliteit? Geef de definitie en formule.
        \item Als de molariteit van \ce{NaCl} $0.01M$ is. Wat is de colligatieve molaliteit dan?
        \item Geef en leidt af: Wet van Raoult.
        \item Wat is de voorwaarde omdat men de wet van Raoult mag gebruiken?
        \item Leid formule af voor de dampdruk van zuivere oplossing A door aanwezigheid solutans B.
        \item Geef een wiskunige formule voor het berekenen van de molaliteit van B.
        \item Leg uit aan de hand van de wet van Raoult waarom men zout strooit in de winter?
        \item Leg osmotische druk uit. Geef voorbeelden.
        \item Wat gebeurt er als je zout of zuiver water zou drinken?
        \item Waarom willen we weten hoe kookprocessen werken van niet zuivere mengsels? Geef een voorbeeld.
        \item Leg uit hoe men aardolie destilleerd.
        \item Leid de formule van de dampdruk van een homogeen ideaal mengsel af + tekening ($p=f(x^{vl}_A)$)
        \item Stel vanuit de wet van Raoult de functie $ p = f(x_A^d) $ op. (homogeen ideaal mengsel)
        \item Wat is het verschil tussen $p=f(x^{vl}_A)$ en $ p = f(x_A^d) $. Teken deze 2 functies. Duid $x_A^d$ en $x^{vl}_A$ aan op de grafiek.
        \item Hoe ga je van een \{P,X\} naar een \{T,X\} diagram?
        \item Wat zijn de karakteristieken van een \{T,X\} diagram?
        \item Teken en leg uit wat er gebeurt bij het koken van een ideaal mengsel van 2 vloeistoffen zonder scheiding.
        \item Teken en leg uit wat er gebeurt bij een gefractioneerde destillatie van een ideaal homogeen mengsel van 2 vloeistoffen. Hoe verloopt de temperatuur bovenaan de destillatiekolom?
        \item Wat bedoelt men met de niet-idealiteit in een homogeen mengsel van 2 vloeistoffen? Wat is het resultaat dan is de \{T,X\} diagrammen dan?
        \item Wat is een azeotropisch mengsel? Geef een tekening. Wat is de vrijheidsgraad hiervan?
        \item Teken en leg uit wat er gebeurt bij een gefractioneerde destillatie van een azeotropisch mengsel.
    \end{enumerate}

    \subsection{Faseovergangen bij heterogene mengsels}
    \begin{enumerate}
        \item Geef een voorbeeld van beperkte mengbaarheid. (vluchtige en minder vluchtige stoffen)
        \item Geef een voorbeeld van beperkte mengbaarheid met een endotherme vermenging. Leg uit + tekening. (Niet vluchtige stoffen)
        \item Leg uit waarom ,bij een gegeven temperatuur, de fasesamenstelling onafhankelijk is van de mengselsamenstelling (endotherme vermenging, Niet vluchtige stoffen). Maar...
        \item Leid de formules voor $X_A^{F1}$ en $X_A^{F2}$ af. (Niet vluchtige stoffen)
        \item Wat is de hefboomregel? Wat kan men hieruit afleiden? (Niet vluchtige stoffen)
        \item Stel dat $n_A^0 = 2.00mol$ en $n_B^0 = 2.50mol$ gemengd worden. Stel dat bij evenwicht, de fasen F1 en F2 de fasensamenstellingen $X_A^{F1} = 0.220$ en $X_A^{F2} = 0.890$. Wat zijn de molhoeveelheden van component A en B op de twee fasen F1 en F2. (Niet vluchtige stoffen)
        \item Teken het \{T,X\} diagram voor het limietgeval dat 2 vloeistoffen helemaal niet mengbaar zijn. (Niet vluchtige stoffen)
        \item Geef een voorbeeld van beperkte mengbaarheid met een exotherme vermenging. (Niet vluchtige stoffen)
        \item Geef een voorbeeld van een mengsel met gesloten 2 fasengebied. (Niet vluchtige stoffen)
        \item Wat is de formule voor de dampdruk van 2 volledig niet mengbare vloetstoffen? + tekening (Vluchtige stoffen)
        \item Wat is er opmerkelijk aan de kooktemperatuur van een heterogeen mengsel van A en B? (Vluchtige stoffen) Waarom is dit?
        \item Wat is een stoomdestillatie? Wanneer is het relevant? Wat is het nut ervan?
        \item Wat is een eutectium of eutectisch mengsel? Geef een toepassing.
        \item Wat is smelt?
        \item Wat is het eutectisch punt?
        \item Leg uit: smeltexperiment in mengsel van 2 helemaal niet mengbare vaste stoffen
        \item Leg uit: stolexperiment in mengsel van 2 helemaal niet mengbare vaste stoffen
        \item Wanneer ontstaan meerdere eutectia? Teken.
    \end{enumerate}

\end{document}
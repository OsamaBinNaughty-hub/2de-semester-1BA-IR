\documentclass[a4paper,12pt]{article}

\usepackage[dutch]{babel}
\usepackage{amsmath}
\usepackage[version=4]{mhchem}

\begin{document}
\title{Chemie I deel 3}
    \author{Dries Van den Brande \and Andreas Declerck}

    \maketitle
    \date{}
    \setcounter{section}{11}


    \section{Organische chemie: structuren en naamgeving}
    \subsection*{Inleiding}
    \begin{enumerate}
        \item Defineer: Organische chemie.
        \item Welke verbindingen kan een C-atoom aangaan? Wat is een direct vervolg hiervan? 
        \item Wanneer wordt het C-atoom \textbf{verzadigd} of \textbf{onverzadigd} genoemd?
        \item Defineer: koolwaterstoffen. Uit wat worden deze ontgonnen? Wat wordt er afgezondert bij een destilatie van dit product.
        \item Defineer: verzadigde koolwaterstoffen.
    \end{enumerate}

    \subsection*{Koolwaterstoffen}
    \begin{enumerate}
        \item Defineer: Alkanen. 
        \item Geef de namen van de lineaire alkanen (1-10). 
        \item Geef de verschillende "old school" grafische voorstellingen van \ce{C4H10}. Waarom zijn deze onpraktisch?
        \item Defineer: conformationele flexibiliteit. 
        \item Teken en benoem de mogelijke vertakte isomeren van butaan en pentaan. 
        \item Wanneer daalt het kookpunt/smeltpunt bij organische moleculen? 
        \item Wat is de algemene molecuulformule van een cyclische alkaan (1 ringstructuur)? Waarom is deze anders?
        \item Defineer: Onverzadiging van een cyclische alkaan. Wat moet men doen per ingevoerde onverzadiging?
        \item Geef alle mogelijke bindingspatronen van een verzadigd hexaan isomeer dat de formule \ce{C6H8} heeft. Hoeveel onverzadigingen heeft dit isomeer?
        \item Wat zijn de voorwaarden om het voorvoegsel "cyclo-" te gebruiken? Teken cyclopropaan /cyclobutaan /cyclopentaan /cyclohexaan.
        \item Hoe bekomt men een substituenten? Hoe stelt men de KWS voor met 1 substutuent?
        \item Wat zijn de voorwaarden om het achtervoegsel "-yl" te gebruiken?
        \item Teken en benoem de substituenten van:
                \begin{enumerate}
                    \item methaan / ethaan
                    \item propaan / cyclopropaan
                    \item butaan / isobutaan / cyclobutaan
                \end{enumerate}
        \item Wat zijn de voorwaarden om het voorvoegsel "sec-" en "tert-" te gebruiken?
        \item Defineer: primair en quaternair C-atoom. 
        \item 
    \end{enumerate}
\end{document}
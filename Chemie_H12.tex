\documentclass[a4paper,12pt]{article}

\usepackage[dutch]{babel}
\usepackage{amsmath}
\usepackage[version=4]{mhchem}
\usepackage{chemfig}
\usepackage{graphicx}
\usepackage{enumerate}

\makeatletter
\newcommand*\bigcdot{\mathpalette\bigcdot@{.5}}
\newcommand*\bigcdot@[2]{\mathbin{\vcenter{\hbox{\scalebox{#2}{$\m@th#1\bullet$}}}}}
\makeatother

\begin{document}
\title{Chemie I deel 3}
    \author{Dries Van den Brande \and Andreas Declerck}

    \maketitle
    \date{}
    \setcounter{section}{11}


    \section{Organische chemie: structuren en naamgeving}
    \subsection*{Inleiding}
    \begin{enumerate}
        \item Defineer: Organische chemie.
        \item Welke verbindingen kan een C-atoom aangaan? Wat is een direct vervolg hiervan? 
        \item Wanneer wordt het C-atoom \textbf{verzadigd} of \textbf{onverzadigd} genoemd?
        \item Defineer: koolwaterstoffen. Uit wat worden deze ontgonnen? Wat wordt er afgezondert bij een destilatie van dit product.
        \item Defineer: verzadigde koolwaterstoffen.
    \end{enumerate}

    \subsection*{Koolwaterstoffen}
    \begin{enumerate}
        \item Defineer: Alkanen. 
        \item Geef de namen van de lineaire alkanen (1-10). 
        \item Geef de verschillende 'old school' grafische voorstellingen van \ce{C4H10}. Waarom zijn deze onpraktisch?
        \item Defineer: conformationele flexibiliteit. 
        \item Teken en benoem de mogelijke vertakte isomeren van butaan en pentaan. 
        \item Wanneer daalt het kookpunt/smeltpunt bij organische moleculen? 
        \item Wat is de algemene molecuulformule van een cyclische alkaan (1 ringstructuur)? Waarom is deze anders?
        \item Defineer: Onverzadiging van een cyclische alkaan. Wat moet men doen per ingevoerde onverzadiging?
        \item Geef alle mogelijke bindingspatronen van een verzadigd hexaan isomeer dat de formule \ce{C6H8} heeft. Hoeveel onverzadigingen heeft dit isomeer?
        \item Wat zijn de voorwaarden om het voorvoegsel 'cyclo-' te gebruiken? Teken cyclopropaan /cyclobutaan /cyclopentaan /cyclohexaan.
        \item Hoe bekomt men een substituenten? Hoe stelt men de KWS voor met een substituent.
        \item Hoe wordt de basisnaam van een substituent bepaald?
        \item Teken en benoem de substituenten van:
                \begin{enumerate}
                    \item methaan / ethaan
                    \item propaan / cyclopropaan
                    \item butaan / isobutaan / cyclobutaan
                \end{enumerate}
        \item Wat zijn de voorwaarden om het voorvoegsel 'sec-' en 'tert-' te gebruiken?
        \item Defineer: primair en quaternair C-atoom. 
        \item Waarom moet de hoofdketen genummerd worden?
        \item Wanneer gebruikt men multiplicerende voorvoegsels?
        \item Hoe vernoem je radicalen afgeleid uit alkanen?
        \item Hoe worden meerdere gelijke samengestelde substituenten aangeduid? 
        \item Hoe duid men het aantal dubbele koolstof bindingen aan in de hoofdketen van een alkaan?
        \item Hoe duid men het aantal meervoudige bindingen aan in de hoofdketen van een alkaan? 
        \item Defineer: Aromatische koolwaterstoffen. 
        \item Defineer: Digesubstitueerde benzeenringen. 
        \item Wat zijn de voorwaarden om de voorvoegsels 'o-' / 'm-' / 'p-' te gebruiken?
        \item Benoem de organische structuren:
            \begin{itemize}
                \item $$\chemfig[angle increment=30]{\ce{H3C}-[-1]\ce{CH2}-[1]\ce{CH2}-[-1]\ce{CH2}-[1]\chemabove{\ce{CH}}{\bigcdot}-[-1]\ce{CH2}}$$
                \item $$\chemfig{\ce{H2C}=C=\ce{CH2}}$$
                \item $$\chemfig[angle increment=30]{\ce{H2C}=C(-[1]\ce{CH3})-[-1]CH=\ce{CH2}}$$
                \item $$\chemfig[]{*6(--=---)}$$
                \item $$\chemfig[]{\ce{H2C}=CH\bigcdot{}}$$
                \item $$\chemfig[angle increment=30]{\ce{H2C}=CH-[1]\chemabove{\ce{CH2}}{\bigcdot}}$$
                \item $$\chemfig[]{HC~CH}$$
                \item $$\chemfig[]{HC~C\bigcdot}$$
                \item $$\chemfig{*8(-=-=-=-=)}$$
                \item $$\chemfig[angle increment=30]{=[1]-[-1]-[1]~[-1]}$$
                \item $$\chemfig{*6(=-*6(-=-=-)=-=-)}$$
                \item $$\chemfig{*6(=-*6(=-*6(-=-=--)=-=)--=-)}$$
                \item $$\chemfig[]{[:-30]*6(-=-*6(-*6(-=-=--)=-=--)=-=)}$$
                \item $$\chemfig[]{*6(-=-(-=[:-30])=-=)}$$
                \item $$\chemfig[]{*6(-=-(-)=-=)}$$
                \item $$\chemfig[]{*6(-=-(-)=(-)-=)}$$
                \item $$\chemfig[]{*6(-=-(-\chemabove{\ce{CH2}}{\bigcdot{}})=-=)}$$
                \item $$\chemfig[]{*6(-=-(\chemabove{}{\bigcdot{}})=-=)}$$
                \item $$\chemfig[]{*6(-=-(-\chemabove{\ce{CH3}}{})=(\chemabove{}{\bigcdot{}})-=)}$$
            \end{itemize}
        \item Teken de organische structuren: (oplossingen staan in volgorde in het boek)
            \begin{itemize}
                \item 3-methylpentaan
                \item methylcyclohexaan
                \item 2,3,3,4-tetramethylpentaan
                \item 5,5-di-t-butylnonaan
                \item 2-methylpentaan
                \item 2,3,5-trimethylhexaan
                \item 2,7,8-trimethyldecaan
                \item 4-isopropyl-5-porpyloctaan
                \item 3-ethyl-2-methylheptaan
                \item 5-methyl-4-propylnonaan
                \item 1-methylpentyl
                \item 2-methylpentyl
                \item 3-methylpentyl
                \item 4-methylpentyl
                \item 4,4-diethyloctaan
                \item 5,6-bis(1,1-dimethylpropyl)decaan
                \item 2-hexeen
                \item 1,4-hexadieen
                \item alleen 
                \item isopreen
                \item cyclohexeen
                \item vinyl
                \item allyl
                \item butenyn
                \item actyleen
                \item ethynyl
                \item 3-penteen-1-yn                                                                                                                                                   
                \item 1,4-cyclohexadieen
                \item cyclooctaterta"een
                \item 1,3-hexadieen-5-yn
                \item 1-penteen-4-yn
                \item benzeen
                \item naftaleen
                \item anthraceen
                \item fenantreen
                \item o-benzeen / m-benzeen / p-benzeen
                \item 1-ethyl-4-methylbenzeen
                \item 1,2,3-trimethylbenzeen
                \item streen
                \item tolueen
                \item o-xyleen
                \item benzyl
                \item fenyl
                \item o-tolyl
            \end{itemize}
    \end{enumerate}


\end{document}
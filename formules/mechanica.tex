
\documentclass[12pt]{article}

\usepackage[dutch]{babel}
\usepackage{amsmath}
\usepackage{amssymb}
\usepackage{physics}

\title{Formularium}
\author{Dries Van den Brande \and Andreas Declerck}

\newcommand{\R}{\mathbb{R}}
\newcommand{\Z}{\mathbb{Z}}


\begin{document}
    \maketitle
	
    \section{Mechanica}
    \label{sec:mechanica}
    
    \begin{enumerate}
	    \item Perkenwet: $ r^2 \dot{\theta} = cte $
	    \item Evolutie straal centrale krachten: $ r = \frac{p}{1\pm e \cos{(\theta - \theta_0^{*} )}} $
		    \begin{itemize}
			    \item $ p = \frac{b^2}{a} = - \frac{mC^2}{k} $
			    \item  $ e = \frac{c}{a} = \sqrt{ 1 + \frac{2mC^2 E_0}{k^2} }$
		    \end{itemize}
    	    \item Geostationaire oplossing: $ v_0 = \sqrt{ \frac{gR^2}{r_0} }$
	    \item Kromme: $ \rho = \sqrt{x''^2 + y''^2 + z''^2 }\ (met\ '=\dv{}{s}) $
	    \item Normaal: $ \vec{1}_n = \rho \dv{\vec{1}_t}{s} $
	    \item Snelheid circulaire slinger: $ v = R \dot{\theta} $
	    \item Cyclo\"idale beweging: 
		    \begin{itemize}
			    \item $x = R{(\alpha - \sin{\alpha} )} $
			    \item $y = R{(1 - \cos{\alpha} )} $
		    \end{itemize}
	    \item Snelheid cyclo\"idale beweging: $ v = 2 R \dot{\alpha} \sin{\frac{\alpha}{2}} $
	    \item Baanlengte bij t=T: $ s(T) = 4R{(1-\cos{\frac{\omega t}{2}})} = 8R$
	    \item Kromtestraal cyclo\"idale beweging:  $ \rho = 4R \sin{\frac{\omega t}{2}} $
	    \item Periode cyclo\"idale beweging:  $ T = \frac{2\pi}{\omega} = 4\pi \sqrt{\frac{R}{g}}$
	    \item Versnelling in relatieve assen:  $ \vec{a} = \vec{a}' + 2\vec{\omega} \cross \vec{v}' + \dv{\vec{\omega}}{t} \cross \vec{r}' + \vec{\omega} \cross {(\vec{\omega} \cross \vec{r}' )} + \vec{a}_0 $
	    \item Sleepversnelling aarde: $ \vec{a}_s = -\vec{OA} \omega^2 $
	    \item Moment koppel: $ \vec{C}_p = \vec{A_1 A_2} \cross \vec{v}_2 = -\vec{A_1 A_2} \cross \vec{v}_1$
	    \item Overzetten van moment: $ \vec{C}_Q = \vec{C}_P + \vec{QP} \cross \vec{R}_P $
	    \item Alternatief voor overzetten moment: $ \vec{C}_Q \cdot \vec{1}_R = \vec{C}_P \cdot \vec{1}_R $
	    \item Vgl centrale as: $ \vec{OG} = \frac{\vec{R}\cross\vec{C}_P}{R^2} + \lambda \vec{R} $
	    \item Definitie massamiddelpunt: $ \sum\limits_i m_i \vec{GA_i} = 0 $
	    \item 1e formule van Guldin: $ \sum = 2\pi L y_G $
	    \item 2e formule van Guldin: $ \sum = 2\pi S y_G $
	    \item Stribeckwrijving: $ F_W = A\ sgn{(v)} + B v^C $
	    \item Formule statische wrijvingscoeff: $ f_s = \tan{(\alpha)} $
	    \item Stelling virtuele arbeit; $ \sum\limits_i Q_i \delta q^i = 0 $
    \end{enumerate}
    
\end{document}

\documentclass[12pt]{article}

\usepackage[dutch]{babel}
\usepackage{amsmath}
\usepackage{amssymb}
\usepackage{physics}

\title{Formularium}
\author{Dries Van den Brande \and Andreas Declerck}

\newcommand{\R}{\mathbb{R}}
\newcommand{\Z}{\mathbb{Z}}


\begin{document}
    \maketitle

	\section{Thermodynamica}%
	\label{sec:Thermodynamica}
	\begin{enumerate}
		\item Ideale gaswet: $ PV = nRT $
		\item Ideale gasconstante:  $ R = 8.314 \frac{J}{mol K} $
		\item Definitie temperatuur: $ T = \lim\limits_{P \to 0} \left(\frac{PV}{R}\right) $
		\item Compressibiliteitsfactor: $ Z = \frac{PV}{RT} $
		\item Viriaal vergelijking:  $ Z = 1 + \frac{B(t)}{V} + \frac{B(t)}{V^2} + \dots $
		\item Definitie Boyle-temperatuur: $ B(T_B) = 0 $
		\item Isotherme compressibiliteit:  $ \kappa = \frac{-1}{V}\left( \pdv{V}{p} \right)_{T_c} $
		\item Van der Waals vergelijking: $ (P + \frac{a}{V^2})(V-b) = RT $
		\item Kracht op piston:  $ dw = F\cdot dl = -P_{ext}\cdot A\cdot dl = -P_{ext}\cdot dV$
	\end{enumerate}
\end{document}

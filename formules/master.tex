
\documentclass[12pt]{article}

\usepackage[dutch]{babel}
\usepackage{amsmath}

\title{Mechanica 2}
\author{Dries Van den Brande \and Andreas Declerck}

\begin{document}
    \maketitle

    \section{Formules Analyse}
    \label{sec:formules_analyse}
   
    \begin{enumerate}
	    \item Eerste formule Guldin: $ V = 2\pi S \bar{y}$
	    \item Formule Green-Riemann: $ \oint_{C^+} P(x, y)dx + Q(x, y)dy = \iint_G \pdv{Q}{x} - \pdv{P}{y} dO $
	    \item Integraal van Fresel: $ I = \int_0^{\infty} e^{-x^2/2} dx$
    	    \item Oppervlakte integraal:  $ \iint_R \| \pdv{\vec{r}}{u} \cross \pdv{\vec{r}}{v} \| $
            \item Tweede formule van Gulding: $ Opp(S) = 2\pi l \bar{x}$
	    \item Flux van vectorfunctie $ \vec{v}$: $ \Phi = \iint_S (\vec{v} \cdot \vec{n}) dO$
	    \item Flux normaal: $\vec{n} = \epsilon \frac{\pdv{\vec{r}}{u} \cross \pdv{\vec{r}}{v}}{\| \pdv{\vec{r}}{u} \cross \pdv{\vec{r}}{v} \| }$
	    \item Gradi\"ent: $ grad f = \grad f$
	    \item Divergent: $ div f = \div f $
	    \item Rotatie: $ rot f = \nabla \cross f $
	    \item Formule van Stokes:  $ \iint_S (rot \vec{ v}\cdot \vec{n}) dO = \oint_C \vec{v}(\vec{r}) \cdot d \vec{r} $
	    \item Oppervlakte parametrisatie in xy: $ \vec{r} = (x, y, h(x,y)); p = \pdv{h}{x}; q = \pdv{h}{y} $ $$ \iint_S f(x,y,z) dO = \iint_g f(x,y,z) \sqrt{1+p^2+q^2} dxdy  $$
	    \item Formule van Ostrogradsky: $ \iint_S (\vec{v} \cdot \vec{n})dO = \iiint_G \div \vec{v} dv$
    \end{enumerate}
    

    \section{Golven en Elektromagnetisme}%
    \label{sec:section_name}
    
    \begin{enumerate}
	    \item Intensiteit: $ I = \frac{E gedragen door golf}{tijd \cdot oppervlakte}$ 
	    \item Doppler effect mechanische golven: $ f' = f \cdot \frac{v+v_0}{v-v_s}$
	    \item Rel.\ doppler effect: $ f' = f \cdot \sqrt{\frac{c+v_r}{c-v_r}} $
	    \item Mach getal:  $ M = \frac{v_s}{v} $
	    \item Permittivitewit van vacu\"um: $ \frac{1}{4\pi\epsilon_0} = 9* 10^9 Nm^2/C^2 $
	    \item permeabiliteit van vacu\"um: $ \frac{\mu_0}{4\pi} = 10^{-7} kg*m/C^2 $
	    \item Wet van Gauss (globaal): $ \oint_S \vec{E} dA = \frac{Q_{netto,in}}{\epsilon_0} $
	    \item Wet van Gauss (locaal): $ \div \vec{E} = \frac{ \vec{j}_{alle}}{\epsilon_0} $
	    \item $ \vec{E}$-veld op geleider: $ \vec{E} = \frac{\sigma_{el}}{\epsilon_0} \cdot \vec{n}_{uitw} $
	    \item $ \vec{E}$-veld op oppervlakte: $ \vec{E} = \frac{\sigma}{2 \epsilon_0} \vec{1_y} $
	    \item Capaciteit: $ C = \kappa \frac{Q}{|\Delta V|} $
	    \item Potentiaal $ \vec{E}$-veld: $ \Delta V = -\int_0^d E(r) dr $
	    \item eV bij $q_0$ en $\Delta V$ in SI-waarden:  $|\Delta U_E| = |q_0 \cdot \Delta V| = |\Delta E_{kin}| = 1 eV = 1.602 \cdot 10^{-19} J $
	    \item Elektrisch dipool moment: $ \vec{p} = Q \cdot d \cdot \vec{1}_d $
	    \item Torge dipool in uniform E-veld: $ \vec{\tau} = \vec{r}\cross\vec{F} = \vec{p} \cross \vec{E} $
	    \item Potentiaal dipool: $ U_{dipool} = - \vec{p} \cdot \vec{E} $
	    \item Magnetisch moment: $ \vec{\mu}_m = I \cdot A \cdot \vec{1}_x $
	    \item Energie in condensator: $ U_e = \epsilon_0 \kappa \frac{1}{2} A d \cdot |\vec{E}|^2 $
	    \item Energiedichteheid condensator: $ u_E = \dv{U_e}{d} = \epsilon_0 \kappa \frac{1}{2} A |\vec{E}|^2 $
	    \item Cyclotron pulsatie: $ \omega_c = \frac{|q||\vec{B}|}{m} $
	    \item Larmorstraal: $ L_R = \frac{v_{\perp}}{\omega_c} $
	    \item Deeltjesfilter:  $v_y = \frac{|\vec{E}|}{|\vec{B}|} $
	    \item Elektrische stroom: $ I = \lim\limits_{\Delta t \to 0} \frac{ \Delta Q}{\Delta t} $
	    \item Stroomdichtheid (def 1):  $ I = \iint\limits_A \vec{j}(\vec{r})\cdot \vec{dA} $
	    \item Stroomdichtheid (def 2): $ \vec{j} = n_e q_e \vec{v}_d $
	    \item Wet van Ohm: $\Delta V = I \cdot R $
	    \item Wet van Pouillet:  $ R = \rho \frac{L}{A} $
	    \item Wet Biot-Savard: $ \vec{B}(\vec{r}) = \frac{\mu_0}{4\pi} \int\limits_{geleider} \frac{d\vec{l}\cross (\vec{r}-\vec{r}')}{|\vec{r}-\vec{r}'|^3}$
	    \item Wet van Laplace: $ \vec{F} = I \int d\vec{l}\cross \vec{B} $
	    \item Krachtwet van Amp\`ere: $ \vec{F} = \frac{\mu_0}{2\pi}\frac{I_1 I_2}{D}L $
	    \item Biot-Savard oneindige lange geleider: $ \vec{B}(\vec{r}) = \frac{\mu_0}{2\pi}\frac{I}{|\vec{r}|}\vec{1}_t$
	    \item Machnetisch veld in spoel: $ B = \mu_0 \frac{N}{L} I $
	    \item EMK generator: $ \epsilon = A |\vec{B}| \omega \sin{\omega t}$
	    \item Wederzijdse inductantie 2 gelijke spoelen (Henri): $ M = \mu_0 \pi \frac{N^2 R^2}{l} $
	    \item Zelfinductantie: $ L = \\mu_0\pi \frac{N^2R^2}{l} $
	    \item Energie spoel (analoog energie condensator): $ U_B = \frac{1}{2} LI^2 $
	    \item Magnetische energiedichtheid: $ u_B = \frac{1}{2} \frac{|\vec{B}|^2}{\mu_0} $
	    \item Totale energie in LC-kring: $ U = \frac{Q_0^2}{2C} $
	    \item Verschuivingsstroom: $ I_D = \epsilon_0 \dv{\Phi_E}{t} $
	    \item Vector van Poynting (stroomdichtheid): $ \vec{S} = \epsilon_0 c^2 (\vec{E}\cross\vec{B})$
	    \item Stelling van Poynting: $ \pdv{\rho_{EM}}{t} = -\div \vec{S} - \vec{E} \cdot \vec{j}_{alle} $
	    \item EM energie dichtheid: $ \rho_{EM} = \frac{\epsilon_0 c^2}{2}|\vec{B}|^2 + \frac{\epsilon_0}{2}|\vec{E}|^2 $
	    \item $<Irr>$: $ <|\vec{S}|> = \frac{1}{2} \epsilon_0 c |\vec{E}|^2 = c <\rho_{EM}> \vec{1}_x $
	    \item Hoofdmaxima: $ d \sin\theta = m\lambda $
	    \item Optisch weglengteverschil: $ \delta = r_1 - r_2 $
	    \item Verplaatsingswet van Wien: $ \lambda_{max} \cdot T = \gamma $
	    \item Drempelfunctie:  $ \nu_0 = \frac{W}{h} $
	    \item Stoppotentiaal: $ V_0 = \frac{h}{e} (\nu - \nu_0) $
	    \item Impuls van een foton: $ p = \frac{E}{c} = \frac{h}{\lambda} $
	    \item Compton effect: $ \lambda_2 = \lambda_1 + \frac{h}{m_e c} (1-\cos\theta) $
	    \item Relativistische energie: $ E_{tot} = E_{kin} + E_{rust} = mc^2 = (\gamma-1)m_0c^2 + m_0c^2 $
	    \item Barn: $ 1 barn = 10^{-24}cm^2 = 10^{-28}m^2 $
	    \item Wet van Beer-Lambert:  $ d\Phi = -I \cdot \Phi \to \Phi(x) = \Phi_0 \cdot e^{-\mu x} $
	    \item Halfwaarde tijd: $ T_{1/2} = \frac{\ln 2}{\lambda} = \frac{0.693}{\lambda} $
    \end{enumerate}

\end{document}

\documentclass[12pt]{article}

\usepackage[dutch]{babel}
\usepackage{amsmath}

\title{Mechanica 2}
\author{Dries Van den Brande \and Andreas Declerck}

\begin{document}
    \maketitle

	\section{Hoofdstuk 1: Hoofdwet van de toestands vergelijking}%
	\label{sec:Hoofdstuk_1}
	\begin{enumerate}
		\item Geef de definitie van \emph{thermodynamica}.
		\item Geef de definitie van het \emph{systeem}.
		\item Definieer de \emph{omgeving van een systeem} en in de context van de ingenieur.
		\item Geef 2 onderverdelingen van systemen (1 met 3 soorten systemen en 1 met 2 soorten systemen)
		\item Geef 3 thermodynamische systemen.
		\item Geef de definitie van \emph{intensieke variabelen}?
		\item Geef de definitie van \emph{extensieve variabelen}.
		\item Geef de definitie van een \emph{systeem bij evenwicht}.
		\item Geef de \emph{nulde hoofdwet}.
		\item Geef de \emph{wet van Boyle}.
		\item Geef de \emph{wet van Charles \& Gay-Lussac}.
		\item Geef de \emph{ideale gaswet}.
		\item Geef de \emph{symbolische definitie van de temperatuur}.
		\item Geef de definitie van de \emph{Kelvin}.
		\item Geef de \emph{wet van Dalton}.
		\item Geef de \emph{compressibilitietsfactor Z}.
		\item Geef 2 oorzaken voor de afwijking van re\"ele gassen ten opzichte van ideale gassen.
		\item Wat gebeurt er als $T$ laag genoeg is?
		\item Geef de \emph{toestandsvergelijking van KamerLingen-Onnes} of de \emph{viriaal vergelijking}.
		\item Wat is de vergelijking van de \emph{Boyle temperatuur}?
	\end{enumerate}

    \section{Algemene info}%
    \label{sec:Algemene_info}
    \begin{itemize}
    	\item Docent: Guy Van Assche
		\item Lokaal: G.8.510
		\item Tel: 02/629.39.41
		\item Examenvragen te vinden op \emph{Canvas}.
		\item Formularium moet van buiten.
    \end{itemize}
\end{document}

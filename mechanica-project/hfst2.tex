\documentclass[12pt]{article}

\usepackage[dutch]{babel}
\usepackage{amsmath}
\usepackage{amssymb}
\usepackage{physics}

\title{Formularium}
\author{Dries Van den Brande \and Andreas Declerck}

\newcommand{\R}{\mathbb{R}}
\newcommand{\Z}{\mathbb{Z}}


\begin{document}
    \maketitle

    \section{Hfst 2: Eindige verplaatsing v/h star lichaam}%
	\label{sec:Hfst_2:_Eindige_verplaatsing_v/h_star_lichaam}
    
	\begin{enumerate}
		\item Wat wordt er bedoeld met \emph{equipollente vectoren}?
		\item Definieer \emph{verplaatsingsamplitude}.
		\item Definieer \emph{Translatiefunctie} in symbolen.
		\item Geef 6 eigenschappen over de translatiefunctie.
		\item Definieer de \emph{rotatie}.
		\item Bespreek de parameters in een \emph{rotatie functie}.
		\item Bewijs dat een punt Q dat op de rotatie as ligt altijd op zichzelf zal worden afgebeeld.
		\item Geef 2 eigenschappen van de rotatie.
		\item Leg in symbolen de \emph{samenstelling van 2 roaties rond eenzelfde d} uit.
		\item Geef 3 eigenschappen van de rotatie rond 2 verschillende assen.
		\item Leg de rotatie uit bij 2 evenwijdige assen.
		\item Wat heb je bij een \emph{rotatie met evenwijdige assen} waar de 2 hoeken elkaars tegengestelde zijn. Bereken ook de \emph{amplitude}.
		\item Bewijs dat de rotatie van stelsels een \emph{isomorphisme} is.
		\item Bewijs dat \emph{product-rotatie} gelijk is aan het product van de \emph{rotatie-matrices}.
	\end{enumerate}
\end{document}
